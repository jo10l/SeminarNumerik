%
% reduktion.tex
%
% (c) 2020 Prof Dr Andreas Müller, Hochschule Rapperswil
%
\begin{frame}
\frametitle{Reduktion der Ordnung}
\begin{block}{Differentialgleichung der Ordnung $n$}
\vspace{-15pt}
\begin{gather*}
y^{(n)} = f(x,y,y',\dots,y^{(n-2)},y^{(n-1)})
\\
\uncover<2->{
y^{(n-1)}(0) = g_{n-1},\quad
y^{(n-2)}(0) = g_{n-1},
\dots,\quad
y'(0) = g_1, \quad
y(0) = g_0}
\end{gather*}
\end{block}
\uncover<3->{%
\begin{block}{Reduktion}
Neue Funktionen $y_0(x)=y(x)$, $y_1(x)=y'(x)$,\dots $y_{n-1}(x)=y^{(n-1)}(x)$
mit
\begin{center}
\begin{tikzpicture}[>=latex,thick]

\fill[color=gray!20] (-6.7,-1.5) rectangle (0.6,1.5);
\fill[color=gray!20] (0.7,-1.5) rectangle (6.3,1.5);

\node at (-3,1.7) {Gleichungen};
\node at (3.3,1.7) {Anfangsbedingungen};

\only<5-6>{
\node at (-3,0) {$\displaystyle
	\begin{aligned}
		{\textstyle\frac{d}{dx}}y_0(x) & = y_0'(x)=y_1(x) \\
		{\textstyle\frac{d}{dx}}y_1(x) & = y_1'(x)=y''(x)=y_2(x) \\
		\vdots \quad &\phantom{=}\quad\vdots\\
		{\textstyle\frac{d}{dx}}y_{n-1}(x)
			&= y^{(n)}(x) =f(x,y_0,y_1,\dots,y_{n-1})\\
	\end{aligned}
	$};
}

\only<7-8>{
\node at (-3,0) {$\displaystyle
	\frac{d}{dx}\begin{pmatrix}y_0\\y_1\\\vdots\\y_{n-1}\end{pmatrix}
	=
	\begin{pmatrix}
	y_1\\y_2\\\vdots\\f(x,y_0,y_1,\dots,y_{n-1})
	\end{pmatrix}
	$};
}

\only<9->{
\node at (-3,0) {$\displaystyle
	\frac{d}{dx} \vec{y} = F(x,\vec{y})
	$};
}

\only<6-7>{
\node at (3.3,0) {$\displaystyle
	\begin{aligned}
	y_0(0) &= g_0\\
	y_1(0) &= g_1\\
	\vdots\quad &\qquad\vdots\\
	y_{n-1}(0) &= g_{n-1}
	\end{aligned}
	$};
}

\only<8-9>{
\node at (3.3,0) {$\displaystyle
	\begin{pmatrix}
	y_0(0)\\
	y_1(0)\\
	\vdots\\
	y_{n-1}(0)
	\end{pmatrix}
	=
	\begin{pmatrix}
	g_0\\g_1\\\vdots\\g_{n-1}
	\end{pmatrix}
	$};
}

\uncover<10->{
\node at (3.3,0) {$\displaystyle
	\vec{y}(0) = \vec{g}
	$};
}

\uncover<11->{
\node at (-3,-1.1) {\usebeamercolor[fg]{title}Ordnung $1$};
\node at (3.3,-1.1) {\usebeamercolor[fg]{title}Dimension $n$};
}

\end{tikzpicture}
\end{center}
\end{block}}
\end{frame}
