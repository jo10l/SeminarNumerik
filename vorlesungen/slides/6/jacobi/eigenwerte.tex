%
% eigenwerte.tex
%
% (c) 2020 Prof Dr Andreas Müller, Hochschule Rapperswil
%
\begin{frame}
\frametitle{EW/EV symmetrischer Matrizen}
\begin{columns}[t]
\begin{column}{0.48\hsize}
\begin{block}{Aufgabe}
$A$ eine symmetrische $n\times n$-Matrix, finde
EW und EV
\end{block}
\uncover<2->{%
\begin{block}{Fakten}
\begin{itemize}
\item<3-> Eigenwerte sind reell
\item<4-> Eigenvektoren sind orthogonal
\item<5-> Es gibt eine orthogonale Matrix $T$ mit $TAT^{-1}=TAT^t=
\operatorname{diag}(\lambda_1,\dots,\lambda_n)=D$
\end{itemize}
\end{block}}
\end{column}
\begin{column}{0.48\hsize}
\uncover<6->{%
\begin{block}{Eigenvektoren}
\vspace{-10pt}
\[
A = T^t\begin{pmatrix}
\lambda_1&    0    &\dots &   0   \\
    0    &\lambda_2&\dots &   0   \\
 \vdots  & \vdots  &\ddots&\vdots \\
    0    &    0    &\dots &\lambda_n
\end{pmatrix}T
=
T^tDT
\]
\uncover<7->{%
Eigenvektoren sind Spalten von $T^t$.
Spalte $i$: $v_i$}
\begin{align*}
\uncover<8->{T^tDTv_i}
&\uncover<9->{=
T^tDe_i}
\uncover<10->{=
T^t\lambda_ie_i}
\uncover<11->{=
\lambda_iv_i}
\end{align*}
\end{block}}
\end{column}
\end{columns}
\end{frame}
