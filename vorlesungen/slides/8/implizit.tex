%
% implizit.tex
%
% (c) 2020 Prof Dr Andreas Müller, Hochschule Rapperswil
%
\begin{frame}
\frametitle{Implizite Verfahren}
\vspace{-15pt}
\begin{columns}[t]
\begin{column}{0.48\hsize}
\begin{block}{Zweite Ableitungen}
\[
\frac{\partial^2 u}{\partial x^2}(x_i,t_j) = Au_j
\]
\end{block}
\begin{block}{Gleichungen}
Crank-Nicolson:
\[
{\textstyle\frac12}(Au_j + Au_{j+1}) = u_{j+1}-u_j
\]
d.~h.~$u_j$ implizit definiert
\end{block}
\begin{block}{Auflösen nach $u_j$}
\vspace{-15pt}
\[
u_{j+1}
=
(E-{\textstyle\frac12}A)^{-1}
(E+{\textstyle\frac12}A)u_j
\]
\end{block}
\end{column}
\begin{column}{0.48\hsize}
\begin{block}{Konvergenz}
Explizite Darstellung für Zeitschritt:
\[
u_{j+1} = Bu_j
%\]
%Lösung durch Iteration
%\[
\quad\Rightarrow\quad
u_j = B^j u_0
\]
Spektralradius: Betrag des betragsgrössten Eigenwert
\[
\varrho(B) = \max\{ |\lambda|\;| \text{$\lambda$ EW von $B$}\}
\]
Stabilität:
\begin{align*}
\varrho(B) &\le 1 &&\text{stabil}\\
\varrho(B) &> 1 &&\text{instabil}
\end{align*}

\end{block}
\end{column}
\end{columns}

\end{frame}
