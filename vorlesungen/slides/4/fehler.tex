%
% fehler.tex
%
% (c) 2020 Prof Dr Andreas Müller, Hochschule Rapperswil
%
\begin{frame}
\frametitle{Fehler der Trapezformel}
\begin{block}{Fehlerformel}
\[
\int_a^b f(x)\,dx
=
T(h) + h^2\cdot\frac{f'(b)-f'(a)}2 + O(h^4)
\]
\uncover<2->{%
Beweis: Euler-Maclaurin-Summenformel}
\end{block}

\uncover<3->{
\begin{block}{Rechenaufwand für Genauigkeit $\varepsilon$}
\[
h^2 \le \varepsilon
\uncover<4->{
\qquad\Rightarrow\qquad}
\uncover<5->{
h=\frac{b-a}n = \sqrt{\varepsilon}}
\uncover<6->{
\qquad\Rightarrow\qquad}
\uncover<7->{
n=\frac{b-a}{\sqrt{\varepsilon}}}
\]
\uncover<8->{
$\Rightarrow$ Verdoppelung der Stützstellenzahl ergibt
2 Bit Genauigkeitsgewinn}
\end{block}}

\uncover<9->{
\begin{block}{Beispiel}
{\tt float}-Genauigkeit: $\varepsilon=2^{-23}$
\uncover<10->{$\Rightarrow$
$n\simeq 2^{12} = 4096$}
\end{block}
}

\end{frame}
