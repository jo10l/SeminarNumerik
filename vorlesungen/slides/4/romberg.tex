%
% romberg.tex
%
% (c) 2020 Prof Dr Andreas Müller, Hochschule Rapperswil
%
\begin{frame}
\frametitle{Konvergenz-Beschleunigung}
\vspace{-15pt}
\begin{columns}[t]
\begin{column}{0.42\hsize}
\begin{block}{Idee}
Fehlergesetz $\Rightarrow$ Fehler wegrechnen
\end{block}
\uncover<2->{%
\begin{block}{Fehler}
\vspace{-15pt}
\begin{align*}
I&=T(h) + Ch^2&&\uncover<3->{|\;\cdot (-1)}
\\
I&=T({\textstyle\frac{h}2}) + C\frac{h^2}4 &&\uncover<3->{|\;\cdot 4}
\\
\uncover<4->{3I}&\uncover<4->{=4T({\textstyle\frac{h}2}) - T(h)}
\end{align*}
\end{block}}
\vspace{-10pt}
\uncover<5->{%
\begin{block}{Konvergenzverbesserung}
\[
T^*({\textstyle\frac{h}2})
=
\frac{4T({\textstyle\frac{h}2})-T(h)}3
\]
\end{block}}
\end{column}
\begin{column}{0.54\hsize}
\uncover<6->{%
\begin{block}{Allgemein}
\vspace{-15pt}
\begin{align*}
T^{*0}(h) &= T(h)
\\
T^{*k}(h) &= \frac{4^kT^{*(k-1)}({\textstyle\frac{h}2}) - T^{*(k-1)}(h)}{4^k-1}
\end{align*}
\end{block}}
\vspace{-10pt}
\uncover<7->{%
\begin{block}{Romberg-Schema}
%\vspace{-10pt}
\begin{center}
\begin{tabular}{>{$}c<{$}|>{$}c<{$}>{$}c<{$}>{$}c<{$}}
T(2^{\phantom{-}0})&               &                            &               \\
T(2^{-1})&\uncover<8->{T^*(2^{-1})}&                            &               \\
T(2^{-2})&\uncover<8->{T^*(2^{-2})}&\uncover<9->{T^{**}(2^{-2})}&               \\
T(2^{-3})&\uncover<8->{T^*(2^{-3})}&\uncover<9->{T^{**}(2^{-3})}&\uncover<10->{T^{***}(2^{-3})}\\[8pt]
O(h^2)   &\uncover<8->{O(h^4)     }&\uncover<9->{O(h^6)        }&\uncover<10->{O(h^8)}
\end{tabular}
\end{center}
\end{block}}
\end{column}
\end{columns}
\end{frame}
