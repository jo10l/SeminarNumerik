%
% rekursion.tex
%
% (c) 2020 Prof Dr Andreas Müller, Hochschule Rapperswil
%
\begin{frame}
\frametitle{Rekursion}
\begin{block}{Satz: Verfeinerung $h\to\frac{h}2$ der Trapezregel}
\vspace{-5pt}
\uncover<2->{%
\[
T\biggl(\frac{h}2\biggr)
=
{\textstyle\frac12}\bigl(
T(h) + M(h)
\bigr)
\]}
\end{block}
\vspace{-20pt}
\uncover<3->{%
\begin{proof}[Beweis]
Mit $h = (b-a)/n$, $x_k = a + \frac{h}2k$, $k=0,1,\dots,2n$:
\begin{align*}
\uncover<4->{
T({\textstyle\frac{h}2})}
&\uncover<4->{=
\frac{h}2\biggl(
\frac12f(x_0) + f(x_1) + f(x_2) + \dots +\frac12f(x_{2n})
\biggr)}
\\
&\uncover<5->{=
\frac12\biggl(
\frac12 f(x_0) + f(x_2) + \dots + \frac12f(x_{2n})
\biggr)h}
\\
&\qquad\qquad\qquad
\uncover<5->{+
\frac12\biggl(
f(x_1)+f(x_3)+\dots + f(x_{2n-1})
\biggr)h}
\\
&\uncover<6->{=\frac12 T(h) + \frac12 M(h)}
\qedhere
\end{align*}
\end{proof}}

\end{frame}
