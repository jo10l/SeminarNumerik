%
% mechanik.tex
%
% (c) 2020 Prof Dr Andreas Müller, Hochsclue Rapperswil
%
\begin{frame}
\frametitle{Mechanik als Variationsproblem}
\vspace{-15pt}
\begin{columns}[t]
\begin{column}{0.48\hsize}

\begin{block}{Lagrange-Funktion}
Differenz von kinetischer und potentieller Energie:
\begin{align*}
L(t,x,v)
&=
E_{\text{kin}} - E_{\text{pot}}
\\
&\uncover<2->{=
\frac{1}{2}mv^2 - V(x)}
\end{align*}
\end{block}
\vspace{-5pt}

\uncover<3->{%
\begin{block}{Prinzip der kleinsten Wirkung}
Das System wählt den Weg $x(t)$ mit
\[
\int_a^b L(t,x(t),\dot x(t))\,dt
\;
\text{minimal}
\]
(Maupertuis)
\end{block}}

\end{column}
\begin{column}{0.48\hsize}
\uncover<4->{%
\begin{block}{Euler-Lagrange-Gleichung}
Einsetzen:
\begin{align*}
\uncover<5->{\frac{\partial L}{\partial x}}&\uncover<5->{= -V'(x) = F(x)}
\\
\uncover<6->{\frac{\partial L}{\partial v}}&\uncover<6->{= mv = \text{Impuls}}
\\
\uncover<7->{\frac{d}{dt}\frac{\partial L}{\partial v} = ma}
&\uncover<7->{=
\frac{\partial L}{\partial x}
=
F}
\end{align*}
\uncover<8->{%
$\Rightarrow$ Newtonsches Gesetz}
\end{block}}
\uncover<9->{%
\begin{block}{Allgemein}
Naturgesetze lassen sich als Minimalprinzipien formulieren
\end{block}}
\end{column}
\end{columns}
\end{frame}
