%
% zweiteabl.tex
%
% (c) 2020 Prof Dr Andreas Müller, Hochschule Rapperswil
%
\begin{frame}
\frametitle{Zweite Ableitungen}
\begin{block}{Teilinterval $(x_{i},x_{i+1})$}
\vspace{-5pt}
\[
[g''(x)h'(x)]_{x_i}^{x_{i+1}}
=
g''(x_{i+1}\uncover<2->{-})h'(x_{i+1})
-
g''(x_{i}\uncover<2->{+})h'(x_{i})
\]
\end{block}
\vspace{-15pt}
\uncover<3->{%
\begin{block}{Summe über alle Teilintervalle}
\vspace{-10pt}
\begin{center}
\begin{tikzpicture}[>=latex,thick]
\uncover<5->{
	\fill[color=red!20] (-1.30,-1.2) rectangle (1.55,0.0);
}
\node at (0,0) {$\displaystyle
\begin{aligned}
0
&=
\sum_{i=0}^{n-1}
[g''(x)h'(x)]_{x_i}^{x_{i+1}}
\\
&\uncover<4->{=
\dots
-
g''(x_{i-1}\uncover<2->{+})h'(x_{i-1})
+
g''(x_{i}\uncover<2->{-})h'(x_{i})}
\\
&\uncover<4->{%
\phantom{\mathstrut=
\dots
-
g''(x_{i-1}\uncover<2->{+})h'(x_{i-1})
}
-
g''(x_i\uncover<2->{+})h'(x_i)
+
g''(x_{i+1}\uncover<2->{-})h'(x_{i+1})
+ \dots}
\end{aligned}$};
\end{tikzpicture}
\end{center}
\end{block}}%
\vspace{-20pt}
\uncover<6->{%
\begin{block}{Spezielle Wahl von $h(x)$}
\vspace{-15pt}
\[
h'(x_i)=\delta_{ij}
\uncover<7->{
\quad\Rightarrow\quad
{\color{red}
\bigl(g''(x_j-)-g''(x_j+)\bigr) h'(x_j)} = 0}
\uncover<8->{\quad\Rightarrow\quad
g''(x_j-)=g''(x_j+) }
\]
\uncover<9->{$g''$ ist stetig!}
\uncover<10->{Ausserdem: $g''(x_0+)=0$, $g''(x_n-)=0$}
\end{block}}%

\end{frame}
