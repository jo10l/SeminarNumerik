%
% lagrange.tex
%
% (c) 2020 Prof Dr Andreas Müller, Hochschule Rapperswil
%
\begin{frame}
\frametitle{Interpolationspolynom}
\begin{block}{Stützstellen $x_0,x_1,\dots,x_n$ $\rightarrow$ Nullstellen}
\[
l(x)
=
(x-x_0)(x-x_1)(x-x_2)\cdots (x-x_{n-1})(x-x_n)
\]
\end{block}
\uncover<2->{%
\begin{block}{Basisfunktionen $l_i(x)$
\uncover<4->{mit $l_i(x_k)=\delta_{ik}$}}
\vspace{-15pt}
\begin{align*}
l_i(x)
&
\ifthenelse{\boolean{presentation}}{
\only<3-4>{=
(x-x_0)(x-x_1)(x-x_2)\cdots\only<4->{(\widehat{x-x_i}) \cdots}(x-x_{n-1})(x-x_n)
}
}{}
\only<5->{
=
\frac{
(x-x_0)(x-x_1)(x-x_2)\cdots(\widehat{x-x_i}) \cdots (x-x_{n-1})(x-x_n)
}{
(x_i-x_0)(x_i-x_1)(x_i-x_2)\cdots(\widehat{x-x_i}) \cdots (x_i-x_{n-1})(x_i-x_n)
}}
\\
&\uncover<6->{=
\frac{
\displaystyle\prod_{k=0, k\ne i}^n (x-x_k)
}{
\displaystyle\prod_{k=0,k\ne i}^n (x_i-x_k)
}}
\uncover<7->{=
c_i (x-x_0)(x-x_1)\cdots(\widehat{x-x_i}) \cdots(x-x_n)}
\end{align*}
\end{block}}
\end{frame}
