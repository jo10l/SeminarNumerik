%
% gleichungen.tex
%
% (c) 2020 Prof Dr Andreas Müller, Hochschule Rapperswi
%
\begin{frame}
\frametitle{Gleichungen und Nullstellen}
\begin{columns}[t]
\begin{column}{0.48\hsize}
\begin{block}{Gleichungen}
Gegeben Funktionen $f,g\colon\mathbb R^n\to\mathbb R^n$, finde $x\in\mathbb R^n$
derart, dass
\[
f(x) = g(x).
\]
\end{block}
\uncover<2->{%
\begin{block}{Nullstellenproblem}
\[
F(x) = f(x)-g(x)
\]
\end{block}}
\end{column}
\begin{column}{0.48\hsize}
\begin{block}{Nullstellen}
Gegeben eine Funktion
$F\colon\mathbb R^n\to\mathbb R^n$, findet $x\in\mathbb R^n$
derat, dass
\[
F(x)=0.
\]
\end{block}
\uncover<3->{%
\begin{block}{Gleichungsproblem}
\[
f(x) = F(x),\; g(x)=0
\]
\end{block}}
\end{column}
\end{columns}
\uncover<4->{%
$\Rightarrow$ Es genügt, Nullstellen finden zu können.}
\end{frame}

