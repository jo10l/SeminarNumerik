%
% chapter.tex
%
% (c) 2020 Prof Dr Andreas Müller
%
\chapter{Interpolation\label{chapter:interpolation}}
\lhead{Interpolation}
\rhead{}
Der Satz von Stone-Weierstrass garantiert, dass jede stetige Funktione
auf einem Intervall beliebig genau durch Polynome approximiert werden
kann.
\index{Satz von Stone-Weierstrasse}%
\index{Stone-Weierstrass!Satz von}%
Polynome sind effizient berechenbar, es ist daher naheliegend,
komplizierte Funktionen durch Polynome zu approximieren, die möglichst
viele für die vorliegende Anwendung relevante Eigenschaften mit der
Funktion gemeinsam habe.
\index{Polynom}%

Leider sagt der Satz von Stone-Weierstrass nichts darüber, wie solche
Polynome gefunden werden könnten.
Ziel dieses Kapitels ist daher, einige Möglichkeiten zusammenzustellen,
solche Approximationspolynome zu finden und insbesondere auch ihre
Fehler abzuschätzen.
\index{Fehler}%

%
% polygon.tex -- Basis aus Polygonzügen
%
% (c) 2020 Prof Dr Andreas Müller, Hochschule Rapperswil
%
\documentclass[tikz]{standalone}
\usepackage{amsmath}
\usepackage{times}
\usepackage{txfonts}
\usepackage{pgfplots}
\usepackage{csvsimple}
\usetikzlibrary{arrows,intersections,math}
\begin{document}
\def\skala{1.95}
\def\yskala{0.6}

\begin{tikzpicture}[>=latex,thick,scale=\skala]

\def\bild#1#2{
\begin{scope}[yshift=#1]

\draw[->] (-0.4,0)--(6.5,0) coordinate[label={$x$}];
\draw[->] (0,-0.35)--(0,0.85) coordinate[label={right:$y$}];

\begin{scope}
\clip (-0.0,-0.4) rectangle (6.0,0.75);
\draw[color=red,line width=1.4pt] (-2,0)--({#2-1},0)--({#2},\yskala)--({#2+1},0)--(8,0);
\end{scope}

\fill[color=red] (#2,\yskala) circle[radius={0.08/\skala}];
\draw ({-0.1/\skala},\yskala)--({0.1/\skala},\yskala);
\node at ({-0.1/\skala},\yskala) [left] {$1$};
\node at (2.5,\yskala) [above] {$j=#2$};
\foreach \x in {1,...,6}{
	\draw (\x,{-0.1/\skala})--(\x,{0.1/\skala});
	\node at (\x,{-0.1/\skala}) [below] {$\x$};
}
\node at ({-0.1/\skala},{-0.1/\skala}) [below left] {$0$};
\end{scope}
}

\bild{0cm}{0}
\bild{-1.4cm}{1}
\bild{-2.8cm}{2}
\bild{-4.2cm}{3}
\bild{-5.6cm}{4}
\bild{-7.0cm}{5}
\bild{-8.4cm}{6}

\end{tikzpicture}
\end{document}


%
% polynom.tex
%
% (c) 2020 Prof Dr Andreas Müller, Hochschule Rapeprswil
%
\section{Interpolationspolynom
\label{buch:section:interpolationspolynom}}
\rhead{Lagrange-Interpolationspolynome}
In diesem Abschnitt wird das folgende Problem gelöst.
\begin{aufgabe}[Interplations-Polynom]
Gegeben Stützstellen
\[
a=x_0<x_1<x_2<\dots < x_{n-1}<x_n=b
\]
und Funktionswerte $f_i, 0\le i\le n$, finde ein Polynome $l(x)$
mit der Eigenschaft $l(x_i)=f_i$ für alle $i=0,1,\dots n$.
\end{aufgabe}
Gegeben sind also $n+1$ Bedingungen, die das Polynom erfüllen muss.
Abgesehen von trivialen Fällen wie dem Null-Polynom, muss ein Polynom
im Allgemeinen mindestens den Grad $n$ haben, damit alle 
Bedingungen durch geeignete Wahl der $n+1$ Koeffizienten erfüllt werden können.
Man könnte das Polynom nämlich in der Form
\[
p(x)
=
a_nx^n + a_{n-1}x^{n-1}+\dots+a_1x+a_0
\]
ansetzen und die Stützstellen einsetzen.
Lösung des Gleichungssystem
\begin{equation}
\begin{linsys}{5}
a_Nx_0^N &+& a_{N-1}x_0^{N-1} &+& \dots &+& a_1x_0 &+& a_0x_0 &=& f_0 \\[5pt]
a_Nx_1^N &+& a_{N-1}x_1^{N-1} &+& \dots &+& a_1x_1 &+& a_0x_1 &=& f_1 \\[5pt]
\vdots   & &    \vdots        & & \ddots& & \vdots & & \vdots & & \vdots\\[5pt]
a_Nx_n^N &+& a_{N-1}x_n^{N-1} &+& \dots &+& a_1x_n &+& a_0x_n &=& f_n 
\end{linsys}
\end{equation}
liefert dann die gesuchten Koeffizienten.
Dieser Weg ist allerdings sehr aufwendig, die Lösung eines linearen
Gleichungssystems mit dem Gauss-Algorithmus benötigt $O(n^3)$ Operationen.
Die sehr spezielle Struktur des Gleichungssystems sollte ermöglichen,
das Polynom $l(x)$ auf direkterem Weg zu ermitteln.

%
% Interpolationspolynom bestimmen
%
\subsection{Bestimmung des Interpolationspolynoms
\label{buch:section:interpolation:bestimmung}}
Das allgemeine Interpolationsproblem kann leicht gelöst werden, wenn
das folgende spezielle Interpolationsproblem gelöst ist.

\begin{aufgabe}[Spezielle Interpolationspolynome]
\label{buch:aufgabe:speziellesinterpolationsproblem}
Gegeben die Stützstellen
\[
a=x_0<x_1<x_2<\dots <x_{n-1}<x_n=b,
\]
finde Polynome $l_j$ vom Grad $n$ derart, dass
\[
l_j(x_i) = \delta_{ij}=\begin{cases}
1&\qquad i=j\\
0&\qquad\text{sonst.}
\end{cases}
\]
\end{aufgabe}

\begin{figure}
\centering
\includegraphics{chapters/30-interpolation/figures/basis.pdf}
\caption{Polynome $l_j(x)$, welche des spezielle Interpolationsproblem
\ref{buch:aufgabe:speziellesinterpolationsproblem}
lösen.
\label{buch:figure:spezielleinterpolation}}
\end{figure}

Jedes der Interpolationspolynome $l_j$ hat Grad $n$, also hat auch eine
beliebige Linearkombination den Grad höchstens $n$.
Die Linearkombination
\[
p(x) = \sum_{j=0}^n f_j l_j(x)
\]
ist das gesuchte Interpolationspolynom, wie Einsetzen von $x_i$ in
\[
p(x_i)
=
\sum_{j=0}^n f_jl_j(x_i)
=
\sum_{j=0}^n f_j\delta_{ij}
=
f_i
\]
bestätigt.

\begin{beispiel}
Ein besonders einfacher Fall ist $n=1$.
Gesucht ist eine lineare Funktion $l(x)=a_1x+a_0$ derart, dass
$l(x_0)=f_0$ und $l(x_1)=f_1$.
Polynome $l_0$ und $l_1$ können leicht angegeben werden:
\[
l_0(x) = \frac{x_1-x}{x_1-x_0}
\qquad\text{und}\qquad
l_1(x) = \frac{x-x_0}{x_1-x_0}
\]
haben die die geforderten Eigenschaften.
Die gesuchte Interplationsfunktion ist daher
\[
p(x)
=
\frac{x_1-x}{x_1-x_0}f_0 + \frac{x-x_0}{x_1-x_0} f_1
=
x \frac{f_1-f_0}{x_1-x_0}   + \frac{x_1f_0-x_0f_1}{x_1-x_0}.
\]
Der Koeffizient von $x$ ist wie erwartet die Steigung der Geraden durch
die Punkte $(x_0,f_0)$ und $(x_1,f_1)$.
\end{beispiel}

Ein Polynom vom Grad $n+1$, welches in {\em allen} Stützstellen verschwindet,
ist leicht zu finden, es ist 
\[
(x-x_0)(x-x_1)(x-x_2)\cdots (x-x_{n-1})(x-x_n).
\]
Ein Polynom, welches nur an der Stützstelle $x_j$ {\em nicht} verschwindet,
ensteht, indem man den Faktor $(x-x_j)$ weglässt, es hat den Grad $n$.
Wir führen dafür die Notation
\[
(x-x_0)(x-x_1)(x-x_2)\cdots \widehat{(x-x_j)}\cdots (x-x_{n-1}(x-x_n),
\]
der Hut bedeutet, dass dieser Faktor weggelassen werden soll.
Allerdings hat dieses Polynom nicht den geforderten Wert $1$, man muss es
also noch mit einer geeigneten Konstante multiplizieren.
Das gesuchte Polynom $l_j(x)$ hat daher die Form
\[
l_j(x)
=
c_j(x-x_0)(x-x_1)(x-x_2)\cdots \widehat{(x-x_j)}\cdots (x-x_{n-1}(x-x_n).
\]
Einesetzen von $x_j$ ergibt
\[
l_j(x_j) = 1 = 
c_j(x_j-x_0)(x_j-x_1)(x_j-x_2)\cdots \widehat{(x_j-x_j)}\cdots(x_j-x_{n-1})(x_j-x_n),
\]
die Konstante $c_j$ ist daher
\[
c_j = \prod_{i=0\atop i\ne j}^n \frac{1}{x_j-x_i}.
\]

\begin{beispiel}
Man finde ein Polynome, welches $l(0)=l(1)=0$ und $l(\frac12)=1$
erfüllt.
Wegen $f_0=f_2=0$ ist nur das Polynome $l_1$ zu ermitteln, es ist
\[
l(x) = l_1(x)
=
\frac{(x-x_0)(x-x_2)}{(x_1-x_0)(x_1-x_2)}
=
\frac{x(x-1)}{\frac12(\frac12-1)}
=
\frac14x(1-x).
\qedhere
\]
\end{beispiel}

%
% Fehler des Interpolationspolynoms
%
\subsection{Fehler von Approximationspolynomen
\label{buch:section:interpolation:fehler}}
Getreu der Maxime, dass wir zu jeder numerischen Lösungsformel auch
Informationen über die zu erwartenden Fehler brauchen, entwickeln
wir in diesem Abschnitt die Theorie des Fehlers der Approximationspolynome.
Wir müssen zu diesem Zweck einen kleinen Ausflug in die Analysis unternehmen
in einen Bereich, der im Unterricht manchmal etwas zu kurz kommt.

Wenn die Ableitung einer Funktion in einem Interval klein ist,
dann werden auch die Funktionswerte im Inneren dieses Intervals
nicht gross von den Werten am Rand abweichen können.
Eine grosse Abweichung würde ja automatisch eine Steigung einer Sekanten
und damit auch eine grosse Steigung einer Tangenten zur Folge haben.
Dies ist die Idee, die den nachfolgend entwickelten Fehlerabschätzungen
zu Grunde liegt.

\subsubsection{Der Zwischenwertsatz}
Der Ausgangspunkt aller nachfolgenden Überlegungen ist die intuitiv
anschauliche Tatsache, dass eine stetige Funktion keine Sprünge macht.

\begin{satz}
Eine auf dem Intervall $[a,b]$ stetige Funktion nimmt jeden Wert im
Interval $[f(a),f(b)]$ an.
Anders ausgedrückt, für jedes $y$ zwischen $f(a)$ und $f(b)$ gibt es ein 
$x$ zwischen $a$ und $b$ derart, dass $y=f(x)$.
\end{satz}

Dieser Satz war natürlich bereits die Grundlage des Verfahrens der
Interval-Halbierung, mit welchem wir in
Abschnitt~\ref{buch:subsection:intervallhalbierung}
Gleichungen gelöst haben.
Wenn die Funktion an den Intervallenden verschiedene Vorzeichen hat,
dann muss es eine Nullstelle im Inneren des Intervalls geben.
Die Intervallhalbierung hat in jedem Schritt ein neues Intervall
konstruiert, das die Nullstelle enthielt.

\subsubsection{Der Satz von Rolle}
\begin{figure}
\centering
\includegraphics{chapters/30-interpolation/figures/rolle.pdf}
\caption{Satz von Rolle: eine nicht konstante differenzierbare Funktion,
die an den Enden eines Intervals den gleichen Funktionswert hat, hat im 
Inneren des Intervals eine Stelle $\xi$ mit Ableitung $0$.
\label{buch:figure:rolle}}
\end{figure}
Der Satz von Rolle erweitert den Zwischenwertsatz auf die Ableitung einer
differenzierbaren Funktion an (Abbildung~\ref{buch:figure:rolle}).

\begin{satz}[Rolle]
\label{buch:satz:rolle}
Sei $f$ eine auf dem Interval $[a,b]$ nicht konstante,
stetig differenzierbare Funktion
mit $f(a)=f(b)$, dann gibt es einen Punkt $\xi\in(a,b)]$ im Inneren
des derart, dass $f'(\xi)=0$.
\end{satz}

Der Satz von Rolle ist eine selbstverständlichkeit, wenn die Ableitung
$f'(x)$ stetig ist, doch dies wird nicht vorausgesetzt, es wird nur
verlangt, dass die Ableitung existiert.
Ausserdem macht der Satz eine Aussage darüber, dass die Zwischenstelle
$\xi$ im Inneren des Intervals sei.

\begin{proof}[Beweis]
Eine stetige Funktion hat auf dem kompakten Interval $[a,b]$ mindestens
ein Maximum und ein Minimum.
Da die Funktion nicht konstant ist, ist das Maximum oder das Minimum
von $f(a)$ verschieden.
Wir nehmen an $\xi\in[a,b]$ sei ein Maximum mit dieser Eigenschaft,
das Argument für das Minimum ist völlig analog.
Wegen $f(\xi)>f(a)$ ist $\xi$ ein Punkt im Inneren des Intervals,
also $\xi\in(a,b)$.

Wegen $f(\xi) \ge f(x)\forall x\in[a,b]$
folgt dann
\begin{align*}
f'(\xi) &= \lim_{h\to 0+}  \frac{f(\xi+h)-f(\xi)}{h} \le 0
\\
f'(\xi) &= \lim_{h\to 0-}  \frac{f(\xi+h)-f(\xi)}{h} \ge 0.
\end{align*}
Da $f$ differenzierbar ist, müssen diese beiden Grenzwerte übereinstimmen,
also ist $f'(\xi)=0$.
\end{proof}


\subsubsection{Nullstellen und der Satz von Rolle}
\begin{figure}
\centering
\includegraphics{chapters/30-interpolation/figures/nullstellen.pdf}
\caption{Schachtelung der Nullstellen von $f(x)$, $f'(x)$ und $f''(x)$.
Der Satz von Rolle~\ref{buch:satz:rolle} impliziert, dass sich zwischen zwei
Nullstellen von $f$ immer eine Nullstelle von $f'$ befindet, und
ebenso zwischen zwei Nullstellen von $f'$ eine von $f''$.
\label{buch:figure:nullstellen}}
\end{figure}

\begin{satz}
\label{buch:satz:nullstellen}
Ist $f$ eine differenzierbare Funktion auf dem Intervall $[a,b]$
mit Nullstellen 
\[
a=x_0 < x_1 < x_2 < \dots < x_{n-1} < x_n = b,
\]
die auf keinem Teilintervall $[x_i,x_{i+1}]$ konstant ist,
dann hat $f'$ im Inneren jedes Teilintervalls $[x_i, x_{i+1}]$
eine Nullstelle.
\end{satz}

Das Polynom 
\[
l(x) = (x-x_0)(x-x_1)\dots (x-x_{n-1})(x-x_n),
\]
welches für die Konstruktion des Interpolationspolynoms verwendet
wurde, hat genau die Nullstellen $x_0,x_1,\dots,x_{n-1},x_n$.
Nach dem Satz~\ref{buch:satz:nullstellen} muss es zwischen je
zwei aufeinanderfolgenden Nullstellen von $l$ eine Nullstelle der
Ableitung geben. 
Diese Situation ist in Abbildung~\ref{buch:figure:nullstellen}
für den Fall $l(x)=(x+2)(x+1)x(x-1)(x-2)$ dargestellt.

Die höheren Ableitungen $f^{(k)}$ haben ihre Nullstellen
natürlich auch wieder zwischen den Nullstellen der Ableitung $f^{(k-1)}$.
Die $n$-te Ableitung ist konstant und hat keine Nullstellen.

\begin{figure}
\centering
\includegraphics{chapters/30-interpolation/figures/norm.pdf}
\caption{Fehler des Lagrange-Interpolationspolynoms für die Funktion
$f(x)=e^{-x^2/2}/\sqrt{2\pi}$.
Der Fehler nimmt mit der Anzahl der Stützstellen bis $n=30$ ab, danach
wird die Berechnung instabil und der Fehler nimmt wieder zu.
\label{buch:figure:lagrangefehler}}
\end{figure}

\subsubsection{Der Mittelwertsatz der Differentialrechnung}

\subsubsection{Taylorreihe mit Restformel}

\subsubsection{Fehler des Lagrange-Interpolationspolynoms}
Der folgende satz gibt vollständige Auskunft über den Fehler des
Interpolationspolynoms.

\begin{satz}
\label{buch:satz:lagrangefehler}
Sei $p$ ein Polynom vom Grad $n$, welches mit der $n+1$-mal differenzierbaren
Funktion $f$ an den $n+1$ Stellen
\[
a = x_0 < x_1 < x_2 < \dots  < x_{n-1} < x_n=b
\]
übereinstimmt.
Dann gibt es für jedes $x\in[a,b]$ ein $\xi_x\in [a,b]$ mit
\begin{equation}
f(x) - p(x) = \frac{f^{(n+1)}(\xi_x)}{(n+1)!} l(x)
\label{buch:equation:polyfehler}
\end{equation}
\end{satz}

\begin{proof}[Beweis]
An den Stützstellen $x_i$ ist $f(x_i)-p(x_i)=0$ und $l(x_i)=0$, die
Gleichung~\eqref{buch:equation:polyfehler} ist also trivialerweise
erfüllt.

Sei jetzt also $x\in[a,b]$ verschieden von allen $x_i$.
Da $l(x)\ne 0$ ist, gibt es eine Zahl $c$ derart, dass
\begin{equation}
f(x)-p(x)=cl(x)
\qquad\Leftrightarrow\qquad
f(x)-p(x)-cl(x)=0.
\label{buch:equation:polyfehler1}
\end{equation}
Die Funktion $g(x)=f(x)-p(x)-cl(x)$ verschwindet in allen Stützstellen $x_i$
und zusätzlich auch noch im Punkt $x$, sie hat also $n+1$ Nullstellen.

Nach dem Nullstellen-Schachtelungssatz~\ref{buch:satz:nullstellen}
hat die $n+1$ Ableitung von $g$ eine Nullstelle im Intervall.
Es gibt also eine Zahl $\xi_x\in[a,b]$ mit $g^{(n+1)}(\xi_x)=0$.

Da $p$ Grad $n$ hat, ist die $n+1$-te Ableitung $0$.
Das Polynom $l(x)$ hat die Form
\[
l(x) = x^{n+1} -(x_0+x_1+\dots+x_{n-1}+x_n)x^{n-1} + \dots + (-1)^{n+1}x_0x_1\dots x_{n-1}x_n,
\]
seine $n+1$-Ableitung ist die Konstanten $(n+1)!$.

Die Folgerung $g^{(n+1)}(\xi_x)=0$ wird damit zu
\[
0 = f^{(n+1)}(\xi_x) -c (n+1)!
\qquad\Rightarrow\quad
c=\frac{f^{(n+1)}(\xi_x)}{(n+1)!}.
\]
Einsetzen in \eqref{buch:equation:polyfehler1} ergibt
\[
f(x)-p(x) = cl(x)=\frac{f^{(n+1)}(\xi_x)}{(n+1)!} l(x),
\]
wie behauptet.
\end{proof}

Dieser Satz erlaubt den Fehler eines Interpolationspolynoms abzuschätzen,
wenn die $n+1$-te Ableitung der Funktion $f$ bekannt ist.
Wir bezeichnen mit
\[
\|g\| = \sup_{a\le x\le b} |g(x)|
\]
die {\em Supremun-Norm} der Funktion $g$ im Intervall $[a,b]$.
\index{Supremum-Norm}

\begin{korollar}
\label{buch:korollar:interpolationsfehler}
Ist $p$ ein Interpolationspolynom vom Grad $n$, welches mit der Funktion
$f$ in den Stellen $a=x_0<x_1<\dots <x_{n+1}<x_n=b$ übereinstimmt, dann
ist 
\[
|f(x)-p(x)| \le \frac{\|f^{(n+1)}\|}{(n+1)!} |l(x)|.
\]
\end{korollar}

\begin{beispiel}
\begin{figure}
\centering
\includegraphics{chapters/30-interpolation/figures/sin.pdf}
\caption{Interpolation der Funktion $f(x)=\sin x$ mit nur drei 
Stützstellen $x_0=0$, $x_1=\frac{\pi}2$ und $x_2=\pi$.
Der Fehler ist deutlich kleiner als die Abschätzung mit
Satz~\ref{buch:satz:lagrangefehler} erwarten lässt.
\label{buch:figure:sin}}
\end{figure}
Die Funktion $f(x)=\sin x$ soll mit den Stützstellen $x_0=0$, $x_1=\frac{\pi}2$
und $x_2=\pi$ interpoliert werden.
Das Interpolationspolynom ist ein quadratisches Polynom mit Nullstellen
$x_0$ und $x_2$, der Funktionswert bei $x_1$ muss $1$ sein.
Man kann sich davon überzeugen, dass das Polynom
\[
p(x) = \frac{4}{\pi^2} x(\pi -x )
\]
diese Eigenschaft hat.
Wie gross ist der Fehler dieses Interpolationspolynoms?

Die dritten Ableitungen der Funktion $f(x)=\sin x$ sind, bekannt, es ist
$f^{(3)}(x)=-\cos x$.
Der Betrag von $f^{(3)}(x)$ wird also nie grösser als $1$.
Es folgt, dass
\[
|f(x)-p(x)| \le \frac{1}{3!} l(x)
=
\frac16 |x(x-{\textstyle\frac{\pi}2})(x-\pi)|
\]
Die Ableitung des Polynoms auf der rechten Seite hat Nullstellen bei
$\frac{\pi}2 \pm \frac{\pi}{2\sqrt{3}}$,
durch Einsetzen erhält man den maximalen Wert
\[
\|f^{(3)}\|
=
\frac{\pi^3}{12\sqrt{3}}\simeq 1.49179.
\]
Wir schliessen, dass das Interpolationspolynom niemals um mehr als $0.24863$
vom Funktionswert abweichen kann.
\end{beispiel}

%
% Tschebyscheff Interpolation
%
\subsection{Wahl der Stützstellen und Tschebyscheff-Interpolationspolynom
\label{buch:section:interpolation:tschebyscheff}}
Das Korllar~\ref{buch:korollar:interpolationsfehler} besagt, dass der
Fehler des Interplationspolynom durch den Betrag von $l(x)$ begrenzt
ist.



\begin{figure}
\centering
\includegraphics{chapters/30-interpolation/figures/tscheb.pdf}
\caption{Fehler des Interpolationspolynomes für die Funktion
$f(x)=e^{-x^2/2}/\sqrt{2\pi}$ mit Stützstellen nach Tschebyscheff.
Der Fehler bleibt über das ganze Intervall gleichmässig.
Für eine grosse Zahl von Stützstellen erreicht die Interpolation die
Maschienengenauigkeit.
\label{buch:figure:tschebyschefffehler}}
\end{figure}

%
% hermite.tex
%
% (c) 2020 Prof Dr Andreas Müller, Hochschule Rapperswil
%
\section{Hermite-Interpolation
\label{buch:section:hermite}}
\rhead{Hermite-Interpolation}
Das Lagrange-Interpolationspolynom nimmt zwar in umittelbarer Nähe der 
Stützstellen zuverlässig Funktionswerte nahe den gegeben Werten an,
doch insbesondere gegen den Rand des Intervalls können die oft beobachteten
Oszillationen eine schlechte Approximation bewirken.
\index{Lagrange-Interpolationspolynom}%
\index{Oszillation}%
\index{Approximation}%
Im Gegensatz zur Taylor-Reihe, deren Ableitung mindestens in der Nähe des
Entwicklungspunktes auch mit der Ableitung der zu approximierenden
Funktion übereinstimmt, gibt es für das Interpolationspolynom keine
solche Garantie.
\index{Taylor-Reihe}%
Beide Schwierigkeiten könnten gemildert werden, indem gefordert wird,
dass das Polynom nicht nur die gleichen Funktionswerte, sondern auch
die gleichen Ableitungen bis zu einer bestimmten Ordnung haben soll.
Dies ist die Idee der {\em Hermite-Interpolation},
\index{Hermite-Interpolation}%
die in diesem Abschnitt vorgestellt werden soll.

%
% Aufgabenstellung
%
\subsection{Aufgabenstellung
\label{b8ch:subsection:hermite:aufgabe}}
Das Hermite-Interpolationspolynom löst die folgende Approximationsaufgabe.

\begin{aufgabe}[Hermite-Interpolationspolynom]
\index{Hermite-Interpolationsproblem}%
\index{Hermite-Interpolationspolynom}%
Gegeben Stützstellen
\[
a=x_0<x_1<x_2<\dots < x_{n-1}<x_n=b
\]
und Funktionswerte $f_i$, $0\le i\le n$, und Werte
$s_i^{(k)}$ der $k$-ten Ableitungen bis zur $m$-ten Ordnung, $1\le k\le m$,
finde ein Polynom $h$, mit
\begin{equation}
h(x_i) = f_i,\quad h^{(k)}(x_i)=s_i^{(k)},\quad 0\le i \le n, 1\le k\le m.
\label{buch:equation:hermitebedingungen}
\end{equation}
\end{aufgabe}
Die Aufgabenstellung formuliert $N=(n+1)(k+1)$ Bedingungen für das Polynom $h$,
es braucht also im Allgemeinen ein Polynom mindestens vom Grade $N=(n+1)(k+1)$,
um alle diese Bedingungen erfüllen zu können.
Ein elementarer Ansatz könnte sein, eine Polynom in der Form
$a_Nx^N+a_{N-1}x^{N-1}+\dots+a_1x+a_0$ anzusetzen, die Bedingungen
\eqref{buch:equation:hermitebedingungen} als lineare Gleichungen für
die Koeffizienten auszuschreiben und das Gleichungssystem zu lössen.
Dieses Vorgehen ist allerdings sehr aufwendig und numerisch nicht besonders
stabil.
Ein Weg analog zur Bestimmung des Lagrange-Interpolationspolynomes in
Abschnitt~\ref{buch:section:interpolation:bestimmung}
ist daher angezeigt.

%
% Bestimmung des Hermite-Interpolationspolynoms
%
\subsection{Bestimmung des Hermite-Interpolationspolynom
\label{buch:subsection:hermite:bestimmung}}
Wir führen die Konstruktion nur für den Fall $m=1$ durch, also für 
Interpolationspolynome, die den Funktionswerten und ersten Ableitungen
übereinstimmen.
Wie in Abschnitt~\ref{buch:section:interpolation:bestimmung} suchen
wir zunächst wieder eine Lösung des folgenden
{\em speziellen Interpolationsproblems}.

\begin{aufgabe}[Spezielles Hermite-Interpolationsproblem]
\index{spezielles Hermite-Interpolationsproblem}%
\index{Hermite-Interpolationsproblem!spezielles}%
Gegeben Stützstellen
\[
a=x_0<x_1<x_2<\dots < x_{n-1}<x_n=b
\]
finde Polynome $h_j$ und $h_j^1$ vom Grad höchstens $2n+1$ derart, dass
\[
\left.
\begin{aligned}
h_j(x_i)&=\delta_{ij}\\
h_j'(x_i)&=0
\end{aligned}\right\}\forall i,j
\qquad\text{und}\qquad
\left.
\begin{aligned}
h_j^1(x_i)&=0\\
h_j^{1\prime}(x_i)&=\delta_{ij}
\end{aligned}\right\}\forall i,j
\]
\end{aufgabe}

\begin{proof}[Lösung]
Ein Polynom vom Grad $2n+2$, welches in allen Stützstellen eine 
doppelte Nullstelle hat, ist das Produkt
\[
(x-x_0)^2 (x-x_1)^2 (x-x_2)^2 \dots (x-x_{n-1})^2 (x-x_n)^2.
\]
Die gesuchten Polynome $h^1_j$ haben in jeder Stützstelle ausser in $x_j$
ein doppelte Nullstelle, die Nullstelle in $x_j$ muss einfach sein.
Ein solches Polynom kann man erhalten, indem man einen der
Faktoren $(x-x_j)$ weglässt, oder zu
\[
p_j(x)
=
(x-x_0)^2 (x-x_1)^2 \cdots \widehat{(x-x_k)^2} \dots (x-x_{n-1})^2(x-x_n)^2
\]
einen solchen Faktor hinzufügt:
\[
h_j^1(x)
=
c_j^2 (x-x_j) p_j(x),
\]
die Konstante $c_j^2$ muss passend gewählt werden, damit die Ableitung
\[
h_j^{1\prime}(x)
=
c_j^2 \underbrace{\frac{d}{dx}(x-x_j)}_{\displaystyle=1}
+
c_j^2
(x-x_j)
\frac{d}{dx} p_j(x)
\]
den richtigen Wert bekommt.
An der Stelle $x=x_j$ fällt der zweite Term weg und es bleibt
\[
h_j^{1\prime}(x_j)
=
c_j^2 p_j(x_j).
\]
und damit ist $c_j^2 = 1/p_j(x_j)$.
Dies ist das Quadrat des entsprechenden Normierungsfaktors, der beim
Lagrange-Interpolationspolynom zur Anwendung kam.

Die Polynome $h_j$ haben in allen Stützstellen ausser $x_j$ eine doppelte
Nullstelle.
Das Produkt $p_j(x)$ teilt diese Eigenschaft.
Da es vom Grad $2n$ ist, haben wir nur die Freiheit, einen Linearfaktor
der Form $(u_j(x-x_j)+v_j)$ hinzuzufügen, um $h_j$ zu erhalten.
Es müssen also $u_j$ und $v_j$ so gewählt werden, dass für
\begin{equation}
h_j(x)=(u_j(x-x_j)+v_j) p_j(x)
\qquad
\text{die Gleichungen}
\qquad
\left\{
\quad
\begin{aligned}
h_j(x_j) &= v_j p_j(x_j) = 1\\
h'_j(x_j) &= u_j p_j(x_j) + v_j p'_j(x_j)=0
\end{aligned}
\right.
\label{buch:equation:hermite:4711}
\end{equation}
gelten.
Aus der ersten Gleichung folgt $ v_j = 1/p_j(x_j) = c_j^2$, aus der zweiten
\[
u_j
=
-\frac{p'_j(x_j)}{p_j(x_j)^2}
=
- c_j^4 p'_j(x_j).
\]
Einsetzen in die \eqref{buch:equation:hermite:4711} ergibt
\[
h_j(x) = \biggl(-
\frac{p'_j(x_j)}{p_j(x_j)^2} (x-x_j) + \frac{1}{p_j(x_j)}\biggr) p_j(x)
=
\frac{p'_j(x_j)(x-x_j) + p_j(x_j)}{p_j(x_j)^2}p_j(x)
\]
Andererseits ist $p_j(x)(x-x_j)=h^1_j(x)/c_j^1$, man kann also auch
\begin{equation}
h_j(x)  
=
\frac{p_j(x)}{p_j(x_j)}
-
\frac{p'_j(x_j)}{c_j^1p_j(x_j)^2} h_j^1(x)
\label{buch:equation:hermite:4712}
\end{equation}
schreiben.
Der erste Term in~\eqref{buch:equation:hermite:4712} ist das Quadrat
des Lagrange-Interpolationspolynoms $l_j(x)$. 
\end{proof}

Mit der Lösung des speziellen Interpolations-Problems findet man jetzt
auch eine Lösung für das allgemeine Problem.
Das gesuchte Interpolationspolynom ist
\[
h(x) 
=
\sum_{j=0}^n f_j h_j(x)
+
\sum_{j=0}^n s_j h^1_j(x).
\]

\subsection{Zwei Stützstellen
\label{buch:subsection:hermite:zweistuetzstellen}}
Der Fall zweier Stützstellen $x_0$ und $x_1$ ist von einiger praktischer
Bedeutung.
Er wird zum Beispiel im Abschnitt~\ref{buch:section:spline}
zum Einsatz kommen.
Die Polynome $h$ und $h^1$ sollen daher für diesen Falll explizit
berechnet werden.

Die Polynome $p_0$ und $p_1$ sind
\[
p_0(x) = (x-x_1)^2
\qquad\text{und}\qquad
p_1(x) = (x-x_0)^2,
\]
woraus $c_0^2 = c_1^2 = (x_1-x_0)^2$ folgt.

Die Polynome $h_0^1$ und $h_1^1$ entstehen durch geeignete Normierung
der Polynome $(x-x_0)p_0(x)$ und $(x-x_1)p_1(x)$, also
\[
h_0^1(x)
=
\frac{(x-x_0)(x-x_1)^2}{(x_0-x_1)^2}
\qquad\text{beziehungsweise}\qquad
h_1^1(x)
=
\frac{(x-x_0)^2(x-x_1)}{(x_0-x_1)^2}.
\]

Für die Polynome $h^1_0$ und $h^1_1$ sind die Konstaten $u_0$ und $u_1$
zu bestimmen.
Die Ableitung der Polynome $p_j$ sind
\[
p_0'(x) = 2(x-x_1)
\qquad\text{und}\qquad
p_1'(x) = 2(x-x_0)
\]
und damit ist
\[
u_0
=
-\frac{p_0'(x_0) }{(x_0-x_1)^4}
=
-2\frac{x_0-x_1}{(x_0-x_1)^4}
=
-\frac{2}{(x_0-x_1)^3}
\qquad\text{und}\qquad
u_1
=
-\frac{p_1'(x_1) }{(x_0-x_1)^4}
=
\frac{2}{(x_0-x_1)^3}.
\]
Aus \eqref{buch:equation:hermite:4711} folgt jetzt
\begin{align*}
h_0(x)
&=
(u_0(x-x_0)+v_0)(x-x_1)^2
=
-\frac{2}{(x_0-x_1)^3}(x-x_0)(x-x_1)^2 + \frac{(x-x_1)^2}{(x_0-x_1)^4}
\\
h_1(x)
&=
(u_1(x-x_1)+v_1)(x-x_0)^2
=
\frac{2}{(x_0-x_1)^3}(x-x_1)(x-x_0)^2 +\frac{(x-x_0)^2}{(x_0-x_1)^2}.
\end{align*}

\subsubsection{Der Spezialfall $x_0=0$}
In diesem Fall schreiben wir $m=x_1$ für die Intervalllänge und erhalten
die Polynome
\begin{equation}
\begin{aligned}
h_0^1(x) &= \frac{x(x-m)^2}{m^2}
&&&
h_1^1(x) &= \frac{x^2(x-m)}{m^2}
\\
h_0(x)   &= \frac{2x(x-m)^2}{m^3} +\frac{(x-m)^2}{m^4}
&&&
h_1(x)   &= -\frac{2x^2(x-m)}{m^3} +\frac{x^2}{m^4}.
\end{aligned}
\end{equation}
\index{h01@$h_0^1(x)$}%
\index{h11@$h_1^1(x)$}%
\index{h0@$h_0(x)$}%
\index{h1@$h_1(x)$}%

\subsubsection{Der Spezialfall $x_0=0$, $x_1=1$}
\begin{figure}
\centering
\includegraphics{chapters/30-interpolation/figures/h.pdf}
\caption{Hermite-Basispolynome für das Intervall $[0,1]$
nach \eqref{buch:equation:hermite:h}
\label{buch:figure:hermite:h}}
\end{figure}
Eine besonders einfache Form nehmen die Polynome $h_j^0$ und $h_j^1$ an,
wenn man sie auf das Intervall $[0,1]$ spezialisiert.
Wir bezeichnen diese Polynome mit grossen Buchstaben, sie sind
\begin{equation}
\begin{aligned}
H_0^1(x) &= x(1-x)^2=x^3-2x^2+x
&&&
H_1^1(x) &= (1-x)x^2=x^3-x^2
\\
H_0(x)   &= (1+2x)(1-x)^2= 2x^3-3x^2+1
&&&
H_1(x)   &= (3-2x)x^2 = -2x^3+3x^2
\end{aligned}
\label{buch:equation:hermite:h}
\end{equation}
\index{H01@$H_0^1(x)$}%
\index{H11@$H_1^1(x)$}%
\index{H0@$H_0(x)$}%
\index{H1@$H_1(x)$}%
Graphen dieser Polynome sind in Abbildung~\ref{buch:figure:hermite:h}
dargestellt.

Diese Polynome können auch verwendet werden, die Polynome für ein
beliebiges Intervall wieder zu gewinnen.
Dazu setzen wir $x=(x-x_0)/m$ in die Polynome ein.
Die Polynome $h_0(x)=H_0((x-x_0)/m)$ und $h_1(x)=H_1((x-x_0)/m)$
haben die Werte
\[
\begin{aligned}
h_0(x)
&=
H_0((x-x_0)/m)\bigg|_{x=x_0}
=
H_0(0)=1
&&\text{und}&
h_0(x)
&=
H_0((x-x_0)/m)\bigg|_{x=x_1}
=
H_0(1)=0
\\
h_1(x)
&=
H_1((x-x_0)/m)\bigg|_{x=x_0}
=
H_1(0)=0
&&\text{und}&
h_1(x)
&=
H_1((x-x_0)/m)\bigg|_{x=x_1}
=
H_1(1)=1
\end{aligned}
\]
und Ableitungen
\begin{align*}
h''_i(x_0)
&=
\frac{d}{dx} H_i((x-x_0)/m)\bigg|_{x=x_0}
=
H'_i((x-x_0)/m) \frac{1}{m}\bigg|_{x=x_0}
=
\frac{H_i'(0)}{m} = 0
\end{align*}
an den Intervallenden.

Tun wir dasselbe für die Polynome $H_0^1$ und $H_1^1$, erhalten wir
\begin{align*}
h_j^{1\prime}(x_i)
&=
\frac{d}{dx} H_j^1((x-x_0)/m) \bigg|_{x=x_i}
=
H_j^{1\prime}((x-x_0)/m)\bigg|_{x=x_i} \frac{1}{m}
=
\frac1m
H_j^{1\prime}(i)
=
\frac1m\delta_{ij},
\end{align*}
dies ist bis auf den Faktor $1/m$ korrekt.
Daraus lesen wir ab, dass wir die Polynome
\[
h_j^1(x)
=
mH_j^1((x-x_0)/m)
\]
für die Ableitungen verwenden müssen.

\subsubsection{Zweite Ableitungen}
\index{zweite Ableitung}%
\index{zweite Ableitung!des Hermite-Interpolationspolynoms}%
\index{Hermite-Interpolationspolynom!zweite Ableitung}%
Für die spätere Anwendung bei der Spline-Interpolation untersuchen
wir auch noch die zweiten Ableitung des Hermite-Interpolationspolynoms
im Fall zweier Stützstellen am Rande des Intervalls.
Wir tun dies für die Polynome~\eqref{buch:equation:hermite:h}
und kümmern uns später darum, was auf anderen Intervallen passiert.
Wir erhalten die Werte
\[
\begin{aligned}
H_0''(0)                &= -6 &&&  H_0''(1)                &=  \phantom{-}6
\\
H_1''(0)                &=\phantom{-}6 &&&  H_1''(1)       &= -6
\\
H_0^{1\prime\prime}(0)  &= -4 &&&  H_0^{1\prime\prime}(1)  &=  \phantom{-}2
\\
H_1^{1\prime\prime}(0)  &= -2 &&&  H_1^{1\prime\prime}(1)  &=  \phantom{-}4.
\end{aligned}
\]
Unter Verwendung der Substition $x\to (x-x_0)/m$ können wir jetzt auch
die Werte für die zweiten Ableitungen an den Intervallenden für $h_j$ und
$h^1_j$ bestimmen.
Dazu berechnen wir erst die zweite Ableitung einer Funktion $f((x-x_0)/m)$:
\[
\frac{d^2}{dx^2} f((x-x_0)/m)
=
\frac{d}{dx} f'((x-x_0)/m) \frac1m
=
f''((x-x_0)/m) \frac1{m^2}.
\]
Angewendet auf die oben gefundenen Polynome bedeutet dies, 
\[
\begin{aligned}
h_0''(x_0) &=          - 6/m^2
&&\text{und}&
h_0''(x_1) &= \phantom{-}6/m^2 \\
h_1''(x_0) &= \phantom{-}6/m^2
&&\text{und}&
h_1''(x_1) &=          - 6/m^2 \\
h_0^{1\prime\prime}(x_0) &=           -4/m
&&\text{und}&
h_0^{1\prime\prime}(x_1) &= \phantom{-}2/m \\
h_1^{1\prime\prime}(x_0) &=          - 2/m
&&\text{und}&
h_1^{1\prime\prime}(x_1) &= \phantom{-}4/m.
\end{aligned}
\]







%
% bary.tex
%
% (c) 2020 Prof Dr Andreas Müller, Hochschule Rapperswil
%
\section{Baryzentrische Formeln für Interpolationspolynome
\label{buch:section:baryzentrisch}}
\index{baryzentrische Formel}%
Die Interpolationspolynome von Lagrange und Hermite haben 
in der bis jetzt gezeigten Form das folgende grundlegende Problem.
Sie sind definiert über das Produkt
\[
l(x)
=
(x-x_0)(x-x_1)\dots (x-x_{n-1})(x-x_n).
\]
Ist die Zahl der Stützstellen gross und erstrecken sich die 
Stützstellen über einen grossen Bereich, dann sind einzelne Faktoren
$(x-x_i)$ immer gross, wie wir bei der Diskussion von Runges Phänomen
bereits diskutiert haben.
\index{Runges Phänomen}%
Zudem tritt bei der Berechnung eines Wertes in inmittelbarer 
Nähe der Stützstelle $x_j$ in dem Faktor $(x-x_j)$ Auslöschung auf.
Der grosse relative Fehler dieses Faktors wird durch die anderen Faktoren
zu einem grossen absoluten Fehler aufgeblasen.

Andererseits ist klar, dass sich das Interpolationspolynom vor allem
in der Nähe einer Stützstelle ändern sollte, wenn man den Wert an der
Stützstelle ändert.
Die anderen Stützstellen sollten also nur einen geringen Einfluss auf
den Wert des Interpolationspolynoms haben.
Dies geht aus der bisherigen Form des Interpolationspolynoms ebenfalls
nicht hervor.

Gesucht ist also eine Form des Interpolationspolynoms, welche einsichtig
macht, dass Änderungen von Stützwerten sich vor allem in der nähe der
betroffenen Stützstelle auswirken und die auch bei einer grossen Zahl
von Stützstellen stabil sind.

Früher wurde gezeigt, dass das Interpolationspolynom für Funktionswerte
$f_j$ an den Stützstellen $x_j$ durch die Linearkombination
\[
p(x) = \sum_{j=0}^n f_j l_j(x)
\]
gegeben ist.
Für die Polynome $l_j(x)$ wurde
\[
l_j(x)
=
\frac{
(x-x_0)(x-x_1)\cdots(\widehat{x-x_j}) \cdots (x-x_n)
}{
(x_j-x_0)(x_j-x_1)\cdots (\widehat{x_j-x_j})\cdots (x_j-x_n)
}
\]
gefunden.
Schreibt man
\[
w_j
=
\frac{1}{
\displaystyle\prod_{\scriptstyle k=1\atop \scriptstyle k\ne j}^n (x_j-x_k)
},
\]
dann kann man die Faktoren $l_j(x)$ auch als
\[
l_j(x)
=
\frac{l(x)}{(x-x_j)}\cdot w_j
\]
ausdrücken.
Damit wird das Interpolationspolynom jetzt
\begin{equation}
p(x)
=
l(x) \sum_{j=0}^n \frac{w_jf_j}{x-x_j}.
\label{buch:bary:px}
\end{equation}
Die Zahlen $w_j$ hängen nur von den Stützstellen ab, nicht von den
Funktionswerten $f_j$. 
Sie können also nach Festlegung der Stützstellen einmalig berechnet
werden und verursachen danach keinen weiteren Berechnungsaufwand.

Das Interpolationspolynom wird besonders einfach, wenn alle Funktionswerte
$f_j=1$ sind.
Da das konstante Polynom $p(x)=1$ genau diese Werte annimmt, muss
\[
1 = l(x) \sum_{j=0}^n \frac{w_j}{x-x_j}
\]
gelten.
Damit erhalten wir eine neue Darstellung für 
\begin{equation}
l(x)
=
\frac{1}{\displaystyle\sum_{j=0}^n \frac{w_j}{x-x_j}}.
\label{buch:bary:lx}
\end{equation}
In dieser Form wird vermieden, dass zur Berechnung von $l(x)$ eine
grosse Anzahl Produkte mit potentiell grossen Faktoren gebildet werden
muss.
Sorgen bereiten in der Produktdarstellung vor allem die Faktoren
$x-x_j$ für $x$ weit entfernt von $x_j$.
Stattdessen wird in \eqref{buch:bary:lx} eine Summe von Summanden gebildet,
die klein sind,
wenn $x$ weit von $x_j$ entfernt ist.

Die vorteilhafte Formulierung~\eqref{buch:bary:lx} kann nun dazu
verwendet werden, auch eine verbesserte Formulierung für das
Interpolationspolynom aufzustellen.
Dazu ersetzen wir den Faktor $l(x)$ in \eqref{buch:bary:px}
durch \eqref{buch:bary:lx} und erhalten
\begin{equation}
p(x)
=
\frac{\displaystyle \sum_{j=0}^n \frac{w_jf_j}{x-x_j}
}{
\displaystyle\sum_{j=0}^n \frac{w_j}{x-x_j}}.
\label{buch:bary:pfinal}
\end{equation}
Diese Form des Interpolationspolynoms ist ein gewichtetes Mittel 
der Werte $f_j$, gewichtet mit den Gewichten $w_j/(x-x_j)$.
\index{Mittel!gewichtet}%
Diese Gewichte sind klein für $x$ weit weg von $x_j$, die grössten
Gewichte haben die Funktionswerte $f_j$ nahe bei $x$.
\index{Gewicht}%
\index{wj@$w_j$}%
Die Formel~\eqref{buch:bary:pfinal} ist daher eine numerisch besonders
vorteilhafte Form der Auswertung eines Interpolationspolynoms.







%
% spline.tex
%
% (c) 2020 Prof Dr Andreas Müller, Hochschule Rapperswil
%

\section{Spline-Interpolation
\label{buch:section:spline}}
Die Hermite-Interpolation ermöglicht Aporoximationspolynome zu finden,
die sowohl Funktionswerte als auch Ableitungen an den Stützstellen mit
der zu approximierenden Funktion gemeinsam haben.
Dadurch wird der Fehler der Approximationspolynome zwar kleiner, aber
es entsteht das zusätzliche Problem, dass die Ableitungen der
Funktion bestimmt werden müssen.

Die Spline-Interpolation umgeht dieses Problem, indem sie an den
Stützstellen nicht die gleichen Steigungen verlangt, sondern Steigungen,
die zu einem möglichst ``wenig gekrümmten'' Graphen des Approximationspolynoms,
welches natürlich immer noch in den Stützstellen die vorgegebenen Werte
annehmen soll.
Die Steigungen in den Stützstellen sind also Lösungen eines
Optimierungsproblems, welches nicht die am besten passende, sonder
die ``schönste'' Kurve durch die Stützstellen sucht.

\subsection{Anforderungen and die interpolierende Funktion
\label{buch:subsection:anforderungen}}
Gegeben seien wie früher Punkte
\[
a=x_0< x_1 < x_2< \dots < x_{n-1} < x_n = b
\]
auf und Funktionswerte $f_i$ einer im übrigen unbekannten, aber 
ausreichend glatten Funktion $f\colon [a,b]\to\mathbb R:x\mapsto f(x)$,
es ist also $f(x_i)=f_i$.

Gesucht ist eine stetige Funktion $g\colon[a,b]\to\mathbb R:x\mapsto g(x)$,
die die folgenden natürliche Eigenschaften haben soll:
\begin{enumerate}
\item
Die Funktion $g$ nimmt in allen Stützstellen die Werte der Funktion
$f$ an, es ist also $g(x_i)=f_i\;\forall 0\le i\le n$.
\item
Die Funktion $g$ ist stetig differenzierbar im ganze Interval.
Insbesondere existiert die Ableitung $g'(x)$ in jedem Punkt $x$ des
Intervals $[a,b]$, der Graph von $g$ kann also keine ``Knicke'' haben.
\item
Im inneren jedes Teilintervalles $[x_i,x_{i+1}]$ ist die Funktion $g$
beliebig oft stetig differenzierbar und die einseitigen Grenzwerte 
an den Enden der Teilintervalle existieren:
\[
\exists\; \lim_{x\to x_i+} g^{(k)}(x) \quad\forall 0\le i < n
\qquad\text{und}\qquad
\exists\; \lim_{x\to x_i-} g^{(k)}(x) \quad\forall 0< i \le n.
\]
Es wird nicht verlangt, dass die rechts- und linksseitigen Grenzwerte
an den inneren Stützstellen $x1,\dots,x_{n-1}$ übereinstimmen müssen.
\item
Der Graph von $g$ soll möglichst wenige gekrümmt sein.
Da die zweite Ableitung einer Funktion ein Mass für die Krümmung des 
Graphen ist, kann dieses Kriterium dadurch realisiert werdenn, dass
die Funktion $g$ unter allen Funktionen, die die Bedingungen 1--3 erfüllen,
das Integral
\[
J(g)
=
\int_a^b (g''(x) )^2\,dx
\]
minimiert.
\end{enumerate}

Man beachte, dass nirgends verlangt wird, dass die Ableitungen von $g$
an den Stützstellen irgendwie mit der Funktion $f$ in Verbingung steht.

\subsection{Das Optimierungsproblem
\label{buch:subsection:variation}}
Zunächst ist nicht klar, ob das eben gestellt Optimierungsproblem überhaupt
eine Lösung hat. 
In jedem Teilinterval $[x_i,x_{i+1}]$ geht es um ein Problem der
folgenden Art.
Gesucht ist eine Funktion, die an den Intervallenden die vorgegebene
Werte $g(x_i)=f_i$ und $g(x_{i+1})=f_{i+1}$ annimmt, im Inneren des
Intervals beliebig oft stetig differenzierbar ist und zudem einen
Integralausdruck
\[
\int_{x_i}^{x_{i+1}} (g''(x))^2\,dx
\]
minimiert.

Diese Art von Problemen hat bereits Leonhard Euler in recht allgemeiner
Form untersucht und zu diesem Zweck das Gebiet der Variationsrechnung
geschaffen.
Sie tauchen in der Physik zum Beispiel in der folgenden Form auf.

\begin{beispiel}
Ein Teilchen der Masse $m$ bewegt sich entlang der $y$-Achse.
Zur Zeit $a$ befindet es sich bei $f_0$, zur Zeit $b$ bei $f_n$.
Auf das Teilchen wirkt ausserdem eine Kraft, die durch ihr Potential $V(y)$
beschrieben werden kann.
Die Geschwindigkeit zur Zeit $t$ ist $\dot y(t)$.
Die Differenz von kinetischer und potentieller Energie ist
die sogenannte Lagrange-Funktion
\begin{equation}
L(t,y,\dot{y})
=
\frac12m\dot{y}(t)^2
-
V(y(t)).
\label{buch:equation:mechlagrange}
\end{equation}
In der Physik wird gezeigt, dass die Bewegung des Teilchens durch diejenige
Funktion $y(t)$ beschrieben wird, welche das Integral
\[
\int_a^b L(t,y(t),\dot{y}(t))\,dt
\]
minimiert.
\end{beispiel}

Um zu zeigen, dass die Interpolationsfunktion $g$ existiert, lösen
wir daher das folgende, wesentlich allgemeinere Problem.

\begin{satz}
\label{buch:satz:eulerlagrange}
Sei $L(x,y,y_1)$ eine in allen Argumenten beliebig oft stetig differenzierbare
Funktion auf $[a,b]\times \mathbb R \times \mathbb R$.
Es gibt eine glatte Funktion $y(x)$, die in den Intervalenden vorgegebene
Werte $y(a)=y_a$ und $y(b)=y_b$ annimmt und ausserdem das Integral
\[
J(y)
=
\int_a^b L(x, y(x), y'(x) ) \,dx
\]
minimiert,
sie ist Lösung der {\em Euler-Lagrange-Differentialgleichung}
\index{Euler-Lagrange-Differentialgleichung}
\begin{equation}
\frac{d}{dx} \frac{\partial L}{\partial y_1} (x,y(x),y'(x))
-
\frac{\partial L}{\partial y} (x,y(x),y'(x))
=
0.
\label{buch:variation:eulerlagrange}
\end{equation}
\end{satz}

\begin{proof}[Beweis]
Wir gehen wie folgt vor: wir zeigen zunächst, dass eine solche Funktion
eine Differentialgleichung erfüllen muss.
Dann beziehen wir uns auf bekannte Sätze der Theorie der gewöhnlichen
Differentialgleichungen, die besagen, dass die Gleichung eine glatte 
Lösung hat.

Sei jetzt also $y(x)$ eine Funktion mit $y(a)=y_a$ und $y(b)=y_b$, die
das Integral $J(y)$ minimiert.
Ändern wir die Funktion ein klein wenig, dann muss der Wert von $J$ zunehmen.
Wir vollziehen die Änderung, indem wir eine Funktion $h(x)$
wählen mit $h(a)=0$ und $h(b)=0$.
Die Funktionen $y_\varepsilon= y+\varepsilon h$ erfüllen dann alle die
Bedingung $y_\varepsilon(a)=y_a$ und $y_\varepsilon(b)=y_b$, insbesondere
müssen sie alle einen Wert $J(g+\varepsilon h)$ ergeben, der grösser ist
als $J(g)$.
Insbesondere muss die Ableitung von $J(y+\varepsilon h)$ nach $\varepsilon$
an der Stelle $\varepsilon=0$ verschwinden.

Wir berechnen die Ableitung von $J(y+\varepsilon h)$ nach $\varepsilon$:
\begin{align}
0
=
\frac{d}{d\varepsilon} J(y+\varepsilon h)\bigg|_{\varepsilon=0}
&=
\frac{d}{d\varepsilon}
\int_a^b L(x, y(x) + \varepsilon h(x), y'(x)+\varepsilon h'(x))\,dx
\bigg|_{\varepsilon=0}
\notag
\\
&=
\int_a^b
\frac{\partial L}{\partial y}(x, y(x), y'(x)) \, h(x)
+
\frac{\partial L}{\partial y_1} L(x, y(x), y'(x) \, h'(x)
\,dx
\notag
\\
&=
\int_a^b
\frac{\partial L}{\partial y}(x, y(x), y'(x)) \, h(x)
\,dx
+
\int_a^b
\frac{\partial L}{\partial y_1} L(x, y(x), y'(x) \, h'(x)
\,dx
\label{buch:variation:zweiintegrale}
\end{align}
Das zweite Integral enthält die Ableitung $h'(x)$, über die wir nicht viel
wissen.
Wir können diese aber durch partielle Integration los werden:
\begin{align*}
\int_a^b
\frac{\partial L}{\partial y_1} L(x, y(x), y'(x) \, h'(x)
\,dx
&=
\biggl[\frac{\partial L}{\partial y_1}
L(x,y(x),y'(x))\,h(x)
\biggr]_a^b
-
\int_a^b \frac{d}{dx} \frac{\partial L}{\partial y_1}
L(x,y(x),y'(x))\,h(x) \,dx
\intertext{$h$ war so gewählt, dass die Werte, an den Intervalenden
verschwinden, also $h(a)=h(b)=0$.
Der erste Terme verschwindet daher und es bleibt}
&=
-
\int_a^b \frac{d}{dx} \frac{\partial L}{\partial y_1}
L(x,y(x),y'(x))\,h(x) \,dx.
\end{align*}
Einsetzen in \eqref{buch:variation:zweiintegrale} ergibt die Gleichung
\begin{equation}
0=
-
\int_a^b 
\biggl(
\frac{d}{dx}\frac{\partial L}{\partial y_1} (x,y(x), y'(x))
-
\frac{\partial L}{\partial y} (x,y(x),y'(x))
\biggr)
h(x)
\,dx.
\label{buch:variation:eulerintegralform}
\end{equation}

Gleichung \eqref{buch:variation:eulerintegralform}
muss für jede beliebige Funktion $h(x)$ gelten.
Wir möchten zeigen, dass das nur möglich ist, wenn die grosse Klammer
im Integral verschwindet.

Nehmen wir an, die grosse Klammer sei an einer Stelle im Intervall 
von $0$ verschieden.
Dann wird sie wegen der Stetigkeit auch in einer kleinen Umgebung dieser
Stelle immer noch das gleiche Vorzeichen haben.
Wir wählen eine Funktion $h$, die in der gleichen kleinen Umgebung
positiv ist und sonst überall verschwindet.
Das Integral muss dann nur noch über diese kleine Umgebung erstreckt
werden und die Funktion, die integriert wird, hat in der ganzen Umgebung
das gleiche Vorzeichen.
Insbesondere kann das Integral nicht verschwinden.
Somit ist gezeigt, dass die grosse Klammer verschwinden muss, oder dass
die Gleichung
\begin{equation}
\frac{d}{dx} \frac{\partial L}{\partial y_1} (x,y(x),y'(x)) 
-
\frac{\partial L}{\partial y}L(x,y(x),y'(x)).
\end{equation}
gelten muss.
\end{proof}

\begin{beispiel}
Wir wenden die Euler-Lagrange-Gleichung auf die Lagrange-Funktion
\eqref{buch:equation:mechlagrange} an, dabei erhalten wir
\[
\left.
\begin{aligned}
\frac{\partial L}{\partial y}
&=
-V'(y)
\\
\frac{\partial L}{\partial\dot{y}}
&=
m\dot{y}
\end{aligned}
\qquad\right\}
\quad\Rightarrow\quad
\frac{d}{dt} \frac{\partial L}{\partial \dot{y}} - \frac{\partial L}{\partial y}
=
\frac{d}{dt} 
m\dot{y} +V'(y)=0
\quad\Rightarrow\quad
m\ddot{y} = -V'(y).
\]
Dies ist das 2.~Newtonsche Gesetz.
\end{beispiel}

\subsection{Lösung des Optimierungsproblems
\label{buch:subsection:splineinterpolant}}
Leider lässt sich der Satz~\ref{buch:variation:eulerlagrange}
nicht direkt auf das Interpolationsproblem anwenden, weil im
Ausdruck $J(g)$ die zweite Ableitung von $g$ vorkommt.
Wir führen daher die Rechnung, die auf die Euler-Lagrange-Differentialgleichung
geführt hat, nochmals in diesem Spezialfall durch.
Wieder sei $h$ eine Funktion, die in jeder Stützstelle verschwindet.
Die Minimalitätsbedingung ist dann
\begin{align}
0
&=
\frac{d}{d\varepsilon}
\int_{x_i}^{x_{i+1}} (g''(x) + \varepsilon h''(x))^2 \,dx\bigg|_{\varepsilon=0}
\notag
\\
&=\int_{x_i}^{x_{i+1}} 2g''(x)h''(x) + 2\varepsilon h''(x)^2\,dx\bigg|_{\varepsilon=0}
\\
\notag
&=
\int_{x_i}^{x_{i+1}} 2g''(x) h''(x)\,dx.
\intertext{Wie bei der Euler-Lagrange-Gleichung können wir durch partielles
Integrieren die zweite Ableitung der Funktion $h$ los werden:}
0
&=
\biggl[ g''(x) h'(x) \biggr]_{x_i}^{x_{i+1}}
-
\int_{x_i}^{x_{i+1}} g'''(x) h'(x) \,dx
\notag
\\
&=
\biggl[ g''(x) h'(x) \biggr]_{x_i}^{x_{i+1}}
-
\biggl[ g'''(x) h(x) \biggr]_{x_i}^{x_{i+1}}
+
\int_{x_i}^{x_{i+1}} g^{(4)}(x) h(x)\,dx.
\label{buch:equation:splines:integiert}
\end{align}
Auf Grund der Definition von $h$ verschwindet der mittlere Term.

\subsubsection{Bedingungen im Inneren der Teilintervalle}
Jetzt nutzen wir wieder die freie Wahlmöglichkeit der Funktion $h$
aus.
Wir können die Funktion so wählen, dass
$h(x_i)=h(x_{i+1})=h'(x_i)=h'(x_{i+1})=0$ ist, dann 
verschwinden die ersten beiden Terme.
Das Integral verschwindet nur dann immer, wenn der Integrand verschwindet,
wenn also $g^{(4)}(x)=0$ im Inneren jedes Teilintervals $[x_i,x_{i+1}]$.
Es folgt, dass in jedem Teilinterval die Funktion $g$ ein kubisches Polynom 
sein muss.

\subsubsection{Bedingungen an den Stützstellen}
Aus dem verbleibenden ersten Term von
Gleichung~\eqref{buch:equation:splines:integiert}
lässt sich noch mehr über die zweiten Ableitungen der Funktion $g$
schliessen.
Die Summe dieser Terme muss ja ebenfalls $0$ ergeben, also
\begin{align*}
0
&=
\sum_{i=0}^{n-1} 
\biggl[ g''(x) h'(x) \biggr]_{x_i}^{x_{i+1}}
=
\sum_{i=0}^{n-1}
\bigl( g''(x_{i+1}-) h'(x_{i+1}) - g''(x_i+) h'(x_i) \bigr)
\\
&=
-g''(x_0+)h'(x_0)
+
\sum_{i=1}^{n-1} h'(x_i) \bigl(g''(x_i-) - g''(x_i+)\bigr)
+
g''(x_n-)h'(x_n).
\end{align*}
Indem man für $h$ eine Funktion wählt, die an allen Stützstellen verschwindet
und in genau einer Stützstelle Ableitung $1$ hat, was mit einem
Hermite-Interpolationspolynom sicher möglich ist, schliesst man
\begin{equation}
g''(x_i-)=g''(x_i+)\quad\forall 1\le i< n.
\label{buch:equation:splineinner}
\end{equation}
Die Funktion ist also zweimal stetig differenzierbar.
Schliesslich müssen auch die Terme an den Enden der Summe verschwinden.
Eine Funktion $h$, die in allen Stützstellen zusammen mit der ersten
Ableitung in den inneren Stützstellen verschwindet und deren
erste Ableitung in genau einem der Endpunkte $1$ ist zeigt,
dass ausserdem
\begin{equation}
g''(x_0+) = g''(x_n-) = 0
\label{buch:equation:splinerand}
\end{equation}
sein muss.

\subsubsection{Ein Gleichungssystem für die Steigungen}
Zur Lösung des eingangs gestellten Interpolationsproblems ist jetzt
also für jedes Teilinterval $[x_i,x_{i+1}]$ ein kubisches Polynom $g_i(x)$
zu finden, mit folgenden Eigenschaften:
\begin{align*}
g_i(x_i)     &=f_i       &g_i(x_{i+1})  &=f_{i+1}   &0&\le i\le n &&\text{$2n+2$ Bedingungen}
\\
g_{i-1}'(x_i)&=g_i'(x_i) &g_{i-1}''(x_i)&=g_i''(x_i)&1&\le i< n &&\text{$2n$ Bedingungen}
\\
g_0''(x_0)   &=0         &g_n''(x_n)    &= 0        & &         &&\text{$2$ Bedingungen}
\end{align*}
Dies sind $4n+4$ lineare Bedingungen für $n+1$ Polynome, die je $4$
Koeffizienten haben.
Es sollte sich also ein lineares Gleichungssystem finden lassen, welches
diese Koeffizienten findet.

Aus Abschnitt~\ref{buch:section:hermite} ist bekannt, dass die kubischen
Polynome $g_i(x)$ durch die bereits bekannten Funktionswerte $f_i$
und die noch zu findenen Steigungen in den Stützstellen bestimmt sind.
Wir schreiben daher $s_i = g_i'(x_i)$ für die Steigungen und machen es
uns zum Ziel ein Gleichungssystem für die $s_i$ zu finden.

In Abschnitt~\ref{buch:subsection:hermite:zweistuetzstellen}
haben wir Hermite-Interopationspolynome für zwei Stützstellen
zusammengestellt.
Wir haben dort die Polynome $H_i$ und $H_i^1$ konstruiert, aus
denen sich mit der Substition $x\to (x-x_0)/m$ die
Hermite-Interpolationspolynome für das Interval $[x_0,x_0+m]$ 
bilden liess.
Wir bezeichnen die Länge des Intervalls $[x_i,x_{i+}]$
mit $m_i=x_{i+}-x_i$.

Die gesuchte Funktion im Interval ist daher
\begin{equation}
g_i(x) = f_i H_0((x-x_i)/m_i) + f_{i+1} H_1((x-x_i)/m_i)
+
s_i m_i H_0^1((x-x_i=)/m_i) + s_{i+1} m_i H_1^1((x-x_i)/m_i).
\label{buch:equation:spline:loesung}
\end{equation}
Diese Funktion hat die richtigen Funktionswerte und Ableitungen
an den Intervallenden.

Die Steigungen $s_i$ in \eqref{buch:equation:spline:loesung}
ist noch nicht bekannt, aber die Bedingung an die zweiten Ableitungen
wurde noch nicht ausgenutzt.
Die zweiten Ableitungen
\begin{align*}
i&=0
&
0
&=
g_0''(x_0)
=
-\frac{6f_0}{m_0^2} + \frac{6f_1}{m_0^2} -\frac{4s_0}{m_0} + \frac{2s_1}{m_0}
\\
i&=1
&
&\phantom{\mathstrut=\mathstrut}
g_0''(x_1)
=
\phantom{-}
\frac{6f_0}{m_0^2} -\frac{6f_1}{m_0^2} +\frac{2s_0}{m_0} -\frac{4s_1}{m_0}
\\
&
&
&=
g_1''(x_1)
=
-\frac{6f_1}{m_1^2}+\frac{6f_2}{m_1^2} - \frac{4s_1}{m_1}+\frac{2s_2}{m_1}
\\
i&=2
&
&\phantom{\mathstrut=\mathstrut}
g_1''(x_2)
=
\phantom{-}
\frac{6f_1}{m_1^2} -\frac{6f_2}{m_1^2} +\frac{2s_1}{m_1} -\frac{4s_2}{m_1}
\\
&
&
&=
g_2''(x_2)
=
-\frac{6f_2}{m_2^2}+\frac{6f_3}{m_2^2} - \frac{4s_2}{m_2}+\frac{2s_3}{m_2}
\\
&\qquad\vdots
&&
\\
i&=n
&
0&=
g_n''(x_n)
=
\phantom{-}
\frac{6f_{n-1}}{m_n^2}-\frac{6f_n}{m_n^2} +\frac{2s_{n-1}}{m_n}-\frac{4s_n}{m_n}
\end{align*}
In allen Gleichungen kommt der Faktor $2$ vor, den wir herausdividieren 
können.
Schaffen wir die Terme in $f_i$ auf die rechte Seite und sammeln die
Terme mit $s_i$ auf der linken Seite, erhalten wir das Gleichungssystem
\begin{equation}
\begin{linsys}{6}
\displaystyle\frac{2}{m_0} s_0
	&+&\displaystyle \frac{1}{m_0}s_1
		& &
			& &
				& &
				& &
					&=&\displaystyle3\frac{f_1-f_0}{m_0^2}
\\
\displaystyle \frac{1}{m_0} s_0
	&+&\displaystyle \biggl(\frac{2}{m_0}+\frac{2}{m_1}\biggr)s_1
		&+& \displaystyle \frac{1}{m_1}s_2
			& &
				& &
				& &
					&=&\displaystyle3\frac{f_2-f_1}{m_1^2}
\\
	& &\displaystyle\frac{1}{m_1} s_1
		&+&\displaystyle\biggl(\frac{2}{m_1}+\frac{2}{m_2}\biggr) s_2
			&+& \displaystyle\frac{1}{m_2} s_3
				& &
				& &
					&=&\displaystyle3\frac{f_3-f_2}{m_2^2}
\\
	& &
		& &
			& &
				&\ddots&
				&\ddots&
					& &\vdots\hspace*{10pt}
\\
	& &
		& &
			& &
			& &\displaystyle \frac{1}{m_{n-2}}s_{n-1}
				&+&\displaystyle \frac{2}{m_{n-1}}s_n
					&=&\displaystyle3\frac{f_n-f_{n-1}}{m_{n-1}^2}
\end{linsys}
\end{equation}
Die Koeffizientenmatrix und die rechte Seite dieses Gleichungsssytems sind
\[
A
=
\begin{pmatrix}
\displaystyle\frac{2}{m_0}
	&\displaystyle\frac{1}{m_0}
		&
			&
				&
					&
\\[8pt]
\displaystyle\frac{1}{m_0}
	&\displaystyle\frac{2}{m_0}+\frac{2}{m_1}
		&\displaystyle\frac{2}{m_1}
			&
				&
					&
\\[8pt]
	&\displaystyle\frac{1}{m_1}
		&\displaystyle\frac{2}{m_1}+\frac{2}{m_2}
			&\displaystyle\frac{1}{m_2}
				&
					&
\\[8pt]
	&
		&\displaystyle\frac{1}{m_2}
			&\ddots
				&\ddots
					&
\\[8pt]
	&
		&
			&\ddots
				&\ddots
					&\displaystyle\frac{1}{m_{n-2}}
\\[8pt]
	&
		&
			&
				&\displaystyle\frac{1}{m_{n-2}}
					&\displaystyle\frac{2}{m_{n-1}}
\end{pmatrix}
\qquad\text{und}\qquad
b
=
\begin{pmatrix}
\displaystyle3\frac{f_1-f_0}{m_0^2} \\[8pt]
\displaystyle3\frac{f_2-f_1}{m_1^2} \\[8pt]
\displaystyle3\frac{f_3-f_2}{m_2^2} \\[8pt]
\vdots \\[8pt]
\displaystyle3\frac{f_{n-1}-f_{n-2}}{m_{n-2}^2} \\[8pt]
\displaystyle3\frac{f_n-f_{n-1}}{m_{n-1}^2} 
\end{pmatrix}.
\]
Die Gleichungen werden besonders einfach, wenn alle Abstände gleich sind,
zum Beispiel $m=m_0=\dots m_{n-1}$.
Dann kann man die Gleichungen mit $m$ multiplizieren und bekommt für die
Koeffizientenmatrix und die rechte Seite
\[
A
=
\begin{pmatrix}
2&1& &      &      & \\
1&2&1&      &      & \\
 &1&2&1     &      & \\
 & &1&\ddots&\ddots& \\
 & & &\ddots&\ddots&1\\
 & & &      &     1&2
\end{pmatrix}
\qquad\text{und}\qquad
b
=
\frac{3}{m}
\begin{pmatrix}
f_1-f_0\\
f_2-f_1\\
f_3-f_2\\
\vdots\\
f_{n-1}-f_{n-2}\\
f_n-f_{n-1}

\end{pmatrix}
\]

\subsection{Bézier-Kurven und Splines in der Ebene
\label{buch:subsection:bezier}}



\section*{Übungsaufgaben}
\aufgabetoplevel{chapters/30-interpolation/uebungsaufgaben}
\begin{uebungsaufgaben}
\uebungsaufgabe{3001}
\uebungsaufgabe{3002}
\uebungsaufgabe{3003}
\end{uebungsaufgaben}
