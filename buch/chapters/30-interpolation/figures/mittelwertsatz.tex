%
% mittelwertsatz.tex -- template for standalon tikz images
%
% (c) 2020 Prof Dr Andreas Müller, Hochschule Rapperswil
%
\documentclass[tikz]{standalone}
\usepackage{amsmath}
\usepackage{times}
\usepackage{txfonts}
\usepackage{pgfplots}
\usepackage{csvsimple}
\usetikzlibrary{arrows,intersections,math}
\begin{document}
\def\skala{1}
\begin{tikzpicture}[>=latex,thick,scale=\skala]

\xdef\xlinks{1}

\xdef\x{\xlinks}
\pgfmathparse{0.5*\x*(8-\x)-3}
\xdef\ylinks{\pgfmathresult}

\xdef\xrechts{5.5}
\xdef\x{\xrechts}
\pgfmathparse{0.5*\x*(8-\x)-3}
\xdef\yrechts{\pgfmathresult}

\pgfmathparse{(\yrechts-\ylinks)/(\xrechts-\xlinks)}
\xdef\m{\pgfmathresult}

\pgfmathparse{atan(\m)}
\xdef\a{\pgfmathresult}

\pgfmathparse{4-\m}
\xdef\xzero{\pgfmathresult}
\xdef\x{\xzero}
\pgfmathparse{0.5*\x*(8-\x)-3}
\xdef\yzero{\pgfmathresult}

\begin{scope}[xshift=-3.5cm]
	\draw[line width=0.2pt] (\xlinks,0) -- (\xlinks,\ylinks);
	\draw[line width=0.2pt] (\xrechts,0) -- (\xrechts,\yrechts);
	\draw[line width=0.2pt] (\xzero,0) -- (\xzero,\yzero);
	\draw[color=red,line width=1.4pt]
		plot[domain=1:5.5,samples=100]
			({\x},{0.5*\x*(8-\x)-3});

	\draw[color=blue] ({\xlinks-0.5},{\ylinks-\m*0.5})
		--
		({\xrechts+0.5},{\yrechts+\m*0.5});
	\begin{scope}
		\clip (0,0) rectangle (6.5,6.5);
		\draw[color=blue] (0.5,{\yzero-\m*(\xzero-0.5)})
			--
			(6,{\yzero+\m*(6-\xzero)});
	\end{scope}
	\fill[color=white] (\xzero,\yzero) circle[radius=0.08];
	\draw[color=blue] (\xzero,\yzero) circle[radius=0.08];

	\fill[color=white] (\xlinks,\ylinks) circle[radius=0.08];
	\draw[color=red] (\xlinks,\ylinks) circle[radius=0.08];
	\fill[color=white] (\xrechts,\yrechts) circle[radius=0.08];
	\draw[color=red] (\xrechts,\yrechts) circle[radius=0.08];

	\draw (\xlinks,-0.1) -- (\xlinks,0.1);
	\node at (\xlinks,-0.1) [below] {$a$};
	\draw (\xrechts,-0.1) -- (\xrechts,0.1);
	\node at (\xrechts,-0.1) [below] {$b$};
	\draw (\xzero,-0.1) -- (\xzero,0.1);
	\node at (\xzero,-0.1) [below] {$\xi$};

	\node at (\xlinks,\ylinks) [above left] {$f(a)$};
	\node at (\xzero,\yzero) [below right] {$f(\xi)$};
	\node at (\xrechts,\yrechts) [below right] {$f(b)$};

	\node[color=blue] at (2,{\yzero+\m*(2-\xzero)})
		[above,rotate=\a] {$f'(\xi)$};

	\draw[->] (-0.1,0) -- (6.5,0) coordinate[label={$x$}];
	\draw[->] (0,-0.1) -- (0,6.5) coordinate[label={right:$y$}];
\end{scope}

\begin{scope}[xshift=3.5cm]

	\coordinate (A) at (3,1);
	\coordinate (B) at (0,3);
	\coordinate (C) at (4,6.5);
	\coordinate (D) at (5,3);

	\xdef\xlinks{3}
	\xdef\ylinks{1}
	\xdef\xrechts{5}
	\xdef\yrechts{3}

	\pgfmathparse{(\yrechts-\ylinks)/(\xrechts-\xlinks)}
	\xdef\m{\pgfmathresult}
	\draw[color=blue] ({\xlinks-0.5},{\ylinks-0.5*\m})
		--
		({\xrechts+0.5},{\yrechts+0.5*\m});

	\draw[color=red,line width=1.4pt] plot[domain=0:1,samples=100]
		({(1-\x)*(1-\x)*(1-\x)*3+3*(1-\x)*(1-\x)*\x*0+3*(1-\x)*\x*\x*4+\x*\x*\x*5},{(1-\x)*(1-\x)*(1-\x)*1+3*(1-\x)*(1-\x)*\x*3+3*(1-\x)*\x*\x*6.5+\x*\x*\x*3});

	\xdef\tzero{0.483}
	\xdef\t{\tzero}
	\pgfmathparse{(1-\t)*(1-\t)*(1-\t)*3+3*(1-\t)*(1-\t)*\t*0+3*(1-\t)*\t*\t*4+\t*\t*\t*5};
	\xdef\xzero{\pgfmathresult}
	\pgfmathparse{(1-\t)*(1-\t)*(1-\t)*1+3*(1-\t)*(1-\t)*\t*3+3*(1-\t)*\t*\t*6.5+\t*\t*\t*3});
	\xdef\yzero{\pgfmathresult}

	\draw[color=blue] ({\xzero-2},{\yzero-2*\m})
		-- ({\xzero+2},{\yzero+2*\m});
	\fill[color=white] (\xzero,\yzero) circle[radius=0.08];
	\draw[color=blue] (\xzero,\yzero) circle[radius=0.08];

	\fill[color=white] (A) circle[radius=0.08];
	\fill[color=white] (D) circle[radius=0.08];

	\draw[color=red] (A) circle[radius=0.08];
	\draw[color=red] (D) circle[radius=0.08];

	\node at (A) [below right] {$(x(a),y(a))$};
	\node at (D) [left] {$(x(b),y(b))$};

	\node at (\xzero,\yzero) [right] {$(x(\xi),y(\xi))$};

	\pgfmathparse{atan(\m)}
	\xdef\a{\pgfmathresult}

	\node[color=blue] at ({\xlinks+1.7},{\ylinks+1.7*\m})
		[below,rotate=\a] {$\displaystyle m = \frac{y(b)-y(a)}{x(b)-x(a)}$};

	\node[color=blue]  at ({\xzero-1},{\yzero-1*\m}) [above,rotate=\a]
		{$\displaystyle\frac{y'(\xi)}{x'(\xi)}=m$};

	\draw[->] (-0.1,0) -- (6.5,0) coordinate[label={$x$}];
	\draw[->] (0,-0.1) -- (0,6.5) coordinate[label={right:$y$}];
\end{scope}

\end{tikzpicture}
\end{document}

