%
% rolle.tex -- Illustration zum Satz von Rolle
%
% (c) 2020 Prof Dr Andreas Müller, Hochschule Rapperswil
%
\documentclass[tikz]{standalone}
\usepackage{amsmath}
\usepackage{times}
\usepackage{txfonts}
\usepackage{pgfplots}
\usepackage{csvsimple}
\usetikzlibrary{arrows,intersections,math}
\begin{document}
\def\skala{4}
\begin{tikzpicture}[>=latex,thick,scale=\skala]

\draw[color=red,line width=1.4pt]
	plot[domain=-1:1,samples=200]
		({\x},{((((64*\x*\x-112)*\x*\x+56)*\x*\x-7)*\x)*(1-\x)*(1+\x)+0.1});

\draw[line width=0.5pt] (-1.1,0.1)--(1.1,0.1);

\draw (-1,{-0.1/\skala})--(-1,{0.1/\skala});
\draw[line width=0.1pt] (-1,0)--(-1,0.1);
\node at (-1,{-0.1/\skala}) [below] {$\mathstrut a$};
\node at (-1,0.1) [above left] {$f(a)$};

\draw (1,{-0.1/\skala})--(1,{0.1/\skala});
\draw[line width=0.1pt] (1,0)--(1,0.1);
\node at (1,{-0.1/\skala}) [below] {$\mathstrut b$};
\node at (1,0.1) [above right] {$f(b)$};

\fill[color=red] (-1,0.1) circle[radius={0.08/\skala}];
\fill[color=red] (1,0.1) circle[radius={0.08/\skala}];

\def\x{-0.2137}
\pgfmathparse{(((-576*\x*\x+1232)*\x*\x-840)*\x*\x+189)*\x*\x-7}
\xdef\m{\pgfmathresult}
\pgfmathparse{((((64*\x*\x-112)*\x*\x+56)*\x*\x-7)*\x)*(1-\x)*(1+\x)+0.1}
\xdef\y{\pgfmathresult}

\node[color=blue] at ({\x},{\y}) [above] {$f'(\xi)=0$};
%\draw[line width=0.1pt] ({\x-0.2},{\y-\m*0.2})--({\x+0.2},{\y+\m*0.2});
\draw[color=blue,line width=0.5pt] ({\x-0.2},{\y})--({\x+0.2},{\y});
\fill[color=blue] ({\x},{\y}) circle[radius={0.08/\skala}];

\draw[line width=0.1pt] ({\x},0)--({\x},{\y});
\draw ({\x},{-0.1/\skala})--({\x},{0.1/\skala});
\node at ({\x},{-0.1/\skala}) [below] {$\xi$};


\draw[->] (-0.8,-1.0)--(-0.8,1.2) coordinate[label={$y$}];
\draw[->] (-1.1,0)--(1.1,0) coordinate[label={$x$}];

\end{tikzpicture}
\end{document}

