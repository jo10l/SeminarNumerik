%
% bezier.tex -- Bezier-Kurven
%
% (c) 2020 Prof Dr Andreas Müller, Hochschule Rapperswil
%
\documentclass[tikz]{standalone}
\usepackage{amsmath}
\usepackage{times}
\usepackage{txfonts}
\usepackage{pgfplots}
\usepackage{csvsimple}
\usetikzlibrary{arrows,intersections,math}
\begin{document}
\def\skala{1}
\begin{tikzpicture}[>=latex,thick,scale=\skala]

\definecolor{darkgreen}{rgb}{0,0.6,0}

\begin{scope}[xshift=-5cm]
\coordinate (P0) at (0,0);
\coordinate (P1) at (4,1);
\draw[line width=1.4pt] (P0) -- (P1);
\fill[color=red] (P0) circle[radius=0.08];
\fill[color=red] (P1) circle[radius=0.08];
\node at (P0) [below] {$P_0$};
\node at (P1) [below] {$P_1$};
\def\punkt#1{({(1-#1)*0+#1*4},{(1-#1)*0+#1*1})}
\node at \punkt{0.6} [above] {$Q(t)$};
\fill[color=blue] \punkt{0.6} circle[radius=0.08];
\end{scope}

\begin{scope}[xshift=0cm]
\coordinate (P0) at (0,0);
\coordinate (P1) at (1,5);
\coordinate (P2) at (4,1);
\draw[color=gray,line width=1.4pt] (P0) -- (P1) -- (P2);
\def\t{0.6}
\def\punkt#1{({(1-#1)*0+#1*1},{(1-#1)*0+#1*5})}
\def\qunkt#1{({(1-#1)*1+#1*4},{(1-#1)*5+#1*1})}
\def\runkt#1{({(1-#1)*(1-#1)*0+2*#1*(1-#1)*1+#1*#1*4},{(1-#1)*(1-#1)*0+2*#1*(1-#1)*5+#1*#1*1})}
\fill[color=darkgreen] \punkt{\t} circle[radius=0.08];
\node at \punkt{\t} [left] {$Q_0(t)$};
\fill[color=darkgreen] \qunkt{\t} circle[radius=0.08];
\node at \qunkt{\t} [above right] {$Q_1(t)$};
\draw[color=darkgreen,line width=1.4pt] \punkt{\t} -- \qunkt{\t};
\draw[color=blue,line width=1.4pt]
	plot[domain=0:1,samples=100]
		({(1-\x)*(1-\x)*0+2*\x*(1-\x)*1+\x*\x*4},{(1-\x)*(1-\x)*0+2*\x*(1-\x)*5+\x*\x*1});
\fill[color=blue] \runkt{\t} circle[radius=0.08];
\node at \runkt{\t} [below] {$R(t)$};
\fill[color=red] (P0) circle[radius=0.08];
\fill[color=red] (P1) circle[radius=0.08];
\fill[color=red] (P2) circle[radius=0.08];
\node at (P0) [below] {$P_0$};
\node at (P1) [above] {$P_1$};
\node at (P2) [below] {$P_2$};
\end{scope}

\begin{scope}[xshift=5cm]
\coordinate (P0) at (0,0);
\coordinate (P1) at (1,5);
\coordinate (P2) at (3,4);
\coordinate (P3) at (4,1);
\draw[line width=1.4pt] (P0) -- (P1) -- (P2) -- (P3);
\fill[color=red] (P0) circle[radius=0.08];
\fill[color=red] (P1) circle[radius=0.08];
\fill[color=red] (P2) circle[radius=0.08];
\fill[color=red] (P3) circle[radius=0.08];
\node at (P0) [below] {$P_0$};
\node at (P1) [above] {$P_1$};
\node at (P2) [above right] {$P_2$};
\node at (P3) [below] {$P_3$};
\end{scope}

\end{tikzpicture}
\end{document}

