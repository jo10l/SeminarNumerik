%
% sin.tex -- interpolation für Sinus
%
% (c) 2020 Prof Dr Andreas Müller, Hochschule Rapperswil
%
\documentclass[tikz]{standalone}
\usepackage{amsmath}
\usepackage{times}
\usepackage{txfonts}
\usepackage{pgfplots}
\usepackage{csvsimple}
\usetikzlibrary{arrows,intersections,math}
\begin{document}
\def\skala{3}
\begin{tikzpicture}[>=latex,thick,scale=\skala]

\draw[->] (0,{-0.1/\skala})--(0,1.1) coordinate[label={right:$y$}];
\draw[->] ({-0.1/\skala},0)--(3.2,0) coordinate[label={$x$}];

\draw[color=red,line width=1.4pt]
	plot[domain=0:180,samples=100] ({3.14159*\x/180},{sin(\x)});

\draw[color=blue,line width=1.4pt]
	plot[domain=0:1,samples=100] ({3.1415*\x},{4*\x*(1-\x)});

\draw ({3.1415/2},{-0.1/\skala})--({3.1415/2},{0.1/\skala});
\node at ({3.1415/2},{-0.1/\skala}) [below] {$\displaystyle\frac{\pi}2$};
\draw ({3.1415},{-0.1/\skala})--({3.1415},{0.1/\skala});
\node at ({3.1415},{-0.1/\skala}) [below] {$\pi$};
\node at (0,{-0.1/\skala}) [below] {$0$};
\draw ({-0.1/\skala},1)--({0.1/\skala},1);
\node at ({-0.1/\skala},1) [left] {$1$};

\fill[color=red] (0,0) circle[radius={0.08/\skala}];
\fill[color=red] ({3.1415/2},1) circle[radius={0.08/\skala}];
\fill[color=red] (3.1415,0) circle[radius={0.08/\skala}];

\end{tikzpicture}
\end{document}

