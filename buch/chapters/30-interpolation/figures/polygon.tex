%
% polygon.tex -- Basis aus Polygonzügen
%
% (c) 2020 Prof Dr Andreas Müller, Hochschule Rapperswil
%
\documentclass[tikz]{standalone}
\usepackage{amsmath}
\usepackage{times}
\usepackage{txfonts}
\usepackage{pgfplots}
\usepackage{csvsimple}
\usetikzlibrary{arrows,intersections,math}
\begin{document}
\def\skala{1.95}
\def\yskala{0.6}

\begin{tikzpicture}[>=latex,thick,scale=\skala]

\def\bild#1#2{
\begin{scope}[yshift=#1]

\draw[->] (-0.4,0)--(6.5,0) coordinate[label={$x$}];
\draw[->] (0,-0.35)--(0,0.85) coordinate[label={right:$y$}];

\begin{scope}
\clip (-0.0,-0.4) rectangle (6.0,0.75);
\draw[color=red,line width=1.4pt] (-2,0)--({#2-1},0)--({#2},\yskala)--({#2+1},0)--(8,0);
\end{scope}

\fill[color=red] (#2,\yskala) circle[radius={0.08/\skala}];
\draw ({-0.1/\skala},\yskala)--({0.1/\skala},\yskala);
\node at ({-0.1/\skala},\yskala) [left] {$1$};
\node at (2.5,\yskala) [above] {$j=#2$};
\foreach \x in {1,...,6}{
	\draw (\x,{-0.1/\skala})--(\x,{0.1/\skala});
	\node at (\x,{-0.1/\skala}) [below] {$\x$};
}
\node at ({-0.1/\skala},{-0.1/\skala}) [below left] {$0$};
\end{scope}
}

\bild{0cm}{0}
\bild{-1.4cm}{1}
\bild{-2.8cm}{2}
\bild{-4.2cm}{3}
\bild{-5.6cm}{4}
\bild{-7.0cm}{5}
\bild{-8.4cm}{6}

\end{tikzpicture}
\end{document}

