%
% l.tex -- template for standalon tikz images
%
% (c) 2020 Prof Dr Andreas Müller, Hochschule Rapperswil
%
\documentclass[tikz]{standalone}
\usepackage{amsmath}
\usepackage{times}
\usepackage{txfonts}
\usepackage{pgfplots}
\usepackage{csvsimple}
\usetikzlibrary{arrows,intersections,math}
\begin{document}
\def\skala{10}
\begin{tikzpicture}[>=latex,thick,scale=\skala]

\draw[color=red,line width=1.4pt]
	plot[domain=0:2,samples=100]
		({\x/2},{\x*(\x-1)*(\x-2)/8});

\draw[color=red,line width=1.4pt]
	plot[domain=0:3,samples=100]
		({\x/3},{\x*(\x-1)*(\x-2)*(\x-3)/81});

\draw[color=red,line width=1.4pt]
	plot[domain=0:4,samples=100]
		({\x/4},{\x*(\x-1)*(\x-2)*(\x-3)*(\x-4)/1024});

\draw[color=red,line width=1.4pt]
	plot[domain=0:5,samples=100]
		({\x/5},{\x*(\x-1)*(\x-2)*(\x-3)*(\x-4)*(\x-5)/15625});

\draw[color=blue,line width=1.4pt]
	plot[domain=0:10,samples=100]
		({\x/10},{(\x/10)*((\x-1)/10)*((\x-2)/10)*((\x-3)/10)*((\x-4)/10)*((\x-5)/10)*((\x-6)/10)*((\x-7)/10)*((\x-8)/10)*((\x-9)/10)*((\x-10)/10)});



\end{tikzpicture}
\end{document}

