%
% bary.tex
%
% (c) 2020 Prof Dr Andreas Müller, Hochschule Rapperswil
%
\section{Baryzentrische Formeln für Interpolationspolynome
\label{buch:section:baryzentrisch}}
Die Interpolationspolynome von Lagrange und Hermite haben 
in der bis jetzt gezeigten Form das folgende grundlegende Problem.
Sie sind definiert über das Produkt
\[
(x-x_0)(x-x_1)\dots (x-x_{n-1})(x-x_n).
\]
Ist die Zahl der Stützstellen gross und liegen erstrecken sich die 
Stützstellen über einen grossen Bereich, dann sind einzelne Faktoren
$(x-x_i)$ immer gross.
Zudem tritt bei der Berechnung eines Wertes in inmittelbarer 
Nähe der Stützstelle $x_j$ in dem Faktor $(x-x_j)$ Auslöschung auf.
Der grosse relative Fehler dieses Faktors wird durch die anderen Faktoren
zu einem grossen absoluten Fehler aufgeblasen.

Andererseits ist klar, dass sich das Interpolationspolynom vor allem
in der Nähe einer Stützstelle ändern sollte, wenn man den an der
Stützstelle ändert.
Die anderen Stützstellen sollten also nur einen geringen Einfluss auf
den Wert des Interpolationspolynoms haben.
Dies geht aus der bisherigen Form des Interpolationspolynoms ebenfalls
nicht hervor.

Gesucht ist also eine Form des Interpolationspolynoms, welche einsichtig
macht, dass Änderungen von Stützwerten sich vor allem in der nähe der
betroffenen Stützstelle auswirken und die auch bei einer grossen Zahl
von Stützstellen stabil sind.


