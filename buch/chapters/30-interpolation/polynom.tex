%
% polynom.tex
%
% (c) 2020 Prof Dr Andreas Müller, Hochschule Rapeprswil
%
\section{Interpolationspolynom
\label{buch:section:interpolationspolynom}}
\rhead{Lagrange-Interpolationspolynome}
In diesem Abschnitt wird das folgende Problem gelöst.
\begin{aufgabe}[Interplations-Polynom]
Gegeben Stützstellen
\[
a=x_0<x_1<x_2<\dots < x_{n-1}<x_n=b
\]
und Funktionswerte $f_i, 0\le i\le n$, finde ein Polynome $l(x)$
mit der Eigenschaft $l(x_i)=f_i$ für alle $i=0,1,\dots n$.
\end{aufgabe}
Gegeben sind also $n+1$ Bedingungen, die das Polynom erfüllen muss.
Abgesehen von trivialen Fällen wie dem Null-Polynom, muss ein Polynom
im Allgemeinen mindestens den Grad $n$ haben, damit alle 
Bedingungen durch geeignete Wahl der $n+1$ Koeffizienten erfüllt werden können.
Man könnte das Polynom nämlich in der Form
\[
l(x)
=
a_nx^n + a_{n-1}x^{n-1}+\dots+a_1x+a_0
\]
ansetzen und die Stützstellen einsetzen.
Lösung des Gleichungssystem
\begin{equation}
\begin{linsys}{5}
a_Nx_0^N &+& a_{N-1}x_0^{N-1} &+& \dots &+& a_1x_0 &+& a_0x_0 &=& f_0 \\[5pt]
a_Nx_1^N &+& a_{N-1}x_1^{N-1} &+& \dots &+& a_1x_1 &+& a_0x_1 &=& f_1 \\[5pt]
\vdots   & &    \vdots        & & \ddots& & \vdots & & \vdots & & \vdots\\[5pt]
a_Nx_n^N &+& a_{N-1}x_n^{N-1} &+& \dots &+& a_1x_n &+& a_0x_n &=& f_n 
\end{linsys}
\end{equation}
liefert dann die gesuchten Koeffizienten.
Dieser Weg ist allerdings sehr aufwendig, die Lösung eines linearen
Gleichungssystems mit dem Gauss-Algorithmus benötigt $O(n^3)$ Operationen.
Die sehr spezielle Struktur des Gleichungssystems sollte ermöglichen,
das Polynom $l(x)$ auf direkterem Weg zu ermitteln.

%
% Interpolationspolynom bestimmen
%
\subsection{Bestimmung des Interpolationspolynoms
\label{buch:section:interpolation:bestimmung}}
Das allgemeine Interpolationsproblem kann leicht gelöst werden, wenn
das folgende spezielle Interpolationsproblem gelöst ist.

\begin{aufgabe}[Spezielle Interpolationspolynome]
Gegeben die Stützstellen
\[
a=x_0<x_1<x_2<\dots <x_{n-1}<x_n=b,
\]
finde Polynome $l_j$ vom Grad $n$ derart, dass
\[
l_j(x_i) = \delta_{ij}=\begin{cases}
1&\qquad i=j\\
0&\qquad\text{sonst.}
\end{cases}
\]
\end{aufgabe}

Jedes der Interpolationspolynome $l_j$ hat Grad $n$, also hat auch eine
beliebige Linearkombination den Grad höchstens $n$.
Die Linearkombination
\[
l(x) = \sum_{j=0}^n f_j l_j(x)
\]
ist das gesuchte Interpolationspolynom, wie Einsetzen von $x_i$ in
\[
l(x_i)
=
\sum_{j=0}^n f_jl_j(x_i)
=
\sum_{j=0}^n f_j\delta_{ij}
=
f_i
\]
bestätigt.

\begin{beispiel}
Ein besonders einfacher Fall ist $n=1$.
Gesucht ist eine lineare Funktion $l(x)=a_1x+a_0$ derart, dass
$l(x_0)=f_0$ und $l(x_1)=f_1$.
Polynome $l_0$ und $l_1$ können leicht angegeben werden:
\[
l_0(x) = \frac{x_1-x}{x_1-x_0}
\qquad\text{und}\qquad
l_1(x) = \frac{x-x_0}{x_1-x_0}
\]
haben die die geforderten Eigenschaften.
Die gesuchte Interplationsfunktion ist daher
\[
l(x)
=
\frac{x_1-x}{x_1-x_0}f_0 + \frac{x-x_0}{x_1-x_0} f_1
=
x \frac{f_1-f_0}{x_1-x_0}   + \frac{x_1f_0-x_0f_1}{x_1-x_0}.
\]
Der Koeffizient von $x$ ist wie erwartet die Steigung der Geraden durch
die Punkte $(x_0,f_0)$ und $(x_1,f_1)$.
\end{beispiel}

Ein Polynom vom Grad $n+1$, welches in {\em allen} Stützstellen verschwindet,
ist leicht zu finden, es ist 
\[
(x-x_0)(x-x_1)(x-x_2)\dots (x-x_{n-1})(x-x_n).
\]
Ein Polynom, welches nur an der Stützstelle $x_j$ {\em nicht} verschwindet,
ensteht, indem man den Faktor $(x-x_j)$ weglässt, es hat den Grad $n$.
Wir führen dafür die Notation
\[
(x-x_0)(x-x_1)(x-x_2)\dots \widehat{(x-x_j)}\dots (x-x_{n-1}(x-x_n),
\]
der Hut bedeutet, dass dieser Faktor weggelassen werden soll.
Allerdings hat dieses Polynom nicht den geforderten Wert $1$, man muss es
also noch mit einer geeigneten Konstante multiplizieren.
Das gesuchte Polynom $l_j(x)$ hat daher die Form
\[
l_j(x)
=
c_j(x-x_0)(x-x_1)(x-x_2)\dots \widehat{(x-x_j)}\dots (x-x_{n-1}(x-x_n).
\]
Einesetzen von $x_j$ ergibt
\[
l_j(x_j) = 1 = 
c_j(x_j-x_0)(x_j-x_1)(x_j-x_2)\dots \widehat{(x_j-x_j)}\dots(x_j-x_{n-1}(x_j-x_n),
\]
die Konstante $c_j$ ist daher
\[
c_j = \prod_{i=0\atop i\ne j}^n \frac{1}{x_j-x_i}.
\]

\begin{beispiel}
Man finde ein Polynome, welches $l(0)=l(1)=0$ und $l(\frac12)=1$
erfüllt.
Wegen $f_0=f_2=0$ ist nur das Polynome $l_1$ zu ermitteln, es ist
\[
l(x) = l_1(x)
=
\frac{(x-x_0)(x-x_2)}{(x_1-x_0)(x_1-x_2)}
=
\frac{x(x-1)}{\frac12(\frac12-1)}
=
\frac14x(1-x).
\qedhere
\]
\end{beispiel}

%
% Fehler des Interpolationspolynoms
%
\subsection{Fehler des Interpolationspolynoms
\label{buch:section:interpolation:fehler}}

%
% Tschebyscheff Interpolation
%
\subsection{Wahl der Stützstellen und Tschebyscheff-Interpolationspolynom
\label{buch:section:interpolation:tschebyscheff}}

