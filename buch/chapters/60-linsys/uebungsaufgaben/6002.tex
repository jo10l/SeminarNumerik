Berechnen Sie die Eigenwerte der Matrix
\[
A=
\begin{pmatrix}
a&\frac12\\
\frac12&0
\end{pmatrix}.
\]
\begin{teilaufgaben}
\item
Wie gross darf $a$ sein, damit $\varrho(A) < 1$ ist?
\item
Berechnen Sie die Konditionszahl $\kappa(A)$ in Abhängigkeit von $a$.
\end{teilaufgaben}

\begin{loesung}
\begin{figure}
\centering
\begin{tikzpicture}[>=latex,thick]

\fill[color=darkgreen!10]  ({-3/4},-1) rectangle ({3/4},1);
\draw[line width=0.7pt,color=darkgreen] (-3,1) -- (3,1);
\draw[line width=0.7pt,color=darkgreen] (-3,-1) -- (3,-1);

\begin{scope}
\clip (-3,0) rectangle (3,3.8);
\draw[color=orange,line width=1.4pt]
	plot[domain=-3:3,samples=100] (\x,{2*(\x*\x+abs(\x)*sqrt(\x*\x+1))+1});
\node[color=orange] at (-0.5,2.5) [left] {$\kappa(A)$};
\end{scope}

\draw[->] (-3.1,0) -- (3.3,0) coordinate[label=$a$];
\draw[->] (0,-4) -- (0,4) coordinate[label={right:$\lambda$}];

\draw[line width=0.7pt] (-0.05,1)--(0.05,1);
\node at (0,1) [below left] {$1$};
\draw[line width=0.7pt] (-0.05,-1)--(0.05,-1);
\node at (0,-1) [below right] {$-1$};

\node at (0,0) [below right] {$0$};

\foreach \x in {1,2,3}{
	\draw[line width=0.7pt] (\x,-0.05)--(\x,0.05);
	\node at (\x,0.05) [above] {$\x$};
	\draw[line width=0.7pt] (-\x,-0.05)--(-\x,0.05);
	\node at (-\x,-0.05) [below] {$-\x$};
}

\draw[line width=0.4pt] (-3,-3) -- (3,3);
\draw[color=red,line width=1.4pt]
	plot[domain=-3:3,samples=100] ({\x},{(\x+sqrt(\x*\x+1))/2});
\draw[color=blue,line width=1.4pt]
	plot[domain=-3:3,samples=100] ({\x},{(\x-sqrt(\x*\x+1))/2});
\node[color=red] at (2,{(2+sqrt(2*2+1))/2}) [above left] {$\lambda_+(a)$};
\node[color=blue] at (-2,{(-2-sqrt(2*2+1))/2}) [below right] {$\lambda_-(a)$};
\end{tikzpicture}
\caption{Eigenwerte $\lambda_{\pm}$ von $A$ in Abhängigkeit vom Parameter $a$
sowie die Konditionszahl $\kappa(A)$.
\label{6002:graphen}}
\end{figure}
Das charakteristische Polynom ist
\[
\chi_A(\lambda)
=
\det (A-\lambda E)
=
\left|\begin{matrix}
a-\lambda  & \frac12  \\
  \frac12  & -\lambda
\end{matrix}\right|
=
(a-\lambda)(-\lambda)-\frac14
=
\lambda^2-a\lambda-\frac14
\]
und hat die Nullstellen
\[
\lambda_{\pm}(a)
=
\frac{a}2 \pm \sqrt{\frac{a^2}4+\frac14}
=
\frac{a\pm \sqrt{a^2+1}}{2}.
\]
Abbildung~\eqref{6002:graphen} zeigt die Graphen von $\lambda_{\pm}$
in Abhängigkeit von $a$.
\begin{teilaufgaben}
\item
Es ist leicht zu sehen, dass es ein um den Nullpunkt symmetrisches
Interval von $a$-Werten gibt, in dem $|\lambda_{\pm}|<1$ ist.
Man kann Interval finden, indem man die Schranken $\pm1$ für $\lambda$
in das chrakteristische Polynom einsetzt:
\[
0=\chi_A(\pm1)
=
1\mp a-\frac14
=
\frac34
\mp a
\qquad\Rightarrow\qquad a=\pm\frac34.
\]
Für $a\in(-\frac34,\frac34)$ ist der Spektralradius also $\varrho(A)<1$.
Man kann natürlich auch eine Formel für den Spektralradius angeben, es
ist
\[
\varrho(A)
=
\frac{|a|+\sqrt{a^2 + 1}}2.
\]
Daraus kann man auch ablesen, dass der kleinstmögliche Spektralradius $\frac12$
ist.
\item
Die Konditionszahl ist das Verhältnis des grössten zum kleinsten Eigenwert.
Für $a>0$ ist dies $-\lambda_+/\lambda_-$, also
\begin{align}
\kappa(A)
&=
-\frac{a+\sqrt{a^2+1}}{a-\sqrt{a^2+1}}
=
\frac{\sqrt{a^2+1}+a}{\sqrt{a^2+1}-a}
=
\frac{(\sqrt{a^2+1}+a)^2}{(a^2+1)-a^2}
\notag
\\
&=
a^2 + 2a\sqrt{a^2+1}+a^2+1
=
2a(a +\sqrt{a^2+1}) + 1.
\label{6002:kappa}
\end{align}
Für negative $a$ ist $\lambda_-$ der betragsgrössere Eigenwert.
Indem man $a$ wo nötig durch $|a|$ ersetzt, kann man die Formel
\eqref{6002:kappa} zu einer geraden Funktion machen.
Damit wird die Konditionszahl von $A$ zu
\[
\kappa(A)
=
2(a^2 +|a|\sqrt{a^2+1}) + 1
\]
für beliebige $a$.
\qedhere
\end{teilaufgaben}
\end{loesung}

