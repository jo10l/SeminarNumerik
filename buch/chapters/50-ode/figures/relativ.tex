%
% relativ.tex -- relativer Fehler des Euler-Verfahrens
%
% (c) 2020 Prof Dr Andreas Müller, Hochschule Rapperswil
%
\documentclass[tikz]{standalone}
\usepackage{amsmath}
\usepackage{times}
\usepackage{txfonts}
\usepackage{pgfplots}
\usepackage{csvsimple}
\usetikzlibrary{arrows,intersections,math}
\begin{document}
\def\skala{4}
\begin{tikzpicture}[>=latex,thick,scale=\skala]

\def\q{0.99}

\xdef\yvalue{1}
\foreach \k in {0,...,199}{
	\draw[color=red,line width=1.4pt]
		({0.01*\k},{1-\yvalue})
		--
		({0.01*(\k+1)},{1-\yvalue*\q});
	\pgfmathparse{\q*\yvalue}
	\xdef\yvalue{\pgfmathresult}
}

\draw[->] ({-0.1/\skala},0)--(2.1,0) coordinate[label={$k$}];
\draw[->] (0,{-0.1/\skala})--(0,1.1) coordinate[label={right:relativer Fehler}];

\end{tikzpicture}
\end{document}

