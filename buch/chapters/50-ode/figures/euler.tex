%
% euler.tex -- Illustration Euler approximation
%
% (c) 2020 Prof Dr Andreas Müller, Hochschule Rapperswil
%
\documentclass[tikz]{standalone}
\usepackage{amsmath}
\usepackage{times}
\usepackage{txfonts}
\usepackage{pgfplots}
\usepackage{csvsimple}
\usetikzlibrary{arrows,intersections,math}
\begin{document}
\def\skala{7}
\begin{tikzpicture}[>=latex,thick,scale=\skala]

% Richtungsfeld
\def\d{0.05}
\foreach \x in {0,\d,...,1.61}{
	\foreach \y in {0,\d,...,1.01}{
		\draw[line width=0.1pt]
			({\x-cos(atan(-\y))*\d/3},{\y-sin(atan(-\y))*\d/3})
			--
			({\x+cos(atan(-\y))*\d/3},{\y+sin(atan(-\y))*\d/3});
	}
}

% Label für Richtungsfeld
\xdef\x{1.3}
\xdef\y{0.8}
\fill[color=white,opacity=0.8]
	({\x-0.1},{\y-0.05}) rectangle ({\x+0.1},{\y+0.05});
\node at (\x,\y) {$y'=-\alpha y$};

% Label für exakte Lösungskurve
\xdef\x{0.8}
\fill[color=white,opacity=0.8]
	({\x},{exp(-\x)}) rectangle ({\x+0.4},{exp(-\x)+0.1});
\node[color=blue] at ({\x+0.2},{exp(-\x)+0.05}) {$y(x)=y_0e^{-\alpha x}$};


\def\euler#1#2#3{
\xdef\y{1}
\xdef\x{0}
\foreach \k in {0,...,#2}{
	\draw[color=red,line width=1.2pt] ({\x},{\y})--({\x+#1},{\y*(1-#1)});
	\fill[color=red] ({\x+#1},{\y*(1-#1)}) circle[radius={0.06/\skala}];
	\pgfmathparse{\x+#1}
	\xdef\x{\pgfmathresult}
	\pgfmathparse{\y*(1-#1)}
	\xdef\y{\pgfmathresult}
}
\draw[color=red,line width=0.1pt] ({\x},{\y})--(1.67,{\l});
\node[color=red] at (1.67,{\l}) [right] {#3};
\pgfmathparse{\l+0.06}
\xdef\l{\pgfmathresult}
}

\xdef\l{0.06}

\euler{0.8}{1}{$n=\phantom{0}2$}
\euler{0.4}{3}{$n=\phantom{0}4$}
\euler{0.2}{7}{$n=\phantom{0}8$}
\euler{0.1}{15}{$n=16$}

\draw[color=blue,line width=1.5pt]
	plot[domain=0:1.60,samples=100] ({\x},{exp(-\x)});

\draw[->] ({-0.1/\skala},0)--(1.85,0) coordinate[label={$x$}];
\draw[->] (0,{-0.1/\skala})--(0,1.1) coordinate[label={right:$y$}];

\end{tikzpicture}
\end{document}

