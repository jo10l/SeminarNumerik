%
% zahlensysteme.tex
%
% (c) 2020 Prof Dr Andreas Müller, Hochschule Rapperswil
%
\section{Zahlensysteme
\label{buch:section:zahlensysteme}}
\rhead{Zahlensysteme}
Auf modernen Allzweck-Prozessoren steht eine ganze Reihe verschiedener
numerischer Datentypen mit unterschiedlichen Eigenschaften
bezüglich Geschwindigkeit und Fehlerverhalten zur Verfügung.
In diesem Abschnitt sollen sie vorgestellt und miteinander verglichen
werden.
Es gilt, den für eine Berechnung zweckmässigsten Typen zu wählen,
wobei Speicherbedarf, Laufzeit und Parallelisierbarkeit wesentliche
Aspekte sind.

Microcontroller sind im Vergleich zu Allzweckprozessoren oft stark
eingeschränkt.
Meist sind nur Ganzahltypen mit oft sehr beschränkter Länge implementiert.
Manchmal kann die Arithmetik-Einheit des Prozessors nicht einmal eine
Multiplikation in Hardware ausführen, sie muss in Software nachgebildet
werden.
Für Floatingpoint Operationen muss oft Bibliotheken zurückgegriffen
werden, die den Speicherbedarf erhöhen und langsam sind.
Die Implementation von numerischen Berechnungen in eingebetteten Anwendungen
ist daher mit besonderen Herausforderungen konfrontiert.

Dieselbe Schwierigkeit haben auch Allzweck-Prozessoren wenn die
Genauigkeitsanforderungen die Möglichkeiten der von der Prozessor-Hardware
implementierten Typen übersteigt.
Dieser Fall tritt beispielsweise bei Berechnungen in der Kryptographie auf,
wo oft mit Ganzzahlen mit Tausenden von Stellen gerechnet werden muss.
Im Abschnitt~\ref{buch:subsection:mp} mit der GNU Multiprecision-Library
ein Beispiel einer Bibliothek vorgestlelt.


\subsection{Festkommazahlen
\label{buch:subsection:integers}}

\subsection{Gleitkommazahlen
\label{buch:subsection:floatinpoing}}

\subsection{Hochpräzisionsbibliotheken
\label{buch:subsection:mp}}
