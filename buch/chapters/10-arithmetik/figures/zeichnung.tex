%
% zeichnung.tex
%
% (c) 2020 Prof Dr Andreas Müller, Hochschule Rapperswil
%
\xdef\xzero{\x}

\def\f{
	\pgfmathparse{\A*(\x-\fx)+\fx+(\B+(\x-\fx)*(\C + \D*(\x-\fx)))*(\x-\fx)*(\x-\fx)}
	\xdef\y{\pgfmathresult}
}
\begin{scope}
\clip (0,0) rectangle (6,6);
\draw[color=red,line width=1.4pt]
	plot[domain=0:6,samples=100]
		({\x},{\A*(\x-\fx)+\fx+(\B+(\x-\fx)*(\C + \D*(\x-\fx)))*(\x-\fx)*(\x-\fx))});
\draw[color=blue,line width=0.2pt] (0,{2*\fx})--({2*\fx},0);
\end{scope}

\def\schritt{
	\xdef\yold{\y}
	\f
	\ifthenelse{\k<\pfeile}{
		\draw[->,color=gray] (\x,\yold)--(\x,\y);
		\draw[->,color=gray] (\x,\y)--(\y,\y);
	}{
		\draw[color=gray] (\x,\yold)--(\x,\y);
		\draw[color=gray] (\x,\y)--(\y,\y);
	}
	\xdef\x{\y}
}

\def\pfad{
	\begin{scope}
	\clip (0,0) rectangle (6,6);
	\foreach \k in {1,...,\anzahl}{
		\schritt
	}
	\end{scope}
}
\pfad

\draw[color=blue,line width=1.4pt] (-0.1,-0.1) -- (6.1,6.1);
\fill[color=red] (\fx,\fx) circle[radius=0.06];
\draw[->] (-0.1,0)--(6.3,0) coordinate[label={$x$}];
\draw[->] (0,-0.1)--(0,6.3) coordinate[label={right:$y$}];
\draw (\xzero,-0.1)--(\xzero,0.1);
\node at (\xzero,-0.1) [below] {$x_0\mathstrut$};
\draw[color=red,line width=0.5pt] (\fx,0)--(\fx,\fx)--(0,\fx);
\draw[color=red] (\fx,-0.1)--(\fx,0.1);
\node[color=red] at (\fx,-0.1) [below] {$x^*\mathstrut$};
\draw[color=red] (-0.1,\fx)--(0.1,\fx);
%\node at (-0.1,\fx) [left] {$x^*=f(x^*)$};

\node[color=blue] at (5.8,5.8) [above,rotate=45] {$x=y$};
