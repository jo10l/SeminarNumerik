%
% iteration.tex -- Iteration einer Funktion
%
% (c) 2019 Prof Dr Andreas Müller, Hochschule Rapperswil
%
\documentclass[tikz]{standalone}
\usepackage{amsmath}
\usepackage{times}
\usepackage{txfonts}
\usepackage{pgfplots}
\usepackage{csvsimple}
\usetikzlibrary{arrows,intersections,math}
\begin{document}
\def\skala{5}
\begin{tikzpicture}[>=latex,thick,scale=\skala]

\draw[color=blue,line width=1.4pt] (-0.1,-0.1)--(2.1,2.1);


\def\xn{1.99}

\def\XX{\xn}

\foreach \x in {1,2,3,4}{
\pgfmathparse{sqrt(\XX)}
\xdef\YY{\pgfmathresult}
\draw[color=gray,line width=2pt] (\XX,\XX)--(\XX,\YY)--(\YY,\YY);
\xdef\XX{\YY}
}

\pgfmathparse{sqrt(\xn)}
\xdef\yn{\pgfmathresult}

\draw[line width=0.1pt] (\xn,0)--(\xn,\yn);
\draw[line width=0.1pt] (0,\yn)--(\xn,\yn);

\fill[color=red] (\xn,\yn) circle[radius={0.1/\skala}];

\draw (\xn,{-0.1/\skala})--(\xn,{0.1/\skala});
\node at (\xn,{-0.1/\skala}) [below] {$x_n\mathstrut$};

\draw ({-0.1/\skala},\yn)--({0.1/\skala},\yn);
\node at ({-0.1/\skala},\yn) [left] {$y_n$};

\node at (0.1,\yn) [above right] {$y_n=f(x_n)$};

\draw[line width=0.1pt] (\yn,\yn)--(\yn,0);
%\draw (\yn,{-0.1/\skala})--(\yn,{0.1/\skala});
%\node at (\yn,{-0.1/\skala}) [below] {$x_{n+1}$};

\node[color=blue] at (\yn,\yn) [above left] {$x_{n+1}=y_n$};
\fill[color=blue] (\yn,\yn) circle[radius={0.1/\skala}];

\xdef\xn{\yn}
\pgfmathparse{sqrt(\xn)}
\xdef\yn{\pgfmathresult}

\draw[line width=0.1pt] (\xn,0)--(\xn,\yn);
\draw[line width=0.1pt] (0,\yn)--(\xn,\yn);

\fill[color=red] (\xn,\yn) circle[radius={0.1/\skala}];

\draw (\xn,{-0.1/\skala})--(\xn,{0.1/\skala});
\node at (\xn,{-0.1/\skala}) [below] {$x_{n+1}\mathstrut$};

\draw ({-0.1/\skala},\yn)--({0.1/\skala},\yn);
\node at ({-0.1/\skala},\yn) [left] {$y_{n+1}$};

\node at (0.1,\yn) [above right] {$y_{n+1}=f(x_{n+1})$};

\node[color=blue] at (\yn,\yn) [above left] {$x_{n+2}=y_{n+1}$};
\fill[color=blue] (\yn,\yn) circle[radius={0.1/\skala}];

\xdef\xn{\yn}

\draw (\xn,{-0.1/\skala})--(\xn,{0.1/\skala});
\node at (\xn,{-0.1/\skala}) [below] {$x_{n+2}\mathstrut$};

\draw[line width=0.1pt] (\xn,0)--(\xn,\yn);

\draw[color=red,line width=1.4pt]
	plot[domain=0:sqrt(2),samples=100]
	({\x*\x},{\x});

\draw[line width=0.1pt] (1,1)--(1,0);
\draw (1,{-0.1/\skala})--(1,{0.1/\skala});
\node[color=red] at (1,{-0.1/\skala}) [below] {$x^*$};
\fill[color=red] (1,1) circle[radius={0.1/\skala}];

\draw[->] (-0.1,0)--(2.1,0) coordinate[label={$x$}];
\draw[->] (0,-0.1)--(0,2.1) coordinate[label={right:$y$}];

\end{tikzpicture}
\end{document}

