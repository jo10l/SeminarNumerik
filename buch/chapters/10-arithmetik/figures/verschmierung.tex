%
% verschmierung.tex -- Illustration des Verschmierungseffektes
%
% (c) 2019 Prof Dr Andreas Müller, Hochschule Rapperswil
%
\documentclass[tikz]{standalone}
\usepackage{amsmath}
\usepackage{times}
\usepackage{txfonts}
\usepackage{pgfplots}
\usepackage{csvsimple}
\usetikzlibrary{arrows,intersections,math}
\begin{document}
\def\pfadf#1{
\fill[color=#1] (0,0) circle[radius=\r];
\fill[color=#1] ({1*\xskala},{1.48*\yskala}) circle[radius=\r];
\fill[color=#1] ({2*\xskala},{2.65*\yskala}) circle[radius=\r];
\fill[color=#1] ({3*\xskala},{3.65*\yskala}) circle[radius=\r];
\fill[color=#1] ({4*\xskala},{4.53*\yskala}) circle[radius=\r];
\fill[color=#1] ({5*\xskala},{5.31*\yskala}) circle[radius=\r];
\fill[color=#1] ({6*\xskala},{6.01*\yskala}) circle[radius=\r];
\fill[color=#1] ({7*\xskala},{6.64*\yskala}) circle[radius=\r];
\fill[color=#1] ({8*\xskala},{7.21*\yskala}) circle[radius=\r];
\fill[color=#1] ({9*\xskala},{7.73*\yskala}) circle[radius=\r];
\fill[color=#1] ({10*\xskala},{8.21*\yskala}) circle[radius=\r];
\fill[color=#1] ({11*\xskala},{8.65*\yskala}) circle[radius=\r];
\fill[color=#1] ({12*\xskala},{9.05*\yskala}) circle[radius=\r];
\fill[color=#1] ({13*\xskala},{9.41*\yskala}) circle[radius=\r];
\fill[color=#1] ({14*\xskala},{9.74*\yskala}) circle[radius=\r];
\fill[color=#1] ({15*\xskala},{10.04*\yskala}) circle[radius=\r];
\fill[color=#1] ({16*\xskala},{10.31*\yskala}) circle[radius=\r];
\fill[color=#1] ({17*\xskala},{10.56*\yskala}) circle[radius=\r];
\fill[color=#1] ({18*\xskala},{10.78*\yskala}) circle[radius=\r];
\fill[color=#1] ({19*\xskala},{10.98*\yskala}) circle[radius=\r];
\fill[color=#1] ({20*\xskala},{11.16*\yskala}) circle[radius=\r];
\fill[color=#1] ({21*\xskala},{11.31*\yskala}) circle[radius=\r];
\fill[color=#1] ({22*\xskala},{11.45*\yskala}) circle[radius=\r];
\fill[color=#1] ({23*\xskala},{11.56*\yskala}) circle[radius=\r];
\fill[color=#1] ({24*\xskala},{11.66*\yskala}) circle[radius=\r];
\fill[color=#1] ({25*\xskala},{11.74*\yskala}) circle[radius=\r];
\fill[color=#1] ({26*\xskala},{11.80*\yskala}) circle[radius=\r];
\fill[color=#1] ({27*\xskala},{11.85*\yskala}) circle[radius=\r];
\fill[color=#1] ({28*\xskala},{11.88*\yskala}) circle[radius=\r];
\fill[color=#1] ({29*\xskala},{11.89*\yskala}) circle[radius=\r];
\fill[color=#1] ({30*\xskala},{11.89*\yskala}) circle[radius=\r];
\fill[color=#1] ({31*\xskala},{11.88*\yskala}) circle[radius=\r];
\fill[color=#1] ({32*\xskala},{11.85*\yskala}) circle[radius=\r];
\fill[color=#1] ({33*\xskala},{11.81*\yskala}) circle[radius=\r];
\fill[color=#1] ({34*\xskala},{11.75*\yskala}) circle[radius=\r];
\fill[color=#1] ({35*\xskala},{11.69*\yskala}) circle[radius=\r];
\fill[color=#1] ({36*\xskala},{11.61*\yskala}) circle[radius=\r];
\fill[color=#1] ({37*\xskala},{11.51*\yskala}) circle[radius=\r];
\fill[color=#1] ({38*\xskala},{11.41*\yskala}) circle[radius=\r];
\fill[color=#1] ({39*\xskala},{11.30*\yskala}) circle[radius=\r];
\fill[color=#1] ({40*\xskala},{11.17*\yskala}) circle[radius=\r];
\fill[color=#1] ({41*\xskala},{11.04*\yskala}) circle[radius=\r];
\fill[color=#1] ({42*\xskala},{10.89*\yskala}) circle[radius=\r];
\fill[color=#1] ({43*\xskala},{10.74*\yskala}) circle[radius=\r];
\fill[color=#1] ({44*\xskala},{10.57*\yskala}) circle[radius=\r];
\fill[color=#1] ({45*\xskala},{10.39*\yskala}) circle[radius=\r];
\fill[color=#1] ({46*\xskala},{10.21*\yskala}) circle[radius=\r];
\fill[color=#1] ({47*\xskala},{10.01*\yskala}) circle[radius=\r];
\fill[color=#1] ({48*\xskala},{9.81*\yskala}) circle[radius=\r];
\fill[color=#1] ({49*\xskala},{9.59*\yskala}) circle[radius=\r];
\fill[color=#1] ({50*\xskala},{9.37*\yskala}) circle[radius=\r];
\fill[color=#1] ({51*\xskala},{9.14*\yskala}) circle[radius=\r];
\fill[color=#1] ({52*\xskala},{8.90*\yskala}) circle[radius=\r];
\fill[color=#1] ({53*\xskala},{8.66*\yskala}) circle[radius=\r];
\fill[color=#1] ({54*\xskala},{8.40*\yskala}) circle[radius=\r];
\fill[color=#1] ({55*\xskala},{8.14*\yskala}) circle[radius=\r];
\fill[color=#1] ({56*\xskala},{7.87*\yskala}) circle[radius=\r];
\fill[color=#1] ({57*\xskala},{7.59*\yskala}) circle[radius=\r];
\fill[color=#1] ({58*\xskala},{7.30*\yskala}) circle[radius=\r];
\fill[color=#1] ({59*\xskala},{7.01*\yskala}) circle[radius=\r];
\fill[color=#1] ({60*\xskala},{6.71*\yskala}) circle[radius=\r];
\fill[color=#1] ({61*\xskala},{6.40*\yskala}) circle[radius=\r];
\fill[color=#1] ({62*\xskala},{6.08*\yskala}) circle[radius=\r];
\fill[color=#1] ({63*\xskala},{5.76*\yskala}) circle[radius=\r];
\fill[color=#1] ({64*\xskala},{5.43*\yskala}) circle[radius=\r];
\fill[color=#1] ({65*\xskala},{5.10*\yskala}) circle[radius=\r];
\fill[color=#1] ({66*\xskala},{4.75*\yskala}) circle[radius=\r];
\fill[color=#1] ({67*\xskala},{4.41*\yskala}) circle[radius=\r];
\fill[color=#1] ({68*\xskala},{4.05*\yskala}) circle[radius=\r];
\fill[color=#1] ({69*\xskala},{3.69*\yskala}) circle[radius=\r];
\fill[color=#1] ({70*\xskala},{3.32*\yskala}) circle[radius=\r];
\fill[color=#1] ({71*\xskala},{2.95*\yskala}) circle[radius=\r];
\fill[color=#1] ({72*\xskala},{2.57*\yskala}) circle[radius=\r];
\fill[color=#1] ({73*\xskala},{2.18*\yskala}) circle[radius=\r];
\fill[color=#1] ({74*\xskala},{1.79*\yskala}) circle[radius=\r];
\fill[color=#1] ({75*\xskala},{1.39*\yskala}) circle[radius=\r];
\fill[color=#1] ({76*\xskala},{0.99*\yskala}) circle[radius=\r];
\fill[color=#1] ({77*\xskala},{0.58*\yskala}) circle[radius=\r];
\fill[color=#1] ({78*\xskala},{0.16*\yskala}) circle[radius=\r];
\fill[color=#1] ({79*\xskala},{-0.26*\yskala}) circle[radius=\r];
\fill[color=#1] ({80*\xskala},{-0.69*\yskala}) circle[radius=\r];
\fill[color=#1] ({81*\xskala},{-1.12*\yskala}) circle[radius=\r];
\fill[color=#1] ({82*\xskala},{-1.55*\yskala}) circle[radius=\r];
\fill[color=#1] ({83*\xskala},{-2.00*\yskala}) circle[radius=\r];
\fill[color=#1] ({84*\xskala},{-2.44*\yskala}) circle[radius=\r];
\fill[color=#1] ({85*\xskala},{-2.89*\yskala}) circle[radius=\r];
\fill[color=#1] ({86*\xskala},{-3.35*\yskala}) circle[radius=\r];
\fill[color=#1] ({87*\xskala},{-3.81*\yskala}) circle[radius=\r];
\fill[color=#1] ({88*\xskala},{-4.28*\yskala}) circle[radius=\r];
\fill[color=#1] ({89*\xskala},{-4.75*\yskala}) circle[radius=\r];
\fill[color=#1] ({90*\xskala},{-5.23*\yskala}) circle[radius=\r];
\fill[color=#1] ({91*\xskala},{-5.71*\yskala}) circle[radius=\r];
\fill[color=#1] ({92*\xskala},{-6.20*\yskala}) circle[radius=\r];
\fill[color=#1] ({93*\xskala},{-6.69*\yskala}) circle[radius=\r];
\fill[color=#1] ({94*\xskala},{-7.19*\yskala}) circle[radius=\r];
\fill[color=#1] ({95*\xskala},{-7.69*\yskala}) circle[radius=\r];
\fill[color=#1] ({96*\xskala},{-8.19*\yskala}) circle[radius=\r];
\fill[color=#1] ({97*\xskala},{-8.70*\yskala}) circle[radius=\r];
\fill[color=#1] ({98*\xskala},{-9.22*\yskala}) circle[radius=\r];
\fill[color=#1] ({99*\xskala},{-9.73*\yskala}) circle[radius=\r];
\fill[color=#1] ({100*\xskala},{-10.26*\yskala}) circle[radius=\r];
\fill[color=#1] ({101*\xskala},{-10.79*\yskala}) circle[radius=\r];
\fill[color=#1] ({102*\xskala},{-11.32*\yskala}) circle[radius=\r];
\fill[color=#1] ({103*\xskala},{-11.85*\yskala}) circle[radius=\r];
\fill[color=#1] ({104*\xskala},{-12.39*\yskala}) circle[radius=\r];
\fill[color=#1] ({105*\xskala},{-12.94*\yskala}) circle[radius=\r];
\fill[color=#1] ({106*\xskala},{-13.48*\yskala}) circle[radius=\r];
\fill[color=#1] ({107*\xskala},{-14.04*\yskala}) circle[radius=\r];
\fill[color=#1] ({108*\xskala},{-14.59*\yskala}) circle[radius=\r];
\fill[color=#1] ({109*\xskala},{-15.15*\yskala}) circle[radius=\r];
\fill[color=#1] ({110*\xskala},{-15.72*\yskala}) circle[radius=\r];
\fill[color=#1] ({111*\xskala},{-16.29*\yskala}) circle[radius=\r];
\fill[color=#1] ({112*\xskala},{-16.86*\yskala}) circle[radius=\r];
\fill[color=#1] ({113*\xskala},{-17.43*\yskala}) circle[radius=\r];
\fill[color=#1] ({114*\xskala},{-18.01*\yskala}) circle[radius=\r];
\fill[color=#1] ({115*\xskala},{-18.60*\yskala}) circle[radius=\r];
\fill[color=#1] ({116*\xskala},{-19.18*\yskala}) circle[radius=\r];
\fill[color=#1] ({117*\xskala},{-19.78*\yskala}) circle[radius=\r];
\fill[color=#1] ({118*\xskala},{-20.37*\yskala}) circle[radius=\r];
\fill[color=#1] ({119*\xskala},{-20.97*\yskala}) circle[radius=\r];
\fill[color=#1] ({120*\xskala},{-21.57*\yskala}) circle[radius=\r];
}
\def\pfadd#1{
\fill[color=#1] (0,0) circle[radius=\r];
\fill[color=#1] ({1*\xskala},{1.48*\yskala}) circle[radius=\r];
\fill[color=#1] ({2*\xskala},{2.65*\yskala}) circle[radius=\r];
\fill[color=#1] ({3*\xskala},{3.65*\yskala}) circle[radius=\r];
\fill[color=#1] ({4*\xskala},{4.53*\yskala}) circle[radius=\r];
\fill[color=#1] ({5*\xskala},{5.31*\yskala}) circle[radius=\r];
\fill[color=#1] ({6*\xskala},{6.01*\yskala}) circle[radius=\r];
\fill[color=#1] ({7*\xskala},{6.64*\yskala}) circle[radius=\r];
\fill[color=#1] ({8*\xskala},{7.21*\yskala}) circle[radius=\r];
\fill[color=#1] ({9*\xskala},{7.73*\yskala}) circle[radius=\r];
\fill[color=#1] ({10*\xskala},{8.21*\yskala}) circle[radius=\r];
\fill[color=#1] ({11*\xskala},{8.65*\yskala}) circle[radius=\r];
\fill[color=#1] ({12*\xskala},{9.05*\yskala}) circle[radius=\r];
\fill[color=#1] ({13*\xskala},{9.41*\yskala}) circle[radius=\r];
\fill[color=#1] ({14*\xskala},{9.74*\yskala}) circle[radius=\r];
\fill[color=#1] ({15*\xskala},{10.04*\yskala}) circle[radius=\r];
\fill[color=#1] ({16*\xskala},{10.31*\yskala}) circle[radius=\r];
\fill[color=#1] ({17*\xskala},{10.56*\yskala}) circle[radius=\r];
\fill[color=#1] ({18*\xskala},{10.78*\yskala}) circle[radius=\r];
\fill[color=#1] ({19*\xskala},{10.98*\yskala}) circle[radius=\r];
\fill[color=#1] ({20*\xskala},{11.16*\yskala}) circle[radius=\r];
\fill[color=#1] ({21*\xskala},{11.31*\yskala}) circle[radius=\r];
\fill[color=#1] ({22*\xskala},{11.45*\yskala}) circle[radius=\r];
\fill[color=#1] ({23*\xskala},{11.56*\yskala}) circle[radius=\r];
\fill[color=#1] ({24*\xskala},{11.66*\yskala}) circle[radius=\r];
\fill[color=#1] ({25*\xskala},{11.74*\yskala}) circle[radius=\r];
\fill[color=#1] ({26*\xskala},{11.80*\yskala}) circle[radius=\r];
\fill[color=#1] ({27*\xskala},{11.85*\yskala}) circle[radius=\r];
\fill[color=#1] ({28*\xskala},{11.88*\yskala}) circle[radius=\r];
\fill[color=#1] ({29*\xskala},{11.89*\yskala}) circle[radius=\r];
\fill[color=#1] ({30*\xskala},{11.89*\yskala}) circle[radius=\r];
\fill[color=#1] ({31*\xskala},{11.88*\yskala}) circle[radius=\r];
\fill[color=#1] ({32*\xskala},{11.85*\yskala}) circle[radius=\r];
\fill[color=#1] ({33*\xskala},{11.81*\yskala}) circle[radius=\r];
\fill[color=#1] ({34*\xskala},{11.75*\yskala}) circle[radius=\r];
\fill[color=#1] ({35*\xskala},{11.69*\yskala}) circle[radius=\r];
\fill[color=#1] ({36*\xskala},{11.61*\yskala}) circle[radius=\r];
\fill[color=#1] ({37*\xskala},{11.51*\yskala}) circle[radius=\r];
\fill[color=#1] ({38*\xskala},{11.41*\yskala}) circle[radius=\r];
\fill[color=#1] ({39*\xskala},{11.30*\yskala}) circle[radius=\r];
\fill[color=#1] ({40*\xskala},{11.17*\yskala}) circle[radius=\r];
\fill[color=#1] ({41*\xskala},{11.04*\yskala}) circle[radius=\r];
\fill[color=#1] ({42*\xskala},{10.89*\yskala}) circle[radius=\r];
\fill[color=#1] ({43*\xskala},{10.74*\yskala}) circle[radius=\r];
\fill[color=#1] ({44*\xskala},{10.57*\yskala}) circle[radius=\r];
\fill[color=#1] ({45*\xskala},{10.39*\yskala}) circle[radius=\r];
\fill[color=#1] ({46*\xskala},{10.21*\yskala}) circle[radius=\r];
\fill[color=#1] ({47*\xskala},{10.01*\yskala}) circle[radius=\r];
\fill[color=#1] ({48*\xskala},{9.81*\yskala}) circle[radius=\r];
\fill[color=#1] ({49*\xskala},{9.59*\yskala}) circle[radius=\r];
\fill[color=#1] ({50*\xskala},{9.37*\yskala}) circle[radius=\r];
\fill[color=#1] ({51*\xskala},{9.14*\yskala}) circle[radius=\r];
\fill[color=#1] ({52*\xskala},{8.90*\yskala}) circle[radius=\r];
\fill[color=#1] ({53*\xskala},{8.66*\yskala}) circle[radius=\r];
\fill[color=#1] ({54*\xskala},{8.40*\yskala}) circle[radius=\r];
\fill[color=#1] ({55*\xskala},{8.14*\yskala}) circle[radius=\r];
\fill[color=#1] ({56*\xskala},{7.87*\yskala}) circle[radius=\r];
\fill[color=#1] ({57*\xskala},{7.59*\yskala}) circle[radius=\r];
\fill[color=#1] ({58*\xskala},{7.30*\yskala}) circle[radius=\r];
\fill[color=#1] ({59*\xskala},{7.01*\yskala}) circle[radius=\r];
\fill[color=#1] ({60*\xskala},{6.71*\yskala}) circle[radius=\r];
\fill[color=#1] ({61*\xskala},{6.40*\yskala}) circle[radius=\r];
\fill[color=#1] ({62*\xskala},{6.08*\yskala}) circle[radius=\r];
\fill[color=#1] ({63*\xskala},{5.76*\yskala}) circle[radius=\r];
\fill[color=#1] ({64*\xskala},{5.43*\yskala}) circle[radius=\r];
\fill[color=#1] ({65*\xskala},{5.10*\yskala}) circle[radius=\r];
\fill[color=#1] ({66*\xskala},{4.75*\yskala}) circle[radius=\r];
\fill[color=#1] ({67*\xskala},{4.41*\yskala}) circle[radius=\r];
\fill[color=#1] ({68*\xskala},{4.05*\yskala}) circle[radius=\r];
\fill[color=#1] ({69*\xskala},{3.69*\yskala}) circle[radius=\r];
\fill[color=#1] ({70*\xskala},{3.32*\yskala}) circle[radius=\r];
\fill[color=#1] ({71*\xskala},{2.95*\yskala}) circle[radius=\r];
\fill[color=#1] ({72*\xskala},{2.57*\yskala}) circle[radius=\r];
\fill[color=#1] ({73*\xskala},{2.18*\yskala}) circle[radius=\r];
\fill[color=#1] ({74*\xskala},{1.79*\yskala}) circle[radius=\r];
\fill[color=#1] ({75*\xskala},{1.39*\yskala}) circle[radius=\r];
\fill[color=#1] ({76*\xskala},{0.99*\yskala}) circle[radius=\r];
\fill[color=#1] ({77*\xskala},{0.58*\yskala}) circle[radius=\r];
\fill[color=#1] ({78*\xskala},{0.16*\yskala}) circle[radius=\r];
\fill[color=#1] ({79*\xskala},{-0.26*\yskala}) circle[radius=\r];
\fill[color=#1] ({80*\xskala},{-0.69*\yskala}) circle[radius=\r];
\fill[color=#1] ({81*\xskala},{-1.12*\yskala}) circle[radius=\r];
\fill[color=#1] ({82*\xskala},{-1.55*\yskala}) circle[radius=\r];
\fill[color=#1] ({83*\xskala},{-2.00*\yskala}) circle[radius=\r];
\fill[color=#1] ({84*\xskala},{-2.44*\yskala}) circle[radius=\r];
\fill[color=#1] ({85*\xskala},{-2.89*\yskala}) circle[radius=\r];
\fill[color=#1] ({86*\xskala},{-3.35*\yskala}) circle[radius=\r];
\fill[color=#1] ({87*\xskala},{-3.81*\yskala}) circle[radius=\r];
\fill[color=#1] ({88*\xskala},{-4.28*\yskala}) circle[radius=\r];
\fill[color=#1] ({89*\xskala},{-4.75*\yskala}) circle[radius=\r];
\fill[color=#1] ({90*\xskala},{-5.23*\yskala}) circle[radius=\r];
\fill[color=#1] ({91*\xskala},{-5.71*\yskala}) circle[radius=\r];
\fill[color=#1] ({92*\xskala},{-6.20*\yskala}) circle[radius=\r];
\fill[color=#1] ({93*\xskala},{-6.69*\yskala}) circle[radius=\r];
\fill[color=#1] ({94*\xskala},{-7.19*\yskala}) circle[radius=\r];
\fill[color=#1] ({95*\xskala},{-7.69*\yskala}) circle[radius=\r];
\fill[color=#1] ({96*\xskala},{-8.19*\yskala}) circle[radius=\r];
\fill[color=#1] ({97*\xskala},{-8.70*\yskala}) circle[radius=\r];
\fill[color=#1] ({98*\xskala},{-9.22*\yskala}) circle[radius=\r];
\fill[color=#1] ({99*\xskala},{-9.73*\yskala}) circle[radius=\r];
\fill[color=#1] ({100*\xskala},{-10.26*\yskala}) circle[radius=\r];
\fill[color=#1] ({101*\xskala},{-10.79*\yskala}) circle[radius=\r];
\fill[color=#1] ({102*\xskala},{-11.32*\yskala}) circle[radius=\r];
\fill[color=#1] ({103*\xskala},{-11.85*\yskala}) circle[radius=\r];
\fill[color=#1] ({104*\xskala},{-12.39*\yskala}) circle[radius=\r];
\fill[color=#1] ({105*\xskala},{-12.94*\yskala}) circle[radius=\r];
\fill[color=#1] ({106*\xskala},{-13.48*\yskala}) circle[radius=\r];
\fill[color=#1] ({107*\xskala},{-14.04*\yskala}) circle[radius=\r];
\fill[color=#1] ({108*\xskala},{-14.59*\yskala}) circle[radius=\r];
\fill[color=#1] ({109*\xskala},{-15.15*\yskala}) circle[radius=\r];
\fill[color=#1] ({110*\xskala},{-15.72*\yskala}) circle[radius=\r];
\fill[color=#1] ({111*\xskala},{-16.29*\yskala}) circle[radius=\r];
\fill[color=#1] ({112*\xskala},{-16.86*\yskala}) circle[radius=\r];
\fill[color=#1] ({113*\xskala},{-17.43*\yskala}) circle[radius=\r];
\fill[color=#1] ({114*\xskala},{-18.01*\yskala}) circle[radius=\r];
\fill[color=#1] ({115*\xskala},{-18.60*\yskala}) circle[radius=\r];
\fill[color=#1] ({116*\xskala},{-19.18*\yskala}) circle[radius=\r];
\fill[color=#1] ({117*\xskala},{-19.78*\yskala}) circle[radius=\r];
\fill[color=#1] ({118*\xskala},{-20.37*\yskala}) circle[radius=\r];
\fill[color=#1] ({119*\xskala},{-20.97*\yskala}) circle[radius=\r];
\fill[color=#1] ({120*\xskala},{-21.57*\yskala}) circle[radius=\r];
}
\def\pfade#1{
\fill[color=#1] (0,0) circle[radius=\r];
\fill[color=#1] ({1*\xskala},{1.48*\yskala}) circle[radius=\r];
\fill[color=#1] ({2*\xskala},{2.65*\yskala}) circle[radius=\r];
\fill[color=#1] ({3*\xskala},{3.65*\yskala}) circle[radius=\r];
\fill[color=#1] ({4*\xskala},{4.53*\yskala}) circle[radius=\r];
\fill[color=#1] ({5*\xskala},{5.31*\yskala}) circle[radius=\r];
\fill[color=#1] ({6*\xskala},{6.01*\yskala}) circle[radius=\r];
\fill[color=#1] ({7*\xskala},{6.64*\yskala}) circle[radius=\r];
\fill[color=#1] ({8*\xskala},{7.21*\yskala}) circle[radius=\r];
\fill[color=#1] ({9*\xskala},{7.73*\yskala}) circle[radius=\r];
\fill[color=#1] ({10*\xskala},{8.21*\yskala}) circle[radius=\r];
\fill[color=#1] ({11*\xskala},{8.65*\yskala}) circle[radius=\r];
\fill[color=#1] ({12*\xskala},{9.05*\yskala}) circle[radius=\r];
\fill[color=#1] ({13*\xskala},{9.41*\yskala}) circle[radius=\r];
\fill[color=#1] ({14*\xskala},{9.74*\yskala}) circle[radius=\r];
\fill[color=#1] ({15*\xskala},{10.04*\yskala}) circle[radius=\r];
\fill[color=#1] ({16*\xskala},{10.31*\yskala}) circle[radius=\r];
\fill[color=#1] ({17*\xskala},{10.56*\yskala}) circle[radius=\r];
\fill[color=#1] ({18*\xskala},{10.78*\yskala}) circle[radius=\r];
\fill[color=#1] ({19*\xskala},{10.98*\yskala}) circle[radius=\r];
\fill[color=#1] ({20*\xskala},{11.16*\yskala}) circle[radius=\r];
\fill[color=#1] ({21*\xskala},{11.31*\yskala}) circle[radius=\r];
\fill[color=#1] ({22*\xskala},{11.45*\yskala}) circle[radius=\r];
\fill[color=#1] ({23*\xskala},{11.56*\yskala}) circle[radius=\r];
\fill[color=#1] ({24*\xskala},{11.66*\yskala}) circle[radius=\r];
\fill[color=#1] ({25*\xskala},{11.74*\yskala}) circle[radius=\r];
\fill[color=#1] ({26*\xskala},{11.80*\yskala}) circle[radius=\r];
\fill[color=#1] ({27*\xskala},{11.85*\yskala}) circle[radius=\r];
\fill[color=#1] ({28*\xskala},{11.88*\yskala}) circle[radius=\r];
\fill[color=#1] ({29*\xskala},{11.89*\yskala}) circle[radius=\r];
\fill[color=#1] ({30*\xskala},{11.89*\yskala}) circle[radius=\r];
\fill[color=#1] ({31*\xskala},{11.88*\yskala}) circle[radius=\r];
\fill[color=#1] ({32*\xskala},{11.85*\yskala}) circle[radius=\r];
\fill[color=#1] ({33*\xskala},{11.81*\yskala}) circle[radius=\r];
\fill[color=#1] ({34*\xskala},{11.75*\yskala}) circle[radius=\r];
\fill[color=#1] ({35*\xskala},{11.69*\yskala}) circle[radius=\r];
\fill[color=#1] ({36*\xskala},{11.61*\yskala}) circle[radius=\r];
\fill[color=#1] ({37*\xskala},{11.51*\yskala}) circle[radius=\r];
\fill[color=#1] ({38*\xskala},{11.41*\yskala}) circle[radius=\r];
\fill[color=#1] ({39*\xskala},{11.30*\yskala}) circle[radius=\r];
\fill[color=#1] ({40*\xskala},{11.17*\yskala}) circle[radius=\r];
\fill[color=#1] ({41*\xskala},{11.04*\yskala}) circle[radius=\r];
\fill[color=#1] ({42*\xskala},{10.89*\yskala}) circle[radius=\r];
\fill[color=#1] ({43*\xskala},{10.74*\yskala}) circle[radius=\r];
\fill[color=#1] ({44*\xskala},{10.57*\yskala}) circle[radius=\r];
\fill[color=#1] ({45*\xskala},{10.39*\yskala}) circle[radius=\r];
\fill[color=#1] ({46*\xskala},{10.21*\yskala}) circle[radius=\r];
\fill[color=#1] ({47*\xskala},{10.01*\yskala}) circle[radius=\r];
\fill[color=#1] ({48*\xskala},{9.81*\yskala}) circle[radius=\r];
\fill[color=#1] ({49*\xskala},{9.59*\yskala}) circle[radius=\r];
\fill[color=#1] ({50*\xskala},{9.37*\yskala}) circle[radius=\r];
\fill[color=#1] ({51*\xskala},{9.14*\yskala}) circle[radius=\r];
\fill[color=#1] ({52*\xskala},{8.90*\yskala}) circle[radius=\r];
\fill[color=#1] ({53*\xskala},{8.66*\yskala}) circle[radius=\r];
\fill[color=#1] ({54*\xskala},{8.40*\yskala}) circle[radius=\r];
\fill[color=#1] ({55*\xskala},{8.14*\yskala}) circle[radius=\r];
\fill[color=#1] ({56*\xskala},{7.87*\yskala}) circle[radius=\r];
\fill[color=#1] ({57*\xskala},{7.59*\yskala}) circle[radius=\r];
\fill[color=#1] ({58*\xskala},{7.30*\yskala}) circle[radius=\r];
\fill[color=#1] ({59*\xskala},{7.01*\yskala}) circle[radius=\r];
\fill[color=#1] ({60*\xskala},{6.71*\yskala}) circle[radius=\r];
\fill[color=#1] ({61*\xskala},{6.40*\yskala}) circle[radius=\r];
\fill[color=#1] ({62*\xskala},{6.08*\yskala}) circle[radius=\r];
\fill[color=#1] ({63*\xskala},{5.76*\yskala}) circle[radius=\r];
\fill[color=#1] ({64*\xskala},{5.43*\yskala}) circle[radius=\r];
\fill[color=#1] ({65*\xskala},{5.10*\yskala}) circle[radius=\r];
\fill[color=#1] ({66*\xskala},{4.75*\yskala}) circle[radius=\r];
\fill[color=#1] ({67*\xskala},{4.41*\yskala}) circle[radius=\r];
\fill[color=#1] ({68*\xskala},{4.05*\yskala}) circle[radius=\r];
\fill[color=#1] ({69*\xskala},{3.69*\yskala}) circle[radius=\r];
\fill[color=#1] ({70*\xskala},{3.32*\yskala}) circle[radius=\r];
\fill[color=#1] ({71*\xskala},{2.95*\yskala}) circle[radius=\r];
\fill[color=#1] ({72*\xskala},{2.57*\yskala}) circle[radius=\r];
\fill[color=#1] ({73*\xskala},{2.18*\yskala}) circle[radius=\r];
\fill[color=#1] ({74*\xskala},{1.79*\yskala}) circle[radius=\r];
\fill[color=#1] ({75*\xskala},{1.39*\yskala}) circle[radius=\r];
\fill[color=#1] ({76*\xskala},{0.99*\yskala}) circle[radius=\r];
\fill[color=#1] ({77*\xskala},{0.58*\yskala}) circle[radius=\r];
\fill[color=#1] ({78*\xskala},{0.16*\yskala}) circle[radius=\r];
\fill[color=#1] ({79*\xskala},{-0.26*\yskala}) circle[radius=\r];
\fill[color=#1] ({80*\xskala},{-0.69*\yskala}) circle[radius=\r];
\fill[color=#1] ({81*\xskala},{-1.12*\yskala}) circle[radius=\r];
\fill[color=#1] ({82*\xskala},{-1.55*\yskala}) circle[radius=\r];
\fill[color=#1] ({83*\xskala},{-2.00*\yskala}) circle[radius=\r];
\fill[color=#1] ({84*\xskala},{-2.44*\yskala}) circle[radius=\r];
\fill[color=#1] ({85*\xskala},{-2.89*\yskala}) circle[radius=\r];
\fill[color=#1] ({86*\xskala},{-3.35*\yskala}) circle[radius=\r];
\fill[color=#1] ({87*\xskala},{-3.81*\yskala}) circle[radius=\r];
\fill[color=#1] ({88*\xskala},{-4.28*\yskala}) circle[radius=\r];
\fill[color=#1] ({89*\xskala},{-4.75*\yskala}) circle[radius=\r];
\fill[color=#1] ({90*\xskala},{-5.23*\yskala}) circle[radius=\r];
\fill[color=#1] ({91*\xskala},{-5.71*\yskala}) circle[radius=\r];
\fill[color=#1] ({92*\xskala},{-6.20*\yskala}) circle[radius=\r];
\fill[color=#1] ({93*\xskala},{-6.69*\yskala}) circle[radius=\r];
\fill[color=#1] ({94*\xskala},{-7.19*\yskala}) circle[radius=\r];
\fill[color=#1] ({95*\xskala},{-7.69*\yskala}) circle[radius=\r];
\fill[color=#1] ({96*\xskala},{-8.19*\yskala}) circle[radius=\r];
\fill[color=#1] ({97*\xskala},{-8.70*\yskala}) circle[radius=\r];
\fill[color=#1] ({98*\xskala},{-9.22*\yskala}) circle[radius=\r];
\fill[color=#1] ({99*\xskala},{-9.73*\yskala}) circle[radius=\r];
\fill[color=#1] ({100*\xskala},{-10.26*\yskala}) circle[radius=\r];
\fill[color=#1] ({101*\xskala},{-10.79*\yskala}) circle[radius=\r];
\fill[color=#1] ({102*\xskala},{-11.32*\yskala}) circle[radius=\r];
\fill[color=#1] ({103*\xskala},{-11.85*\yskala}) circle[radius=\r];
\fill[color=#1] ({104*\xskala},{-12.39*\yskala}) circle[radius=\r];
\fill[color=#1] ({105*\xskala},{-12.94*\yskala}) circle[radius=\r];
\fill[color=#1] ({106*\xskala},{-13.48*\yskala}) circle[radius=\r];
\fill[color=#1] ({107*\xskala},{-14.04*\yskala}) circle[radius=\r];
\fill[color=#1] ({108*\xskala},{-14.59*\yskala}) circle[radius=\r];
\fill[color=#1] ({109*\xskala},{-15.15*\yskala}) circle[radius=\r];
\fill[color=#1] ({110*\xskala},{-15.72*\yskala}) circle[radius=\r];
\fill[color=#1] ({111*\xskala},{-16.29*\yskala}) circle[radius=\r];
\fill[color=#1] ({112*\xskala},{-16.86*\yskala}) circle[radius=\r];
\fill[color=#1] ({113*\xskala},{-17.43*\yskala}) circle[radius=\r];
\fill[color=#1] ({114*\xskala},{-18.01*\yskala}) circle[radius=\r];
\fill[color=#1] ({115*\xskala},{-18.60*\yskala}) circle[radius=\r];
\fill[color=#1] ({116*\xskala},{-19.18*\yskala}) circle[radius=\r];
\fill[color=#1] ({117*\xskala},{-19.78*\yskala}) circle[radius=\r];
\fill[color=#1] ({118*\xskala},{-20.37*\yskala}) circle[radius=\r];
\fill[color=#1] ({119*\xskala},{-20.97*\yskala}) circle[radius=\r];
\fill[color=#1] ({120*\xskala},{-21.57*\yskala}) circle[radius=\r];
}
\def\pluslinie#1{
\draw[color=#1] (0,{-13.03*\yskala})--({120*\xskala},{-13.03*\yskala});
\node[color=#1] at ({0.5*120*\xskala},{-13.03*\yskala}) [below] {$e^{-30}$};
}
\def\minuslinie#1{
\draw[color=#1] (0,{13*\yskala})--({120*\xskala},{13*\yskala});
\node[color=#1] at ({0.5*120*\xskala},{13*\yskala}) [above] {$e^{30}$};
}
\def\maxlinie#1{
\draw[color=#1,line width=0.5] (0,{11.8900*\yskala})--({120*\xskala},{11.8900*\yskala});
}
\def\grenzef#1#2{
\draw[color=#1] (0,{4.6653*\yskala})--({120*\xskala},{4.6653*\yskala});
\node[color=#1] at ({0.3*120*\xskala},{4.6653*\yskala}) [above] {#2};
}
\def\grenzed#1#2{
\draw[color=#1] (0,{-3.7636*\yskala})--({120*\xskala},{-3.7636*\yskala});
\node[color=#1] at ({0.3*120*\xskala},{-3.7636*\yskala}) [below] {#2};
}
\def\grenzee#1#2{
\draw[color=#1] (0,{-7.0749*\yskala})--({120*\xskala},{-7.0749*\yskala});
\node[color=#1] at ({0.3*120*\xskala},{-7.0749*\yskala}) [below] {#2};
}

\begin{tikzpicture}[>=latex,thick]

\def\xskala{0.1}
\def\yskala{0.2}
\def\r{0.04}

\pluslinie{red}
\minuslinie{red}

\definecolor{darkgreen}{rgb}{0,0.6,0}
\grenzef{darkgreen}{Minimum für \texttt{float}}
\grenzed{darkgreen}{Minimum für \texttt{double}}
\grenzee{darkgreen}{Minimum für \texttt{long double}}

\draw[->] (-1*\xskala,0)--({125*\xskala},0) coordinate[label={$k$}];
\draw[->] ({-0*\xskala},{-22*\yskala})--({-0*\xskala},{16*\yskala})
	coordinate[label={left:$\log_{10}$}];

\foreach \k in {0,10,20,...,120}{
	\draw ({\k*\xskala},-0.1)--({\k*\xskala},0.1);
}
\foreach \k in {20,40,...,120}{
	\node at ({\k*\xskala},-0.1) [below] {$\k$};
}

\draw (-0.1,{10*\yskala})--(0.1,{10*\yskala});
\node at (-0.1,{10*\yskala}) [left] {$10$};
\draw (-0.1,{-10*\yskala})--(0.1,{-10*\yskala});
\node at (-0.1,{-10*\yskala}) [left] {$-10$};
\node at (-0.1,0) [left] {$0$};
\draw (-0.1,{-20*\yskala})--(0.1,{-20*\yskala});
\node at (-0.1,{-20*\yskala}) [left] {$-20$};

\pfade{blue}
\maxlinie{blue}

\end{tikzpicture}
\end{document}

