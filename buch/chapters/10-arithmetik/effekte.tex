%
% effekte.tex
%
% (c) 2020 Prof Dr Andreas Müller, Hochschule Rapperswil
%
\section{Numerische Effekte
\label{buch:section:numerische-effekte}}
\rhead{Numerische Effekte}
Die Unzulänglichkeiten der in Computern verwendeten Zahlensysteme haben 
zwei Effekte zur Folge, denen bei der Konzeption eines numerischen
Lösungsverfahrens Rechnung getragen werden muss.

\subsection{Auslöschnung}
{\em Auslöschung} tritt auf, wenn die Differenz zweier ähnlich grosser
Zahlen gebildet wird.
\index{Auslöschung}
Als Beispiel betrachten wir die beiden Zahlen $a=\pi$ und $b=\sqrt{10}$.
Berechnen wir deren Differenz in Octave, erhalten wir:
\verbatiminput{chapters/10-arithmetik/pi10.txt}
Die ersten zwei Stellen von $a$ und $b$ stimmen überein.
Octave zeigt sowohl von $a$ als auch von $b$ 15 signifikante Stellen an.
Ein Vergleich mit einer Berechnung mit noch mehr Stellen zeigt, dass diese
15 Stellen auch zuverlässig sind.
Für die Differenz zeigt Octave ebenfalls 15 Stellen an, doch die
letzte Stelle is falsch, wie zum Beispiel die Berechnung mit 20 Stellen
Genauigkeit zeigt\footnote{Diese Berechnung wurde mit dem
Linux-Kommandozeilenprogramm \texttt{bc} durchgeführt, welches mit
einstellbarer Festkommapräzision von tausenden von Stellen rechnen kann.
Es ist Teil jeder Linux-Distribution.}
\verbatiminput{chapters/10-arithmetik/pi20.txt}
Schreiben wir die Subtraktion in der tabellarischen Form
\begin{center}
\begin{tabular}{>{\tt}r}
 3.16227766016838\\
-3.14159265358979\\
\hline
 0.02068500657859\\
\hline
\end{tabular}
\end{center}
wird erkennbar, dass nur 13 Stellen der Differenz tatsächlich bekannt sind.

Die Rechnung wird in Binärdarstellung etwas klarer.
In der folgenden Tabelle sind die Werte in der mittleren Spalte in
binärer Gleitkommadarstellung gezeigt, Vorzeichen, Exponent und Mantisse
sind zur Verdeutlichung durch ein Leerzeichen getrennt.
Zu Beginn der Mantisse muss man sich eine implizite \texttt{1}
denken, die nicht gespeichert wird.
In der dritten Spalte werden die gleichen Zahlen als binäre Festkommawerten
geschrieben.
Zur Berechnung der Differenz muss der Prozessor die Mantissen ja
zunächst so schieben, dass sie den gleichen Exponenten bekommen,
der Prozessor berechnet also die Differenz implizit in einer
Festkommadarstellung.
\begin{center}
\renewcommand\arraystretch{1.2}
\begin{tabular}{|>{$}c<{$}|>{\tt}r|>{\tt}r|}
\hline 
              & \textrm{Gleikommawert}            &\textrm{Festkommawert}   \\
\hline
\sqrt{10}     & 0 10000000 10010100110001011000010&11.0010100110001011000010\\
\pi           & 0 10000000 10010010000111111011011&11.0010010000111111011011\\
\hline
\sqrt{10}-\pi & 0 01111001 01010010111001110000000& 0.0000010101001011100111\\
\hline
\end{tabular}
\end{center}
Man kann gut erkennen, dass die Differenz nur 17 signifkante Stellen hat.
Bei der nachfolgenden Darstellung als Gleitkommazahl werden 7 Nullen
hinzugefügt, die aber nichts mit der tatsächlichen Differenz 
$\sqrt{10}-\pi$ zu tun haben.
Aus zwei Zahlenwerten mit einer Genauigkeit von 24 Binärstellen ist ein
Wert mit einer Genauigkeit von nur 17 Binärstellen geworden.
Es sind 7 Binärstellen Genauigkeit ausgelöscht worden.

\begin{beispiel}
\label{buch:beispiel:erfc}
Sei $X$ ein standardnormalverteilte Zufallsvariable, es soll die 
Wahrscheinlichkeit dafür berechnet werden, dass $a\le X\le b$ ist.
In der Wahrscheinlichkeitsrechnung lernt man, dass man dazu die
Verteilungsfunktion
\[
\Phi(x)
=
\frac1{\sqrt{2\pi}} \int_{-\infty}^x e^{-s^2/2} \,ds
=
\frac12 + \frac{1}{\sqrt{2\pi}} \int_0^x e^{-s^2/2} \,ds
\]
der Standardnormalverteilung verwenden
kann:
\[
P(a\le X \le b)
= 
\Phi(b) - \Phi(a).
\]
Die Funktion $\Phi(x)$ wird in vielen Bibliotheken nicht direkt zur
Verfügung gestellt, oft ist nur die sogenannte Fehlerfunktion
\[
\operatorname{erf}(x)
=
\frac{2}{\sqrt{\pi}}
\int_0^x e^{-t^2}\,dt
\]
verfügbar.
Die Variablentransformation $t=s/\sqrt{2}$ oder $s=\sqrt{2}t$ macht aus dem
Integral für
$\Phi(x)$ den Ausdruck
\begin{align*}
\Phi(x)
&=
\frac12 + \frac{1}{\sqrt{2\pi}} \int_0^x e^{-s^2/2} \,ds
=
\frac12 + \frac{1}{\sqrt{2\pi}} \int_0^{\sqrt{2}x} e^{-t^2} \,\sqrt{2}\, dt
=
\frac12 + \frac{2}{\pi} \int_0^{\sqrt{2}x} e^{-t^2}\,dt
\\
&= \frac12 + \operatorname{erf}(\sqrt{2}x),
\end{align*}
die Fehlerfunktion kann also zur Berechnung der gesuchten Wahrscheinlichkeit
verwendet werden.

In der C-Bibliothek stehen Funktionen zur Berechnung von $\sqrt{2}$ und
$\operatorname{erf}(x)$ für alle zur Verfügung stehenden Datentypen zur
Verfügung.
Die Tabelle~\ref{buch:table:erfcancellation}
zeigt die Resultate\footnote{Diese Resultate wurden
mit dem Programm \texttt{normal.cpp} Im Verzeichnis
\texttt{buch/chapters/experimente/ausloeschung} von \cite{buch:repo}
berechnet.}
für $a=4.18$ und $b=a+1$.
\begin{table}
\centering
\renewcommand\arraystretch{1.2}
\begin{tabular}{|l|>{$}r<{$}|>{$}r<{$}|}
\hline
Datentyp       & \textrm{Rechnung mit $\operatorname{erf}(x)$}&
\textrm{Rechnung mit $\operatorname{erfc}(x)$}\\
\hline
\texttt{float} & 0.000000\phantom{\mathstrut\cdot10^{-00}}&
6.271826\cdot10^{-17}\\
\texttt{double}& 1.110223\mathstrut\cdot10^{-16}&
6.271826\cdot10^{-17}\\
\texttt{long}  & 6.272110\mathstrut\cdot10^{-17}&
6.271826\cdot10^{-17}\\
\hline
\end{tabular}
\caption{Berechung der Wahrscheinlichkeit $P(4.18\le X\le 5.18)$
einer standardnormalverteilten Zufallsvariable mit Hilfe der
Bibliotheksfunktionen $\operatorname{erf}(x)$ und $\operatorname{erfc}(x)$.
Starke Auslöschung macht die Berechnung mit $\operatorname{erf}(x)$
unbrauchbar.
\label{buch:table:erfcancellation}}
\end{table}

Da die Werte von $\operatorname{erf}(\sqrt{2}b)$ und
$\operatorname{erf}(\sqrt{2}b)$ fast gleich gross sind, findet 
starke Auslöschung statt.
Beim Datentyp \texttt{float} ist überhaupt kein Unterschied mehr
feststellbar.

Um dieses Problem in den Griff zu bekommen, stellt die C-Bibliothek
zusätzlich die sogenannte komplementäre Fehlerfunktion
\[
\operatorname{erfc}(x) = 1-\operatorname{erf}(x)
\qquad\Rightarrow\qquad
\operatorname{erf}(x) = 1-\operatorname{erfc}(x)
\]
zur Verfügung.
Damit kann die Wahrscheinlichkeit natürlich auch berechnet werden:
\[
P(a\le X \le b)
=
\operatorname{erf}(\sqrt{2}b)
-
\operatorname{erf}(\sqrt{2}a)
=
(1-\operatorname{erfc}(\sqrt{2}b))
-
(1-\operatorname{erfc}(\sqrt{2}a))
=
\operatorname{erfc}(\sqrt{2}a)
-
\operatorname{erfc}(\sqrt{2}b).
\]
Für grosse Werte von $x$ streben die Werte dieser Funktion gegen $0$,
es kann also nicht mehr passieren, dass man einen kleinen Wert zu finden
versucht, indem man zwei vergleichsweise grosse Zahlen voneinander subtrahiert.
In der dritten Spalte von Tabelle~\ref{buch:table:erfcancellation}
sind die Resultate der Berechnung mit Hilfe von $\operatorname{erfc}(x)$
gezeigt.
Der Auschlöschungseffekt ist vollständig verschwunden.
Man kann sogar ablesen, dass die Verwendung des Datentyps \texttt{long double}
dem Problem der Auslöschung ebenfalls nicht begegnen konnte.
Der mit $\operatorname{erf}(x)$ berechnete Wert hat selbst bei Verwendung
dieses längsten verfügbaren Typs nur drei korrekte Dezimalstellen.
\end{beispiel}

\subsection{Verschmierung}
Auslöschung kann nicht nur auftreten, wenn zwei fast gleich grosse
Zahlen subtrahiert werden.
Sie kann in etwas weniger offensichtlicher Form stattfinden, wenn
bei der Summation einer Reihe im Vergleich zum Resultat grosse
Zwischenresultate entstehen.
Diesen Verlust an Genauigkeit infolge grosser Zwischenresultate
wird {\em Verschmierung} genannt.
\index{Verschmierung}

Die Taylorreihe
\[
e^x = 1 + x + \frac{x^2}{2!}
+\frac{x^3}{3!}
+\frac{x^4}{4!}
+\frac{x^5}{5!}
+\frac{x^6}{6!}
+\dots
=\sum_{k=0}^\infty \frac{x^k}{k!}
\]
der Exponentialfunktion ist sehr gut geeignet, Werte von $e^x$ für
positive $x$ zu berechnen.
Da der Nenner $k!$ exponentiell schnell anwächst, werden späte
Terme in der Reihe sehr schnell vernachlässigbar klein.

\begin{figure}
\centering
\includegraphics{chapters/10-arithmetik/figures/verschmierung.pdf}
\caption{Verschmierung bei der Berechnung von $e^{-30}$ und $e^{30}$
mit der Exponentialreihe.
Auf der vertikalen Achse ist der Zehnerlogarithmus der verschiedenen
Grössen abgetragen.
Die beiden Werte $e^{30}$ und $e^{-30}$ sind als rote Linien
am oberen und unteren Rand eingetragen.
{\color{blue}Blau} ist der absolute Betrag des Terms $s_k=x^k/k!$ in der
Exponentialreihe.
Die {\color{darkgreen}grünen} Linien zeigen den kleinsten Unterschied,
der zwischen zwei Termen möglich ist, die die Grösse des grössten Summanden
in der Exponentialreihe haben.
\label{buch:figure:expversch}}
\end{figure}
\begin{table}
\centering
\begin{tabular}{|l|>{$}r<{$}|}
\hline
Datentyp            & e^{-30} \\
\hline
\texttt{float}      & -7.2959523438\cdot 10^{\phantom{-}4\phantom{0}}\\
\texttt{double}     &  6.1030424789\cdot 10^{-6\phantom{0}}\\
\texttt{long double}& -1.2489259417\cdot 10^{-8\phantom{0}}\\
\hline
exakt               &  9.3576229688\cdot 10^{-14}\\
\hline
\end{tabular}
\end{table}

Für negative Exponenten alternieren die Terme, werden zwischenzeitlich
sehr gross, das Resultate ist aber ein sehr kleiner Wert.
Im Laufe der Rechnung müssen sich also grosse Terme wieder wegheben.
Dies ist in Abbildung~\ref{buch:figure:expversch} illustriert.
Dort sind die Werte $e^{30}$ und $e^{-30}$ auf einer logarithmischen
Skala vertikal als {\color{red}rote} Linien eingezeichnet.
Die einzelnen Summanden der Reihe sind in {\color{blue}blau} dargestellt.
Man kann sehen, dass 

\bgroup
\definecolor{darkgreen}{rgb}{0,0.6,0}
Die {\color{darkgreen}grünen} Linien zeigen, welche Genauigkeit mit verschiedenen Datentypen
überhaupt noch möglich ist.
Der \texttt{float}-Typ hat eine Mantisse von 24 bit, eine Zahl $m$ ist
daher nur unterscheidbar von $m(1+\epsilon)$, wenn $\epsilon > 2^{-24}$.
Dies entspricht $24*\log_10(2) = 7.22$ Dezimalstellen.
Die {\color{darkgreen}grüne} Linie für den \texttt{float}-Typ ist daher
$7.22$ unter dem Maximum der Terme Exponentialreihe eingetragen.

Beim \texttt{double}-Typen ist die Mantiesse 52 bit lang, beim
\texttt{long double} sind es 63 bit.
Doch selbst beim \texttt{long double} ist die Verschmierung vollständig,
das Resultat für $e^{-30}$ hat nichts mit der Realität zu tun.
Im Gegenteil, sie zeigen eher an, wie gross der verwendete Datentyp ist.
Die {\color{darkgreen}grünen} Linien in Abbildung~\ref{buch:figure:expversch}
befinden sich ungefähr dort, wo die gefundenen Werte
eingetragen werden müssten.
\egroup

