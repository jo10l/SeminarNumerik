%
% vorwort.tex -- Vorwort zum Buch zum Seminar
%
% (c) 2019 Prof Dr Andreas Mueller, Hochschule Rapperswil
%
\chapter*{Vorwort}
\lhead{Vorwort}
\rhead{}
Dieses Buch entstand im Rahmen des Mathematischen Seminars
im Frühjahrssemester 2020 an der Hochschule für Technik Rapperswil.
Die Teilnehmer, Studierende der Abteilungen für Elektrotechnik,
Informatik, Bauingenieurwesen und Erneuerbare Energie und Umwelttechnik
der HSR, erarbeiteten nach einer Einführung in das Themengebiet jeweils
einzelne Aspekte des Gebietes in Form einer Seminararbeit, über
deren Resultate sie auch in einem Vortrag informierten. 

Im Frühjahr 2020 war das Thema des Seminars die Numerik, also die
\index{Numerik}%
Berechnung von mathematischen Resultaten mit Hilfe von Computern.

In einigen Arbeiten wurde auch Code zur Demonstration der 
besprochenen Methoden und Resultate geschrieben, soweit
möglich und sinnvoll wurde dieser Code im Github-Repository
\index{Github-Repository}
dieses Kurses%
\footnote{\url{https://github.com/AndreasFMueller/SeminarNumerik.git}}
\cite{buch:repo}
abgelegt.
Im genannten Repository findet sich auch der Source-Code dieses
Skriptes, es wird hier unter einer Creative Commons Lizenz
zur Verfügung gestellt.

Das Umschlagbild zeigt die Resultate einer numerischen Simulation
der Überschallströmung durch die Düse eines Raketenmotors mit Hilfe der
Software OpenFOAM.
\index{OpenFOAM}%
\index{Raketenmotor}%
Rechnergestützte Fluiddynamik bringt viele der in diesem Buch
\index{Fluiddynamik}%
besprochenen Techniken zusammen.
Die Lösungsverfahren für die partiellen Differentialgleichungen der
\index{partielle Differentialgleichung}
\index{Differentialgleichung!partielle}
Strömungsdynamik führen zum Beispiel auf sehr grosse lineare Gleichungssysteme.
\index{Strömungsdynamik}%
\index{Gleichungssystem, lineares}%
\index{lineares Gleichungssystem}%
Sie führen aber oft auch numerische Instabilitäten ein, was nicht
\index{Instabilität}
\index{numerische Instabilität}
weiter verwunderlich ist, da auch die Strömungen selbst oft instabil sind.
Es illustriert damit die besondere Bedeutung, die eine sorgfältige
numerische Analyse auf der Basis der in diesem Buch vorgestellten
Techniken hat.


