Berechnen Sie eine Approximation des Integrals der Funktion
$f(x)$ über das Interval $[a,b]$ mit folgender Idee.
Bestimmen Sie erst ein Interpolationspolynom zweiten Grades, welches
mit $f$ in den Punkte $a$, $b$ und dem Mittelpunkt $m$ des Intervalls
$[a,b]$ übereinstimmt.
Dann ist
\[
I=\int_a^b p(x)\,dx
\simeq
\int_a^b f(x)\,dx
\]
eine Approximation.
Schreiben Sie $I$ als Ausdruck in $f(a)$, $f(b)$ und $f(m)$.
Diese Formel ist bekannt als die {\em Simpsonsche Regel} 
\index{Simpsonsche Regel}%
oder auch die
{\em Keplersche Fassregel}.
\index{Keplersche Fassregel}%
\index{Fassregel, Kepler}%


\begin{loesung}
Zur Bestimmung des Interpolationspolynoms braucht man das
$l(x) = (x-a)(x-m)(x-b)$ und die Polynome vom Grad 2
\begin{equation}
\begin{aligned}
l_0(x) &= \frac1{(a-m)(a-b)} (x-m)(x-b),
\\
l_1(x) &= \frac1{(m-a)(m-b)} (x-a)(x-b),
\\
l_2(x) &= \frac1{(b-a)(b-m)} (x-a)(x-m).
\end{aligned}
\label{4001:lpolys}
\end{equation}
Das Interpolationspolynom ist dann
\[
p(x) = f(a) l_0(x) + f(m) l_1(x) + f(b) l_2(x).
\]
Für das Integral muss man jetzt die Integrale der einzelnen Polynome 
$l_k(x)$ bestimmen.
Dies ist etwas mühsam, aber durch Variablentransformation auf
die Werte $a=-1$, $b=1$ und $m=0$ wird es einfacher:
\begin{align*}
\int_{-1}^1 \frac{x(x-1)}{(-1)(-1-1)}\,dx
&=
\frac12
\biggl[\frac13x^3 -\frac12x^2\biggr]_{-1}^1
=
\frac13,
\\
\int_{-1}^1 \frac{(x+1)(x-1)}{(0-(-1))(0-1)}\,dx
&=
-\int_{-1}^1 x^2 -1 \,dx
=
-\biggl[\frac13x^3-x\biggr]_{-1}^1
=
-\frac23 + 2
=
\frac43,
\\
\int_{-1}^1 \frac{(x+1)x}{1-(-1))(1-0)}\,dx
&=
\frac13.
\end{align*}
Zusammen mit den Normierungsnennern in \eqref{4001:lpolys} erhalten wir
somit
\begin{align*}
\int_a^b l_0(x)\,dx
&=
\frac{b-a}{2}
\cdot
\frac13
=
\frac{b-a}6
\\
\int_a^b l_1(x)\,dx
&=
\frac{b-a}{2}
\cdot
\frac{4}{3}
=
\frac{4(b-a)}{6}
\\
\int_a^b l_2(x)\,dx
&=
\frac{b-a}{2}
\cdot
\frac13
=
\frac{b-a}{6}.
\end{align*}
Die verschiedenen Faktoren im Nenner können wir alle durch die
Intervalllänge $b-a$ ausdrücken. 
Es ist zum Beispiel $m-a=\frac12(b-a)$ und $m-b=-\frac12(m-a)$.
Damit erhalten wir jetzt für das Integral von $p(x)$
\begin{align*}
\int_a^b p(x)\,dx
&=
\frac{b-a}{6}\bigl(f(a) + 4f(m) + f(b)\bigr).
\qedhere
\end{align*}
\end{loesung}




