%
% chapter.tex
%
% (c) 2020 Prof Dr Andreas Müller
%
\chapter{Gleichungen lösen\label{chapter:gleichungen}}
\lhead{Gleichungen lösen}
Im Januar 1535 stellten sich Niccolò Tartaglia und 
Antonio Maria Fior öffentlich je 30 kubische Gleichungen mit
der Form $x^3 + px = q$ oder $x^3 = px + q$ und forderten sich
gegenseitig heraus, diese Gleichungen innert 50 Tagen zu lösen.
Für moderne Leser scheint es zwischen diesen Gleichungen keinen 
Unterschied zu geben, aber negative Zahlen waren damals noch nicht
in Gebrauch.
Fior war ein Schüler von Scipione dal Ferro, der ein Lösungsmethode
für einige Typen dieser kubischen Gleichungen aufgestellt hatte.
Tartaglia strengte sich darauf hin besonders an und fand 12.~Februar
1535 eine Lösungsformel für beide Typen und am darauffolgenden Tag
auch eine für die Gleichung $x^3 + q = px$.

Der Wettbewerb ging sehr ungleich aus: mit seiner Lösungsformel konnte
Tartaglia alle gestellten Aufgaben lösen, während Fior keine einzige
lösen konnte.
Solche öffentlichen Wettbewerbe unter Gelehrten waren in der
Renaissance durchaus üblich, sie waren Teil des Marketings mit dem
Gelehrte bekannt werden und neue Kunden gewinnen konnten.
Tartaglia zum Beispiel verdiente seinen Lebensunterhalt vorwiegend
als kaufmännischer Rechner und Privatlehrer.
Tartaglia ist auch der Autor eines Buches über Ballistik, seine
mathematischen Forschungen waren durchaus auch von konkreten 
Anwendungen motiviert.

Die Lösung der kubischen Gleichung durch Tartaglia und die spätere
Verallgemeinerung durch Gerolamo Cardano (1501--1576) sowie die 
Lösung der Gleichung vierten Grades durch Lodovico Ferrari (1522--1565)
waren Lösungsformeln wie sie heute jeder Schüler für die quadratische
Gleichung kennelernt.
Wie sieht die Lösungsformel für allgemeine Gleichungen fünften
Grades aus?
Die überraschende Antwort gab 1824 Niels Henrik Abel, er zeigte,
dass es eine allgemeine Lösungsformel nicht geben kann.
Dies war eines der ersten Resultate in einer langen Reihe von
Unmöglichkeitsaussagen.
So wissen wir zum Beispiel heute, dass die Funktion $e^{-x^2}$ 
keine analytische Darstellung mit Hilfe von Potenzfunktionen,
Brüchen, Exponential- und Logarithmus-Funktionen hat.
Es gibt sogar einen Algorithmus\footnote{Eigentlich handelt es
sich um einen Pseudo-Algorithmus, denn einzelne Schritte des Algorithmus
verlangen, dass entschieden werden muss, ob zwei Terme identisch sind.
Auch dies ist ein Problem, welches von einem Computer nicht in voller
Allgemeinheit gelöst werden kann, allerdings aus ganz anderem Grund.}
von Risch, mit dem man entscheiden kan, ob eine solche Darstellung
für einen vorgegebenen Integranden möglich ist.

Diese Beispiel zeigen uns, dass die Lösung einer Gleichung mit
einer Lösungsformel eher die Ausnahme als die Regel darstellt.
Gefragt sind daher numerische Methoden, die Gleichungen effizient
und zuverlässig lösen können.
Dieses Kapitel befasst sich mit den besonderen Schwierigkeiten dieser
Aufgabenstellung.

%
% nullstellen.tex -- Nullstellen der Ableitung zwischen den Nullstellen der fkt
%
% (c) 2020 Prof Dr Andreas Müller, Hochschule Rapperswil
%
\documentclass[tikz]{standalone}
\usepackage{amsmath}
\usepackage{times}
\usepackage{txfonts}
\usepackage{pgfplots}
\usepackage{csvsimple}
\usetikzlibrary{arrows,intersections,math}
\begin{document}
\def\skala{2.4}
\begin{tikzpicture}[>=latex,thick,scale=\skala]

\definecolor{darkgreen}{rgb}{0,0.6,0}

\def\zero#1{
	\fill[color=red] (#1,0) circle[radius={0.08/\skala}];
}

\def\one#1{
	\xdef\x{#1}
	\pgfmathparse{0.3*((\x*\x-5)*\x*\x+4)*\x});
	\xdef\y{\pgfmathresult}
	\pgfmathparse{0.3*((5*\x*\x-15)*\x*\x+4)};
	\xdef\m{\pgfmathresult}
%	\draw[line width=0.1pt] ({\x-0.2},{\y})--({\x+0.2},{\y});
%	\draw[line width=0.1pt] ({\x-0.2},{\y-0.2*\m})--({\x+0.2},{\y+0.2*\m});
	\draw[color=blue,line width=0.5pt] ({\x},{\y})--({\x},-3);
	\fill[color=blue] ({\x},{\y}) circle[radius={0.08/\skala}];
	\fill[color=blue] ({\x},{-3}) circle[radius={0.08/\skala}];
}

\def\two#1{
	\xdef\x{#1}
	\pgfmathparse{0.04*((5*\x*\x-15)*\x*\x+4)};
	\xdef\y{\pgfmathresult}
	\pgfmathparse{0.04*((20*\x*\x-30)*\x)};
	\xdef\m{\pgfmathresult}
%	\draw[line width=0.1pt] ({\x-0.2},{\y})--({\x+0.2},{\y});
%	\draw[line width=0.1pt] ({\x-0.2},{\y-0.2*\m})--({\x+0.2},{\y+0.2*\m});
	\draw[color=darkgreen,line width=0.5pt] ({\x},{\y})--({\x},-2);
	\fill[color=darkgreen] ({\x},{\y}) circle[radius={0.08/\skala}];
	\fill[color=darkgreen] ({\x},{-2}) circle[radius={0.08/\skala}];
}

\zero{-2}
\zero{-1}
\zero{0}
\zero{1}
\zero{2}

\draw[color=red,line width=1.5pt]
	plot[domain=-2.1:2.1,samples=100]
		({\x},{0.3*((\x*\x-5)*\x*\x+4)*\x});

\draw[->] (-2.15,0)--(2.25,0) coordinate[label={$x$}];
\draw[->] (-2.1,-1.2)--(-2.1,1.2) coordinate[label={left:$y$}];

\one{-1.643}
\one{-0.549}
\one{0.549}
\one{1.643}

\begin{scope}[yshift=-3cm]

\draw[color=blue,line width=1.5pt]
	plot[domain=-2.1:2.1,samples=100] ({\x},{0.04*((5*\x*\x-15)*\x*\x+4)});
\draw[->] (-2.15,0)--(2.25,0) coordinate[label={$x$}];
\draw[->] (-2.1,-0.5)--(-2.1,1.7) coordinate[label={left:$y'$}];

\two{-1.23}
\two{0}
\two{1.23}

\end{scope}

\begin{scope}[yshift=-5cm]
\draw[color=darkgreen,line width=1.5pt]
	plot[domain=-2.1:2.1,samples=100] ({\x},{0.01*((20*\x*\x-30)*\x)});
\draw[->] (-2.15,0)--(2.25,0) coordinate[label={$x$}];
\draw[->] (-2.1,-1.4)--(-2.1,1.4) coordinate[label={left:$y''$}];
\end{scope}

\end{tikzpicture}
\end{document}


%
% newton.tex
%
% (c) 2020 Prof Dr Andreas Müller, Hochschule Rapperswil
%
\section{Newton-Verfahren
\label{buch:section:newtion}}
\rhead{Newton-Verfahren}


%
% homotopie.tex
%
% (c) 2020 Prof Dr Andreas Müller, Hochschule Rapperswil
%
\section{Homotopie-Verfahren
\label{buch:section:homotopie}}
\rhead{Homotopie-Verfahren}
Der Erfolg des Newton-Verfahrens hängt entscheidend von der Qualität der
Anfangsschätzung $x_0$ ab.
Allerdings ist es oft nicht einfach, eine solche Schätzung zu produzieren.
Die folgende Idee kann dabei helfen.

Oft ist ein schwieriges Problem ein ``deformierte'' Variante eines
weniger schwierigen Problems.
Der Begriff der Homotopie 

\begin{definition}
Zwei Funktionen $f_0(x)$ und $f_1(x)$ heissen homotop, wenn es
eine stetige Funktion
\[
F\colon \mathbb R\times I : (x,t)\mapsto F(x,t)
\]
mit $I=[0.1]$
gibt derart, dass $f_0(x)=F(x,0)$ und $f_1(x)=F(x,1)$.
Die partielle Funktion $x\mapsto F(x,t)$ für $t\in I$ wird auch mit
$f_t$ bezeichnet: $f_t(x)=F(x,t)$.
\end{definition}

\begin{beispiel}
Die Kepler-Gleichung ist 
\[
M=E-e\sin E,
\]
wobei $M$ gegeben und $E$ gesucht ist.
Dazu gehört die Funktion
\[
f(E)=M-E+e\sin E.
\]
Der Fall $e=0$ ist ein trivial einfaches Problem, $E=M$ ist Nullstelle
der Funktion
\[
f_0(E)=M-E.
\]
Eine Homotopie zwischen $f_0$ und $f_1=f$ ist
\[
F(x,t) = M-E+et\sin E.
\qedhere
\]
\end{beispiel}

Eine Homotopie kann dazu verwendet werden, Startwerte für das Newtonverfahren
zu liefern.
Ist $x_0(t)$ eine Nullstelle der partiellen Funktion $x\mapsto F(x,t)$,
dann kann $x_0(t)$ als Startwert zur Bestimmung einer Nullstelle von
der partiellen Funktion $F(x,t')$ für $|t-t'|<\varepsilon$ dienen.
Ist $F$ differenzierbar bezüglich $x$, dann können einige Iterationen
des Newton-Verfahrens aus dem Startwert $x_0(t)$ eine gute Lösung für
$x_0(t')$ sein.
Damit lässt sich der folgende Algorithmus konstruieren:

\begin{enumerate}
\item 
Starte mit der exakten Lösung $x_0=x_0(0)$ und $t=0$
\item
Inkrementiere $t$ um $\Delta t$
\item
verbessere $x_0$ durch einige Iterationen des Newton-Verfahrens
zu einer Nullstelle von $f_t(x)$.
\item 
Wiederhole Schritte 2 und 3 bis $t=1$.
\end{enumerate}
Auf diese Weise kann sichergestellt werden, dass jede Iteration
des Newton-Verfahrens mit einem guten Schätzwert startet, wenn auch
nur für eine immer bessere Approximation des eigentlichen Problems.










