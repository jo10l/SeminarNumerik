%
% normalverteilung.tex
%
% (c) 2020 Prof Dr Andreas Müller, Hochschule Rapeprswil
%
\subsection{Inverse der Normalverteilungsfunktion
\label{buch:subsection:inversenormal}}
Das Integral der Standardnormalverteilungsdichte
\[
\Phi(x) = \int_{-\infty}^x e^{-t^2/2}\,dt
\]
kann nicht in geschlossener Form berechnet werden und erst recht
nicht invertiert werden.
Für die Anwendung wird jedoch die Umkehrfunktion benötigt, zu einem Wert
$p\in[0,1]$ ist dasjenige $x$ zu finden, für welches $F(x)=p$ gilt.
Im Beispiel auf Seite~\pageref{buch:beispiel:erfc} wurde gezeigt,
wie die Fehlerfunktion
\[
\operatorname{erf}(x) = \frac{2}{\sqrt{\pi}}\int_0^x e^{-t^2}\,dt
\]
dazu verwendet werden kann
\[
\Phi(x) = \frac12+\operatorname{erf}(\sqrt{2}x)
\]
zu berechnen.
In diesem Abschnitt soll untersucht werden, wie zu gegebenen Funktionswert
das $x$ bestimmt werden kann.
Es soll also die Gleichung
\[
\Phi(x)=p
\qquad\Rightarrow\qquad
f(x)=\frac12+\operatorname{erf}(\sqrt{2}x)-p=0
\]
gelöst werden.

\subsubsection{Sekantenverfahren}
TODO

\subsubsection{Newton-Verfahren}
Das Newton-Verfahren benötigt ausser dem Funktionswert auch noch die 
Ableitung
\[
f'(x)
=
\frac{d}{dx}\frac{2}{\sqrt{\pi}}\int_0^{\sqrt{2}x} e^{-t^2}\,dt
=
\frac{2\sqrt{2}}{\sqrt{\pi}}e^{-2x^2}.
\]
Damit wird die Iterationsformel für das Newton-Verfahren:
\begin{equation}
x_{n+1} = x_n - \frac{\sqrt{\pi}}{2\sqrt{2}}e^{2x_n^2}
\biggl(\frac12+\operatorname{erf}(x_n) -p \biggr).
\end{equation}
Wie erwartet konvergiert die Iterationsfolge quadratisch für geeignete
Startwerte (siehe Tabelle~\ref{buch:table:normalnewton}).
Der Startwert $x_0=0$ funktioniert für jedes beliebige $p$.
Bei weiter von $0$ entfernten Starwerte läuft man Gefahr, dass die Iteration
zu betragsmässig grossen Werten $x$ springt, was dann zu einem Überlauf führt.

\begin{table}
\centering
\begin{tabular}{|>{$}r<{$}|>{$}r<{$}|}
\hline
 k &   x_n                    \\
\hline
 0 &   0.00000000000000000000 \\
 1 &   \underline{0.2}5066282746310003808 \\
 2 &   \underline{0.262}13276541328668328 \\
 3 &   \underline{0.26220025}396582809170 \\
 4 &   \underline{0.2622002563540203}8710 \\
 5 &   \underline{0.262200256354020390}13 \\
 6 &   \underline{0.262200256354020390}05 \\
 7 &   \underline{0.262200256354020390}13 \\
 8 &   \underline{0.262200256354020390}05 \\
 9 &   \underline{0.262200256354020390}13 \\
\hline
\end{tabular}
\caption{Newton-Iteration zur Bestimmung der Inversen der Verteilungsfunktion
der Normalverteilung, berechnet mit dem Typ \texttt{long double}.
Die letzten zwei Stellen können wegen numerischer Unsicherheit nicht
berechnet werden.
\label{buch:table:normalnewton}}
\end{table}





