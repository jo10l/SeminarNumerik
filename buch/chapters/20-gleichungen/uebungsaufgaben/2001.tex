Finden Sie eine Lösung der Gleichung
\begin{equation}
x^5 - \frac54x^4 +\frac14 = 0
\label{2001:polynom}
\end{equation}
mit Hilfe des Newton-Verfahrens.
\index{Newton-Verfahren}%
Welche Konvergenzgeschwindigkeit stellen Sie fest?
\index{Konvergenzgeschwindigkeit}%

\begin{loesung}
\begin{figure}
\centering
\def\skala{4}
\begin{tikzpicture}[>=latex,thick,scale=\skala]
\begin{scope}
\clip (-1,-1) rectangle (1.4,1);
\draw[color=red,line width=1.4pt]
	plot[domain=-1.0:1.5,samples=100] ({\x},{(\x-5/4)*\x*\x*\x*\x+1/4});
\end{scope}

% Achsen
\draw[->] (0,-1.05)--(0,1.1) coordinate[label={$y$}];
\draw[->] (-1.05,0)--(1.5,0) coordinate[label={$x$}];

% Ticks y-Achse
\draw ({-0.1/\skala},-1)--({0.1/\skala},-1);
\draw ({-0.1/\skala},1)--({0.1/\skala},1);
\node at ({-0.1/\skala},1) [left] {$1$};
\node at ({-0.1/\skala},-1) [left] {$-1$};

% Ticks x-Achse
\draw (-1,{-0.1/\skala})--(-1,{0.1/\skala});
\draw (1,{-0.1/\skala})--(1,{0.1/\skala});
\node at (-1,{-0.1/\skala}) [below] {$-1$};
\node at (1,{-0.1/\skala}) [below] {$1$};

\end{tikzpicture}
\caption{Graph zur Aufgabe~\ref{2001}, Nullstelle des
Polynoms~\eqref{2001:polynom}.
\label{2001:figure}}
\end{figure}
Die Iterationsformel für das Newton-Verfahren braucht die Ableitung von
$f(x)=x^5-\frac54x^4+\frac14$, also
\[
f'(x)= 5x^4-5x^3 = 5(x-1)x^3,
\]
insbesondere hat $f'(x)$ eine Nullstelle bei $x=1$, was in der Newton-Iteration
zu Schwierigkeiten führen könnte.
\index{Nullstelle}%
\index{Ableitung}%
Ebenso ist $0$ eine Nullstelle der Ableitung, so dass $0$ und $1$ ganz
bestimmt keine guten Startwerte für die Newton-Iteration sind.

Die Iteration lautet jetzt
\[
x_{n+1} = x_n - \frac{f(x_n)}{f'(x_n)}
=
x_n -\frac{x_n^5-\frac54x_n^4+\frac14}{5(x_n-1)x_n^3}.
\]

Um eine Lösung zu finden, braucht man jetzt noch einen guten Startwert.
Der Graph in Abbilddung~\ref{2001:figure} zeigt, dass $1$ eine doppelte
Nullstelle ist, und dass es noch eine weitere Nullstelle in der Nähe von
$x_0=-0.5$ gibt.

Die Iteration liefert die folgenden Werte:
\begin{center}
\begin{tabular}{|>{$}r<{$}|>{$}r<{$}|}
\hline
n& x_n\\
\hline
0& -0.500000000000000\\
1& -0.\underline{6}50000000000000\\
2& -0.\underline{6}10646335912608\\
3& -0.\underline{6058}93367985182\\
4& -0.\underline{6058295}97525286\\
5& -0.\underline{605829586188268}\\
6& -0.\underline{605829586188268}\\
\hline
\end{tabular}
\end{center}
Man hat also in 5 Iterationsschritten ein Resultate mit 15 Stellen
Genaugikeit erhalten, quadratische Konvergenz ist klar sichtbar.

Die Nullstelle bei $x=1$ macht dem Newton-Algorithmus dagegen etwas Mühe.
Die Iteration mit Startwert $x_0=0.5$ liefert
\begin{center}
\begin{tabular}{|>{$}r<{$}|>{$}r<{$}|}
\hline
 n& x_n\\
\hline
 0 &  0.50000000000000\\
 1 &  \underline{1}.15000000000000\\
 2 &  \underline{1.0}8416125585600\\
 3 &  \underline{1.0}4522243974522\\
 4 &  \underline{1.0}2356853334768\\
 5 &  \underline{1.0}1205249984377\\
 6 &  \underline{1.00}609759001936\\
 7 &  \underline{1.00}306721671745\\
 8 &  \underline{1.00}153829071860\\
 9 &  \underline{1.000}77032580427\\
10 &  \underline{1.000}38545926066\\
11 &  \underline{1.000}19280387677\\
12 &  \underline{1.0000}9642051980\\
13 &  \underline{1.0000}4821490799\\
14 &  \underline{1.0000}2410861546\\
15 &  \underline{1.0000}1205459851\\
16 &  \underline{1.00000}602736919\\
17 &  \underline{1.00000}301369634\\
18 &  \underline{1.00000}150685215\\
19 &  \underline{1.000000}75342128\\
20 &  \underline{1.000000}37677627\\
\hline
\end{tabular}
\end{center}
Die Konvergenz ist wie erwartet nur linear.

Die Ableitung $f'(x)=4x^3(x-1)$ der Funktion $f(x)$ hat an der Stelle 
$x=0$ eine dreifache Nullstelle, was auch daran erkennbar ist, dass 
$f(x)$ abgesehen von der Konstante von vierter Ordnung in $x$ ist.
Der Graph von $f(x)$ ist daher in einer Umgebung von $0$ sehr flach.
Startwerte $x_0$ in der Umgebung von $0$ führen daher automatisch
zu einem Wert $x_1$ sehr weit weg vom Nullpunkt.
Zum Beispiel führt $x_0=-10^{-4}$ auf $x_1=-49995000499.9501$.
Es braucht dann 114 Iterationen, bis $x_n$ nahe genug bei der negativen
Nullstelle liegt, dass man wieder quadratische Konvergenz beobachten kann.
\end{loesung}


