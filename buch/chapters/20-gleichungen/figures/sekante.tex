%
% sekante.tex -- template for standalon tikz images
%
% (c) 2020 Prof Dr Andreas Müller, Hochschule Rapperswil
%
\documentclass[tikz]{standalone}
\usepackage{amsmath}
\usepackage{times}
\usepackage{txfonts}
\usepackage{pgfplots}
\usepackage{csvsimple}
\usetikzlibrary{arrows,intersections,math}
\begin{document}
\def\skala{1}
\begin{tikzpicture}[>=latex,thick,scale=\skala]

\def\a{1}
\def\b{8}
\def\fa{-2}
\def\fb{3}

\fill[color=gray!20] (\a,0)--(\a,\fa)--(\b,\fb)--(\b,0)--cycle;

\draw[color=red,line width=1.4pt] plot[domain=0:9,samples=100]
	({\x},{((-0.0087464*\x+0.0466472)*\x+0.9329446)*\x-2.9708455});

\draw[->] (-0.1,0)--(9.3,0) coordinate[label={$x$}];
\draw[->] (0,-3.1)--(0,3.3) coordinate[label={right:$y$}];

\draw[color=blue] (1,0)--(1,-2);
\draw[color=blue] (8,0)--(8,3);
\fill[color=blue] (1,-2) circle[radius=0.08];
\fill[color=blue] (8,3) circle[radius=0.08];
\draw[color=blue] (1,-2)--(8,3);

\pgfmathparse{(\a*\fb-\b*\fa)/(\fb-\fa)}
\xdef\xneu{\pgfmathresult}

\node at (\xneu,-0.1) [below] {$x_0\mathstrut$};
\fill[color=blue] (\xneu,0) circle[radius=0.08];
\node[color=blue] at (\a,{0.5*\fa}) [left] {$f(a)$};
\node[color=blue] at (\b,{0.5*\fb}) [right] {$f(b)$};

\draw (\a,-0.1)--(\a,0.1);
\node at (\a,0.1) [above] {$a$};
\draw (\b,-0.1)--(\b,0.1);
\node at (\b,-0.1) [below] {$b$};

\node at ({0.5*(\a+\xneu)},0.1) [above] {$x_0-a\mathstrut$};
\node at ({0.5*(\b+\xneu)},-0.1) [below] {$b-x_0\mathstrut$};

\end{tikzpicture}
\end{document}

