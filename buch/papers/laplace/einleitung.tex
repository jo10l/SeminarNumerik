%
% einleitung.tex -- Beispiel-File für die Einleitung
%
% (c) 2020 Severin Weiss
%
Die Laplacetransformation einer auf $[0, \infty)$ definierten und in jedem endlichen Intrevall  $[0, a)$ absolut integrierbaren Funktion ist die Funktion
\[
F(s) = \int_0^\infty e^{-st}f(t)\,dt,~~Re(s)>\gamma_{0}.
\]
Das Argument s kann eine komplexe Zahl sein, im Folgenden wird vorausgesetzt, dass das Integral für $Re(s)>\gamma_{0}$ konvergieren muss.
Das inverse Laplaceproblem ist, die Funktionen $f(t)$ im Zeitbereich aus jenen bekannten Funktionen $F(s)$ im Frequenzbereich zu rekonstruieren.

Mit der Riemanninversionsformel ergibt sich $f(t)$ aus $F(s)$
\[
f(t) = \frac{1}{2\pi i} \oint_{B} e^{st}F(s)\,ds,~~t>0,~~i=\sqrt{-1}
\]
wobei $B$ die Bromwich Kontur von $\gamma-i\infty$ bis $\gamma+i\infty$, mit $\gamma>\gamma_{0}$, parallel zur imaginären Achse ist. Die Bromwich Kontur ist eine geschlossen Kurve. Die Kurve ist in untestehender Abbildung ersichtlich. Sie besteht aus zwei Teilen. C1 ist ein Teilkreis mit Radius R vom Mittelpunkt. C2 ist die vertikale Linie AB, welche sich zur rechten aller Singularitäten von $F(s)$ befindet.  Bei Vergrösserung des Radius R wird die Bromwich Kontur alle Singularitäten umschliessen. Unter diesen Bedingung konvergiert das Integral über C1 gegen Null, wenn der Radius R gegen unendlich strebt. Desweiteren wird in diesem Fall das Integral über C2 gleich der Summe der Residuen, der in der Bromwich Kontur enthaltenen Singularitäten.

\begin{figure}
\centering
\includegraphics[width=6.9cm]{papers/laplace/Bromwich_Contour}
\caption{Fehler von $Bromwich Kontur$
\label{laplace:bromwichkontur}
}
\end{figure}

Im Allgemeinen möchte man die Kontur zur linken Seite platzieren, sodass der Betrag des Faktors $e^{st}$ im Integrand kleiner wird.
Die Kontur sollte jedoch nicht zu nahe an Singularitäten von $F(s)$ angenähert werden, dies würde Spitzen des Integranden zur Folge haben.

