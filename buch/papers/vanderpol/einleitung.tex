%
% einleitung.tex -- Beispiel-File für die Einleitung
%
% (c) 2020 Prof Dr Andreas Müller, Hochschule Rapperswil
%
\section{Einleitung\label{vanderpol:section:einleitung}}
\rhead{Einleitung}
In diesem Kapitel werden wir uns mit der Empfindlichkeit einiger numerischer Methoden, der Runge-Kutta-Familie, in Bezug auf die Länge des Integrationsschritts befassen.
Chaotische Systeme reagieren sehr empfindlich auf die Wahl der Ausgangsbedingungen, wenn wir beispielsweise eine sehr kleine Änderung daran vornehmen, erhalten wir eine völlig andere Reaktion.
Die Ausbreitung des numerischen Fehlers kann den gleichen Effekt haben und für verschiedene Werte völlig unterschiedliche Lösungen ergeben.
In unserem Fall ist das untersuchte chaotische System der Van-der-Pol Oszillators.
Die Differentialgleichung, die ihn beschreibt, wurde in 1927 von Balthasar van der Pol während seiner Studien über Oszillatoren mit Vakuumröhren formuliert.
Sie wird wie folgt beschrieben:
\begin{equation}
\frac{d^{2}x}{dt^{2}} - \mu (1 - x^{2}) \frac{dx}{dt} + x = 0
\label{vanderpol:equations:vdp}
\end{equation}
Die in diesem Kapitel verwendeten numerischen Methoden sind der Runge-Kutta-Algorithmus vierter Ordnung (Abschnitt \ref{subsection:buch:ode:runge-kutta}) und der Euler-Algorithmus, der als Runge-Kutta-Algorithmus erster Ordnung betrachtet werden kann (Satz \ref{buch:ode:einschritt}).

