%
% loesung.tex -- Beispiel-File für die Beschreibung der Loesung
%
% (c) 2020 Prof Dr Andreas Müller, Hochschule Rapperswil
%
\section{Lösung
\label{quadratur:section:loesung}}
\textcolor{red}{
    TODO
    \begin{itemize}
        \item Gauss Legendre Formel zeigen
        \item Beispiel berechnen mit Formel
        \item Berechnung der Gewichtung
        \item Berechnung der Position der Stützstellen
        \item Erklären der verschiedenen Formen der Gauss-Quadratur und ihrer Anwendungen
        \item 
    \end{itemize}
}


\subsection{De finibus bonorum et malorum
\label{quadratur:subsection:bonorum}}
\textcolor{red}{TODO: Lösung in python}


\subsection{Formen der Gauss-Quadratur
\label{quadratur:subsection:gaussformen}}
Es gibt verschiedene Ausprägungen der Gauss-Integration, abhängig vom jeweiligen Anwendungsbereich 
erkennbar an den Grenzen des Integrals und der Wahl der Gewichtung.
Die vier häufigsten Formen sind in der Tabelle~\ref{buch:table:gaussformen} abgebildet.

\begin{table}[h!]
    \begin{tabular}{|>{$}c<{$}|>{$}c<{$}|>{$}c<{$}|>{$}c<{$}|}
        \hline
        \text{Name} &  \text{Untere Grenze} & \text{Obere Grenze} & \text{Formel} \\
        \hline  
        \text{Legendre} & -1 & 1 & p_{n}(x) = \int_{-1}^{1} f(x)\,dx \approx \sum_{i=0}^{n} A_{i} f(x_{i}) \\
        \text{Chebyshev} &  -1 & 1 & T_{n}(x) = \int_{-1}^{1} (1-x^{2})^{-1/2} f(x)\,dy \approx \frac{\pi}{n+1} \sum_{i=0}^{n} f(x_{i}) \\
        \text{Laguerre} &  0 & \infty & L_{n}(x) = \int_{0}^{\infty} e^{-x} f(x)\,dx \approx \sum_{i=0}^{n} A_{i} f(x_{i}) \\
        \text{Hermite} & -\infty & \infty & H_{n}(x) = \int_{-\infty}^{\infty} f(x)\,dx \approx \sum_{i=0}^{n} A_{i} f(x_{i})\\
        \hline
    \end{tabular}
    \caption{Formen der Gauss-Quadratur
    \label{buch:table:gaussformen}}
    
\end{table}


