%
% main.tex -- Paper zum Thema arctan
%
% (c) 2020 Hochschule Rapperswil
%
\chapter{Interpolation und numerische Ableitung\label{chapter:interdiff}}
\lhead{Interpolation und numerische Ableitung}
\rhead{}
\begin{refsection}
\chapterauthor{Andreas Müller}

{\parindent0pt
In} Kapitel~\ref{chapter:interpolation} wurde gezeigt, wie man durch
Interpolation gute Approximationen einer Funktion in einem Interval
finden kann.
Für äquidistante Stützstellen ist das Interpolationspolynom vor allem in
der Mitte des Intervalls sehr genau.
Man kann daher versuchen, diese Approximation durch das Interpolationspolynom
auch für eine Approximation der Ableitung der Funktion zu verwenden, indem
man das Interpolationspolynom ableitet.
Am Ende dieses Prozesses sollte sich eine Formel ergeben, welche die
Ableitung näherungsweise aus Funktionswerten in der Nähe des
interessierenden Punktes bestimmt, ganz ähnlich wie das aus der
Taylor-Reihe abgeleitet, in Kapitel~\ref{chapter:ableitung} beschriebene
Verfahren.
Dieses Kapitel führt die Berechnung der Gewichte der Funktionswerte durch
und untersucht an Hand von numerischen Beispielen, was man von dem Verfahren
erwarten kann.

\section{Das Problem der numerischen Ableitung}
\rhead{Das Problem der numerischen Ableitung}
Wir gehen von einer in einem Intervall $I=[a,b]$ definierten differenzierbaren
Funktion $f\colon[a,b]\to\mathbb R$ aus. 
Zu bestimmen ist die Ableitung von $f$ in einem Punkt $x\in[a,b]$.
Der Differenzenquotient
\[
f'(x)
\approx
\frac{f(x+h)-f(x)}{h}
\]
ist die naheliegende Approximation, doch für kleines $h$ werden $f(x+h)$ und
$f(x)$ fast gleich gross, die Differenz ist daher stark verschmiert.
Grosses $h$ wiederum beschränkt die Genauigkeit, denn die Taylor-Entwicklung
liefert für den Differenzenquotienten
\[
f'(x)
\approx
\frac{f(x) + hf'(x) + h^2f''(x)/2 + O(h^3) - f(x)}{h}
=
f'(x) + h\frac{f''(x)}2 + O(h^2),
\]
für nicht allzu kleines $h$ wird der Einfluss des $f''(x)$-Terms
merklich.
Es muss also angestrebt werden, genaue Resultate auch bei relativ grossem
$h$ bekommen zu können.

Eine erste Verbesserung kann mit dem symmetrischen Differenzenquotienten
\[
f'(x)
\approx
\frac{f(x+\frac{h}2) -f(x-\frac{h}2)}{h}
\]
erreicht werden.
Setzt man wieder die Taylor-Reihe ein, erhält man
\begin{align*}
f'(x) &\approx
\frac{1}{h} \biggl(
f(x) + \frac{h}2 f'(x) + \frac{h^2}{8} f''(x) + \frac{h^3}{48}f'''(x) + O(h^4)
\\
&\qquad\qquad
-
\biggl(
f(x) - \frac{h}2 f'(x) + \frac{h^2}{8} f''(x) - \frac{h^3}{48}f'''(x) + O(h^4)
\biggr)
\biggr)
\\
&=
f'(x) + \frac{h^2}{24}f'''(x) + O(h^3).
\end{align*}
Danke der Symmetrie ist der $f''(x)$-Term verschwunden, der Fehler ist nur
noch von Ordnung $h^2$.

Den symmetrischen Differenzenquotienten kann man auch als Wert eines
Interpolationspolynoms ansehen.
Die beiden Stützstellen $x_{0,1}=x\pm \frac{h}2$ definieren die lineare
Interpolationsfunktion
\begin{align*}
p(x)
&=
f(x_0) l_0(x) + f(x_1) l_1(x)
=
f(x_0) \frac{x-x_1}{x_0-x_1}
+
f(x_1) \frac{x-x_0}{x_1-x_0}
=
-f(x_0) \frac{x-x_1}{h}
+
f(x_1) \frac{x-x_0}{h}
\\
&=
\frac{f(x_1) - f(x_0)}{h}
x
+
\text{const}.
\intertext{%
Die symmetrische Differenz ist daher auch die Ableitung}
p'(x)
&=
\frac{f(x_1)-f(x_0)}{h}
\end{align*}
des Interpolationspolynoms.
Es liegt daher nahe, die Ableitung $f'(x)$ als Ableitung eines
Interpolationspolynoms höherer Ordnung von $f(x)$ zu berechnen.


\section{Die Ableitung des Interpolationspolynoms}
\rhead{Die Ableitung des Interpolationspolynoms}
In Kapitel~\ref{chapter:interpolation} wurde für das Interpolationspolynom
der Ausdruck
\[
p(x)
=
\sum_{j=0}^n f(x_j) l_j(x)
\qquad\text{mit}\quad
l_j(x)
=
\frac{(x-x_0)(x-x_1)\cdots\widehat{(x-x_j)}\cdot (x-x_n)}{(x_j-x_0)(x_j-x_1)\cdots\widehat{(x_j-x_j)}\cdot (x_j-x_n)}
\]
gefunden.
Die Ableitung seine Ableitung ist
\[
p'(x)
=
\sum_{j=0}^n f(x_j) l'_j(x).
\]
Die Ableitungen der speziellen Interpolationspolynome $l_j(x)$ kann
man ebenfalls direkt berechnen:
\begin{equation}
l_j'(x)
=
\frac{1}{(x_j-x_0)(x_j-x_1)\cdots\widehat{(x_j-x_j)}\cdot (x_j-x_n)}
\sum_{k\ne j} (x-x_0)\cdots \widehat{(x-x_k)}\cdots\widehat{(x-x_j)}\cdots (x-x_n).
\label{interdiff:koeffizienten}
\end{equation}
Durch Einsetzen der Stütztstellen lassen sich die $l_j(x)$ aus
Formel~\eqref{interdiff:koeffizienten} direkt berechnen, so dass
man die Näherungsformel 
\begin{equation}
f'(x) = \sum_{j=0}^n f(x_j) l'_j(x)
\label{interdiff:ableitung}
\end{equation}
erhält.

Für äquidistante Stützstellen mit Abstand $h$, alle Differenzen im
Nenner von $l_j(x)$ sind Vielfache von $h$.
Bei $n$ Stützstellen geben die Faktoren vor dem Faktor $(x_j-x_j)$, der
weggelassen muss, einen Beitrag $jh\cdot (j-1)h \cdot h$ im Nenner,
die Faktoren danach liefern den Beitrag $h\cdot 2h\cdot \dots\cdot (n-j)h$.
Der Nenner ist daher
\[
(-1)^j
j!\cdot (n-j)! h^n,
\]
die Ableitung kann jetzt als
\begin{equation}
l_j'(x)
=
\frac{(-1)^j}{j!\cdot (n-j)! h^n}
\sum_{k\ne j} (x-x_0)\cdots \widehat{(x-x_k)}\cdots\widehat{(x-x_j)}\cdots (x-x_n).
\label{interdiff:koeffizienten2}
\end{equation}
geschrieben werden.


\section{Ableitungsverfahren}
\rhead{Ableitungsverfahren}
Wir wenden die Formeln
\eqref{interdiff:ableitung}
und
\eqref{interdiff:koeffizienten}
zunächst auf den Fall zweier Stützstellen $x_0=\xi-\frac{h}2$ und
$x_1=\xi+\frac{h}2$ an.
Es folgt
\begin{align*}
l_0(x)
&=
-\frac{1}{h} (x-x_1)
&
l_0'(x)
&=
-\frac1h \cdot 1
\\
l_1(x)
&=
\frac1{h} (x-x_0)
&
l_1'(x)
&=
\frac1h \cdot 1
\\
f'(x)
&\approx
\frac1h(- f(x_0)+f(x_1)) = \frac{f(x+\frac{h}2)-f(x-\frac{h}2)}{2},
\end{align*}
der bereits bekannte symmetrische Differenzenquotient.

Für $n=2$ und die Stützstellen $x_0=x-h$, $x_1=x$ und $x_2=x+h$ erhalt man
dagegen
\begin{align*}
l_0(x)
&=
\frac{1}{h\cdot 2h} (x-x_1)(x-x_2)
&
l_0'(x)
&=
\frac{1}{2h^2}(2x-x_2-x_1)
&
l_0'(\xi)
&=
\frac{1}{2h^2}(2\xi -(\xi+h) -\xi)
=
-\frac{1}{2h}
\\
l_1(x)
&=
\frac{1}{h^2} (x-x_0)(x-x_2)
&
l'_1(x)
&=
\frac{1}{h^2} (2x-x_0-x_2)
&
l'_1(\xi)
&=
\frac{1}{h^2}(2\xi-(\xi-h)-(\xi+h)) = 0
\\
l_2(x)
&=
\frac{1}{2h\cdot h} (x-x_0)(x-x_1)
&
l'_2(x)
&=
\frac{1}{2h^2} (2x-x_0-x_1)
&
l'_2(\xi)
&=
\frac{1}{2h^2} (2\xi-(\xi-h)-\xi)
=
\frac{1}{2h}
\\
f'(\xi)
&\approx
\frac{1}{2h}(f(x_2) - f(x_0))
\rlap{$\displaystyle= \frac{f(\xi+h)-f(\xi-h)}{2h},$}
\end{align*}
also wieder eine symmetrische Differenz.
Die zusätzliche Stützstelle in der Mitte bringt keinen Genauigkeitsgewinn.




\section{Numerische Experimente}
\rhead{Numerische Experimente}

\cite{interdiff:numerical-analysis}

\printbibliography[heading=subbibliography]
\end{refsection}
