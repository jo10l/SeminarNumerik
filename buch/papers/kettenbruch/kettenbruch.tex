%
% kettenbruch.tex -- Paper zum Thema <kettenbruch>
%
% (c) 2020 Hochschule Rapperswil
%
\documentclass{book}
\usepackage{etex}
\usepackage{geometry}
\geometry{papersize={170mm,240mm},total={140mm,200mm},top=21mm,bindingoffset=10mm}
\usepackage[english,ngerman]{babel}
\usepackage[utf8]{inputenc}
\usepackage[T1]{fontenc}
\usepackage{cancel}
\usepackage{times}
\usepackage{amsmath,amscd}
\usepackage{amssymb}
\usepackage{amsfonts}
\usepackage{amsthm}
\usepackage{graphicx}
\usepackage{fancyhdr}
\usepackage{textcomp}
\usepackage{txfonts}
\newcommand\hmmax{0}
\newcommand\bmmax{0}
\usepackage{bm}
\usepackage{epic}
\usepackage{verbatim}
%\usepackage{suffix}
\usepackage{paralist}
\usepackage{makeidx}
\usepackage{array}
\usepackage{hyperref}
\usepackage{subfigure}
\usepackage{tikz}
\usepackage{pgfplots}
\usepackage{pgfplotstable}
\usepackage{pdftexcmds}
\usepackage{pgfmath}
\usepackage[autostyle=false,english=american]{csquotes}
\usepackage{wasysym}
\usepackage{environ}
\usepackage{appendix}
\usepackage[all]{xy}
\usetikzlibrary{calc,intersections,through,backgrounds,graphs,positioning,shapes,arrows,fit,math}
\usetikzlibrary{patterns,decorations.pathreplacing}
\usetikzlibrary{decorations.pathreplacing}
\usetikzlibrary{external}
\usetikzlibrary{datavisualization}
\usepackage[europeanvoltages,
            europeancurrents,
            europeanresistors,   % rectangular shape
            americaninductors,   % "4-bumbs" shape
            europeanports,       % rectangular logic ports
            siunitx,             % #1<#2>
            emptydiodes,
            noarrowmos,
            smartlabels]         % lables are rotated in a smart way
           {circuitikz}          %
\usepackage{siunitx}
\usepackage{tabularx}
\usetikzlibrary{arrows}
\usepackage{algpseudocode}
\usepackage{algorithm}
\usepackage{gensymb}
\usepackage{mathtools}

% import the listing styles

\usepackage{caption}
\usepackage[mode=buildnew]{standalone}
\usepackage[backend=bibtex]{biblatex}
\begin{document}
\def\chapterauthor#1{{\large #1}\bigskip\bigskip}

\newenvironment{beispiel}{%
\begin{proof}[Beispiel]%
\renewcommand{\qedsymbol}{$\bigcirc$}
}{\end{proof}}
\setcounter{page}{352}
\setcounter{chapter}{16}
\allowdisplaybreaks
\renewcommand{\floatpagefraction}{0.7}
\pagestyle{fancy}
\lhead{}
\rhead{}

\chapter{Kettenbrüche\label{chapter:kettenbruch}}
\lhead{Kettenbrüche}
\begin{refsection}
\chapterauthor{Benjamin Bouhafs-Keller}

%
% einleitung.tex -- Beispiel-File für die Einleitung
%
% (c) 2020 Prof Dr Andreas Müller, Hochschule Rapperswil
%
\section{Einleitung\label{legendre:section:einleitung}}
\rhead{Einleitung}
Zur Berechnung respektive zur Auswertung von zugeordneten Legendrepolynome (\textit{engl.} associated Legendre polynomials) wird normalerweise auf eine Rekursionsbeziehung (\textit{engl.} recurrence relation) zurückgegriffen.
Die Rekursionsbeziehungen werden bevorzugt, da die Alternativen entweder rechnerisch zu aufwändig wären wie beispielsweise die Verwendung der geschlossenen Form (siehe Gleichung \eqref{legendre:geschlosseneform}) oder schlicht nicht valide sind für höherrangige Legendrepolynome wie zum Beispiel das Auflisten der verschiedenen Polynomfunktionen.
\begin{equation}
P^{m}_{l}(x)
=
(-1)^m*2^l*(1-x^2)^{m/2}
* \sum_{k=m}^{l} \frac{k!}{(k-m)!}*x^{k-m}
* \binom{l}{m} \binom{\frac{l+k-1}{2}}{l}
\label{legendre:geschlosseneform}
\end{equation}
Von den Rekursionsbeziehungen gibt es eine ganze Liste voll und es ist gut möglich, dass nicht alle davon numerisch stabile Formeln sind.
Es ist daher möglich, dass bei einer falschen Wahl einer solchen Rekursionsbeziehung numerische Probleme auftreten können, die zu völlig falschen Resultaten führen.
Dass ein solches numerisches Problem auftreten kann, wird leider oft vergessen.
Zusätzlich ist es eine anspruchsvolle Aufgabe, solche numerischen Instabilitäten vorzeitig zu erkennen.
Aus diesen Gründen ist es nicht verwunderlich, dass sogar namhafte Bibliotheken numerisch instabile Implementationen enthalten.
So ist beispielsweise die Implementation der Legendrepolynome auf Wolfram Alpha \cite{legendre:wolfram-alpha} numerisch nicht stabil, wie gut in der Abbildung \ref{legendre:fig:wolframalpha} zu sehen ist.
Aus diesen Gründen befasst sich dieses Kapitel mit der Stabilität oder eben Instabilität der Rekursionsbeziehungen für zugeordnete Legendrepolynome.
Es wird dabei untersucht, wie numerisch instabile Rekursionsbeziehungen erkannt werden können und wieso diese instabil sind.

\begin{figure}[!h]
\centering
\includegraphics[width=0.9\linewidth]{papers/legendre/plots/wolframalpha}
\caption{Von Wolram Alpha \cite{legendre:wolfram-alpha} generierter Plot des Legendrepolynoms mit Grad 50 und Ordnung 3. Deutliche numerische Instabilitäten nahe den Randbereichen.}
\label{legendre:fig:wolframalpha}
\end{figure}

% Unterkapitel Beispiel
%\subsection{Titel Unterkapitel
%\label{legendre:subsection:unterkapitellabel}}

% Quelle Zitieren Beispiel
%\cite{legendre:bibtex}

% Abschnittsverweis Beispiel
%\ref{legendre:section:loesung}

% Gleichung Beispiel
%\begin{equation}
%\int_a^b x^2\, dx
%=
%\left[ \frac13 x^3 \right]_a^b
%=
%\frac{b^3-a^3}3.
%\label{legendre:equation1}
%\end{equation}

% Gleichung Referenzieren Beispiel
%\eqref{legendre:equation1}




%
% rationalezahlen.tex -- 
%
% (c) 2020 Benjamin Bouhars-Keller
%
\section{Rationale Zahlen
\label{kettenbruch:section:Zahlen}}
\rhead{Rationale Zahlen}
\subsection{Definition}
Die Kettenbruch entwicklung von $x \in \mathbb{R}$ bricht genau dann nach endlich
vielen Schritten ab, wenn $x$ rational ist.
Die Menge aller rationalen Zahlen wird mit $\mathbb{Q}$ bezeichnet.
Ein endlicher Kettenbruch ist ein Bruch der Form
\begin{equation}
a_0 + \cfrac{1}{a_1+\cfrac{1}{a_2+\cfrac{\cdots}{\cdots+\cfrac{1}{a_{n-1} + \cfrac{1}{a_n}}}}}
\end{equation}
in welchem $a_0, a_1,\dots,a_n$ und $b_1,b_2,\dots,b_n$ ganze Zahlen
darstellen, die mit Ausnahme möglicherweise von $a_0$ alle positiv sind.
Die Kettenbruchentwicklung von $x \in \mathbb{R}$ bricht genau dann
nach endlich vielen Schritten ab, wenn $x$ rational ist. So bilden
rationale Zahlen endliche Kettenbrüche.

\subsection{Euklidischer Algorithmus}
Die Umwandlung einer rationalen Zahl in einen Kettenbruch erfolgt
mit Hilfe des euklidischen Algorithmus.
Als Beispiel rechnen wir für $\frac{17}{10} = [1;1,2,3]$.
\begin{beispiel}
\begin{equation}
\frac{17}{10}
=
1 + \frac{7}{10}
=
1 + \cfrac{1}{\frac{10}{7}}
=
1 + \cfrac{1}{1+\frac{3}{7}}
=
1 + \cfrac{1}{1+\cfrac{1}{1+\cfrac{1}{\frac{7}{3}}}}
=
1 + \cfrac{1}{1+\cfrac{1}{2+\frac{1}{3}}}
\end{equation}
\begin{align*}
17 &= 1\cdot + 7 \\
10 &= 1\cdot 7 + 3 \\
7 &= 2\cdot 3 + 1 \\
3 &= 3\cdot 1 + 0
\end{align*}
Diese Methode kann auch umgekehrt angewendet werden.
Berechnen wir nun den Kettenbruch in einem Bruch zurück.
\begin{equation}
[1;1,2,3] \rightarrow	2 + \frac{1}{3} = \frac{7}{3}
						1 + \frac{3}{7} = \frac{10}{7}
						1 + \frac{7}{10} = \frac{17}{0}
\end{equation}
\end{beispiel}
Wie man im Beispiel sieht, ist die intuitive Berechnung der
Näherungsbrüche eines Kettenbruchs, indem man ihn von unten her
auflöst, sehr umständlich. Durch ein rekursives Bildungsgesetz für
Zähler und Nenner kann man die Berechnung erheblich vereinfachen.
Ausserdem kann man mit Hilfe dieser Rekursionsformel die Grenzwerte
unendlicher Kettenbrüche untersuchen.

Jede rationale Zahl $x$ lässt sich auf eindeutige Weise in einen
endlichen Kettenbruch entwicklen, dessen letzer Teilnenner grösser
oder gleich 2 ist.

\subsection{Rekursionsformel}
Die Berechnung der Konvergenten (siehe Abschnitt 10.3.3) eines Kettenbruches 
kann erheblich vereinfacht werden, indem eine Rekursionsformeln für den Zähler 
und den Nenner einführt wird.
\begin{beispiel}
\begin{align*}
A_0 &= b_0                     &   B_0 &= 1                                  \\
A_1 &= a_1a_0 + 1              &   B_1 &= a_1                                \\
A_k &= a_kA_{k-1} + A_{k-2}    &   B_k &= a_kB_{k-1} + B_{k-2} &(i &\ge 1),   \\
A_2 &= b_2b_1b_0 + b_2 + b_0   &   B_2 &= b_2b_1 + 1		             \\
A_3 &= b_3b_2b_1b_0 + b_3b_2 + b_3b_0 + b_1b_0 + 1 & B_3 &= b_3b_2b_1 + b_3 + b_1
\end{align*}
Dies lässt sich auch durch die folgende Matrizenschreibweise ausdrücken:
\begin{equation}
P_0 = 	\begin{pmatrix}
			A_{_1}&	A_{_2}\\
			B_{_1}&	A_{_2}
		\end{pmatrix}
		=\begin{pmatrix}
			1&	0\\
			0&	1
		\end{pmatrix}
\end{equation}
\begin{equation}
P_i = 	\begin{pmatrix}
			A_i&	A_{i-1}\\
			B_i&	B_{i-1}
		\end{pmatrix}
		=\begin{pmatrix}
			A_{i-1}&	A_{i-2}\\
			B_{i-1}&	B_{i-2}
		\end{pmatrix} 
		\begin{pmatrix}
			b_i	&	1\\
			1	&	0
		\end{pmatrix} 
\end{equation}
Rekursives Einsetzen ergibt:
\begin{equation}
		\begin{pmatrix}
			A_i&	A_{i-1}\\
			B_i&	B_{i-1}
		\end{pmatrix}
		=\begin{pmatrix}
			b_0	&	1\\
			1	&	0
		\end{pmatrix}
		\begin{pmatrix}
			b_1	&	1\\
			1	&	0
		\end{pmatrix}
		\cdots
		\begin{pmatrix}
			b_i	&	1\\
			1	&	0
		\end{pmatrix} 
		=\displaystyle\prod_{j=0}^{i}\begin{pmatrix}
			b_j	&	1\\
			1	&	0
		\end{pmatrix}
\end{equation}
Aus diesen Überlegungen resultiert eine Aussage, von deren letztem 
Teil wir noch sehr oft Gebrauch machen werden:
Somit gilt 
\begin{equation}
\det(P_i) = (-1)^i
\end{equation}
und für alle $i$:
\begin{equation}
A_iB_{i-1} - B_iA_{i-1} = (-1)^i
\end{equation}
\subsubsection{Beispiel}
Für $x = [b_0;b,\cdots,b_{i-1},\beta_1]$ gilt
\begin{equation}
x = \frac{A_{i-1}\beta_i + A_{i-2}}{B_{i-1}\beta_i + B_{i-2}}
\qquad \text{für alle $i \ge 0$.}
\end{equation}
Aus der Rekursionsformeln ergeben sich folgende Gleichung
\begin{equation}
[b_0;b_1,\cdots,b_n]
=
\frac{A_n}{B_n} = \frac{b_nA_{n-1} + A_{n-2}}{b_nB_{n-1} + B_{n-2}}
\end{equation}
und analog
\begin{equation}
[b_0;b,\cdots,b_{i-1},\beta_1] = [b_0;b_1,\cdots,b_{i-1},\beta_i]
=
\frac{A_{i-1}\beta_i + A_{i-2}}{B_{i-1}\beta_i + B_{i-2}}
\end{equation}
\end{beispiel}
Hiermit haben wir die wichtigsten Zusammenhänge in Bezug auf die Näherungszahlen herausgearbeitet.


%
% irrationalezahlen.tex 
%
% (c) 2020 Benjamin Bouhafs-Keller
%
\section{Irrationale Zahlen
\label{kettenbruch:section:Irrationale Zahlen}}
\rhead{Irrationale Zahlen}
\subsection{Definition}
Zahlen, die nicht rational sind, heissen irrational. Die rationale
und die irrationalen Zahlen bilden zusammen die reellen Zahlen.
Anders formuliert sind irrationale Zahlen von einem Quotienten der
nicht durch zweie ganzer Zahlen darstellbar ist gekenntzeichnet
(Bsp:  $\pi$, $e$
Die Menge aller reellen Zahlen bezeichnet man mit $\mathbb{R}$.

Irrationale Zahlen bilden unendliche Kettenbrüche, d.~h.~sind durch
eine periodische oder nicht periodische Kettenbruchentwicklung
ausgezeichnet.

Ein unendlicher regelmässiger Kettenbruch wird in folgender Form dargestellt
\begin{equation}
a_0 + \cfrac{1}{a_1+\cfrac{1}{a_1+\cfrac{1}{a_3+\cfrac{1}{\cdots}}}}
\end{equation}
wobei die $a_0,a_1,a_2,\dots$ eine unendliche Folge von positiven
ganzen Zahlen bilden. Sie sind auch wie beim endlichen Kettenbruch
alle bis auf möglicherweise $a_0$ positiv.

Zunächst sollen einige Beispiele für die Kettenbruchenentwicklung
irrationaler Zahlen betrachtet werden.

\subsection{Periodische Kettenbrüche}
In diesem Abschnitt wollen wir nun auf eine spezielle Form eingehen
und zwar auf unendliche regelmässige Kettebrüche, die ein bemerkenswertes
Bildungsgesetz befolgen. Das besondere an diesen Kettenbrüchen ist,
dass gleiche Teilnenner wiederholt auftreten.

Für den Kettenbruch $[3;1,2,1,6,1,2,1,6,\dots]$ heisst das: den
Kettenbruch die Periode 1,2,1,6 mit der Periodenlänge $n=4$ beträgt
und wird in der Form $[3;\overline{1,2,1,6}]$ geschrieben.

Kommen wir nun zu einem Beispiel. Wir betrachten den periodischen
einfachen Kettenbruch $[3;\bar{6}] = (3,6,6,6,\dots)$.
\begin{equation}
[3;\bar{6}]
=
3 + \cfrac{1}{6+\cfrac{1}{6+\cfrac{1}{6+\cfrac{1}{6\dots}}}}
=
x
\end{equation}
Die Euklidische Methode mit der rekursive Bildungsgesetz für Zähler
und Nenner würde hier unendlich sein und deshalb schwierig zu
berechnen.

Um diesen Kettenbruch vollständig darzustellen müssen wir ein System
erzeugen. Daher werden nur die ersten Brüche (Zahlen) betrachtet.

\begin{equation}
[3;6;6;6]
=
3 + \cfrac{1}{6+\cfrac{1}{6+\frac{1}{6}}}
=
3 + \cfrac{1}{6+\cfrac{1}{\frac{37}{6}}}
=
3 + \cfrac{1}{6+\frac{6}{37}}
=
3 + \cfrac{1}{\frac{228}{37}}
=
3 + \frac{37}{228}
=
\frac{684+37}{228}
=
\frac{721}{228}
\approx
3.162280702
\end{equation}
Hier wird der Kettenbruch so verändert das eine 6 steht.
Wir können unser Kettenbruch als $x$ bezeichnet und addieren 3
darauf. Dies ergibt folgendes System das wir infolge quadratische
Gleichungen lösen können:
\begin{align*}
y = x+3 &= 6 + \cfrac{1}{6+\cfrac{1}{6+\dots}} = [6;\bar{6}]
\\
\Rightarrow y &= 6 + \frac{1}{y}	&&\vert\;\cdot y
\\
\Rightarrow y^2 &= 6y + 1
\\
\Rightarrow y &= 3\pm \sqrt{9+1} = 3 \pm \sqrt{10}\qquad\Rightarrow\qquad x = \sqrt{10}
\approx
3.16227766
\end{align*}
Jede irrationale Zahl $\Phi$ besitzt unendlich viele Näherungsbrüche
$\frac{p}{q}$ mit
\begin{equation}
\biggl|\Phi-\frac{p}{q}\biggr|<\frac{1}{\sqrt{5 q^2}}.
\end{equation}

Betrachten wir ein anderes System, $(\overline{2,3}) =  (2,3,2,3,\dots)$.
Sein Wert ist als unendlicher Kettenbruch irrational und lässt sich
berechnen, denn setzen wir $x:=(\overline{2,3})$, dann gilt
\begin{equation}
x
=
2 + \cfrac{1}{3+\cfrac{1}{2+\cfrac{1}{3+\dots}}}
=
2 + \cfrac{1}{3+\frac{1}{x}}.
\end{equation}
Dies führt auf die quadratische Gleichung $x^2 - 2x - \frac{2}{3}
= 0$, was die positive Lösung $x = \frac{3+\sqrt{15}}{3}$ liefert.

Der oben aufgeführte Kettenbruch $x$ ist ein Beispiel für periodische
einfache Kettenbrüche, die Nullstelle eines quadratischen Polynoms
mit rationalen Koeffizienten ist. Anders gesagt ist die reelle
Irrationalzahl $x$ Wurzel einer quadratischen Gleichung:
\begin{equation}
ax^2 + bx + c = 0
\end{equation}
dann ist die Kettenbruchentwicklung von $x$ periodisch, das bedeutet
die Existenz einer Schranke $n_0$ und einer Periode $k \in \mathbb{N}$
mit $x_n+k = x_n$ für alle $n\ge n_0$.

\subsubsection{Satz von Euler-Lagrange}
Jeder periodische einfache Kettenbruch ist eine quadratische
Irrationalzahl und umgekehrt. Dabei bezeichnet eine quadratische
Irrationalzahl eine irrationale Zahl und stellt eine algebraische
Zahl dar.

\subsection{Nicht periodische Kettenbrüche}
Es stellen sich dieselben Fragen wie im vorangegangenen Abschnitt.
Neu hinzu kommt das Problem, ob bzw. wann die Kettenbruchentwicklung
überhaupt konvergiert.

Für eine unendliche Folge $x_0,x_1,\dots$ ist der Kettenbruch
$[x_0,x_1,\dots]$ nur dann definiert wenn die Folge der Näherungsbrüche
$(\frac{p_n}{q_n})$ konvergiert. In diesem Fall hat der unendliche
Kettenbruch $[x_0,x_1,\dots]$ den Wert
\begin{equation}
\lim_{n\to\infty} [x_0;x_1;\cdots;x_n]
\end{equation}
oder anders dargestellt
\begin{equation}
\omega
=
x_0 + \cfrac{1}{x_1+\cfrac{1}{x_2+\frac{1}{x_n+\cdots}}}
\end{equation}

Folgt $\omega > 0$ durch einen unendliche Kettenbruch darstellbar
ist, wenn die endlichen Kettenbrüche $n$-ter Ordnung
$[x_0;x_1,x_2,\dots,x_n]$ gegen $\omega$ konvergieren.

\subsubsection{Beweis}
Betrachten wir folgenden Kettenbruch
\begin{align*}
\frac{19}{51} &= [0;2,1,2,6]
\\
	K_0 &= [0] = 0
\\
	K_1 &= [0;2] = 0 + \frac{1}{2} = \frac{1}{2}
\\
	K_2 &= [0;2,1] = 0 + \cfrac{1}{2+\frac{1}{1}} = \frac{1}{3}
\\
	K_3 &= [0;2,1,2] = 0 + \cfrac{1}{2+\cfrac{1}{1+\frac{1}{2}}} = \frac{3}{8}
\\
	K_4 &= [0;2,1,2,6] = \frac{19}{51}
\end{align*}

Folge der Näherungsbrüche
\begin{enumerate}
\item
$K_0 < K_2 < K_4 < \cdots$
\item
$K_1 > K_3 > K_5 > \cdots$
\item
$K_{2s} < K_{2r+1}, r,s \in \mathbb{N}$
\end{enumerate}

Es gilt offensichtlich
$K_0 < K_2 < K_4 < \cdots < K_{2n} < \cdots < K_{2n+1} < \cdots < K_5
< K_3 < K_1$
und $\frac{19}{51}$ wird von jeweils zwei aufeinanderfolgenden
Näherungsbrüchen eingeschlossen.
 
Die Folge der Näherungsbrüche mit geraden Index bilden eine streng
monoton steigende Folge und sind nach oben begrenzt. Also konvergiert
diese Folge von unten gegen einen Grenzwert, den wir als $\alpha$
bezeichnen. Anderseits bilden die Näherungsbrüche mit ungeradem
Index eine streng monoton fallende Folge und sind nach unten begrenzt.
Somit sind beide Folgen monoton und beschränkt und konvergieren in
$\alpha$.

Es gibt auch Zahlen, deren Kettenbruchdarstellung gewisse
Regelmässigkeiten aufweisen, ohne periodisch zu sein. Zum Beispiel
die Identität $e = [2;1,2,1,1,4,1,1,6,1,\cdots]$. Dieser
Kettenbruch ist nicht periodisch, die Teilnenner können aber durch
eine rekursive Folge bestimmt werden.

\subsubsection{Bemerkung}
\begin{itemize}
\item
Jede positive rationale Zahl lässt sich durch einen endlichen
Kettenbruch darstellen, und jeder endliche Kettenbruch stellt eine
positve rationale Zahl dar.
\item
Jeder unendliche Kettenbruch stellt eine positive irrationale Zahl
dar, und jede irrationale Zahl lässt sich durch einen unendliche
Kettenbruch darstellen.
\item
Jeder periodische Kettenbruch stellt eine quadratische Irrationalität
dar und jede quadratische Irrationalität lässt sich durch einen
periodischen Kettenbruch darstellen.
\end{itemize}


%
% approximation.tex
%
% (c) 2020 Prof Dr Andreas Müller, Hochschule Rapperswil
%
\subsection{Approximation
\label{pde:fem:subsection:approximation}}
\index{Approximation}%
Das äquivalente Minimalproblem zu einer partiellen Differentialgleichung
hat das Problem etwas vereinfacht, die Ordnung der Ableitung ist
reduziert worden, aber ist ist immer noch ein Problem, in dem eine
Funktion bestimmt werden muss, also ein unendlichdimensionales Problem.
In dieser Form ist es daher immer noch nicht einer effizienten
numerischen Lösung zugänglich.

\subsubsection{Beispielproblem}
Zur Illustration soll in diesem Abschnitt das folgende Problem 
gelöst werden.
Auf dem Interval $\Omega=[0,\pi]$ ist eine Funktion $f(x)$ gegeben.
Gesucht ist eine Funktion $u\colon [0,\pi]\to\mathbb R$ derart, dass
\begin{equation}
\begin{aligned}
u''(x) &= f(x)
\\
u(0) &= u_0 & u(\pi)&= u_1.
\end{aligned}
\label{buch:pde:approx:beispiel}
\end{equation}
Als erstes müssen wir das äquivalente Minimalproblem finden.

\begin{problem}
Sei $F(x)$ eine Stammfunktion von $f(x)$.
\index{Stammfunktion}%
Eine Lösung des Differentialgleichungsproblems~\eqref{buch:pde:approx:beispiel}
minimiert
\[
I(u)
=
\int_0^\pi (u'(x) - F(x))^2\,dx.
\]
\end{problem}

\begin{proof}[Beweis]
Die Richtungsableitung von $I(u)$ ist
\begin{align*}
\frac{dI}{d\varepsilon}(u+\varepsilon h)\bigg|_{\varepsilon=0}
&=
\int_0^\pi
2(u'(x)+\varepsilon h'(x)-F(x)) h'(x) 
\,dx
\\
&=
2
\int_0^\pi
(
u'(x)
-
F(x)
)
h'(x)
\,dx
\\
&=
2\biggl[(u'(x)-F(x))h(x)\biggr]_0^\pi
-
2\int_0^\pi
(u''(x)
-
F'(x))
h(x)
\,dx.
\intertext{Der erste Term verschwindet wegen $h(0)=h(\pi)=0$ und es bleibt}
&=
2\int_0^\pi
(u''(x)
-
f(x))
h(x)
\,dx
=0.
\end{align*}
Dies muss für alle $h(x)$ gelten, so dass folgt
$u''(x) -f(x)=0$ oder $u''(x)=f(x)$.
\end{proof}

Die Stammfunktion $F(x)$ ist nicht eindeutig bestimmt, vielmehr
ist jede $F(x)+C$ ebenfalls eine Stammfunktion.
Nach obiger Rechnung führt sie jedoch auf die gleichen Minima.

\subsubsection{Approximation mit Polynomen}
\index{Approximation durch Polynome}
Wir können aber davon ausgehen, dass die Lösungsfunktionen einigermassen
glatt sind, also sich gut durch ein Polynom approximieren lässt.
Wir approximieren jetzt $u(x)$ als Polynom und schreiben
\[
u(x) = a_0 + a_1x + a_2x^2 + \dots + a_nx^n.
\]
Das Minimalprinzip für $u(x)$ führt auf ein Minimalprinzip für 
die Koeffizienten $a_k$.
\index{Minimalprinzip}%
Zunächst brauchen wir die Ableitung von $u(x)$:
\begin{align*}
u'(x) &= \sum_{k=1}^n ka_kx^{k-1}.
\end{align*}
Jetzt müssen wir das Integral von
\begin{equation}
(u'(x)-F(x))^2
=
u'(x)^2 - 2u'(x)F(x) + F(x)^2
\label{buch:fem:integrand}
\end{equation}
durch 
die Koeffizienten $a_k$ ausdrücken.
Für die drei Terme rechts in \eqref{buch:fem:integrand} müssen die Integrale
\begin{align*}
\int_0^\pi u'(x)^2\,dx
&=
\int_0^\pi
\sum_{i,j=1}^n ija_ia_jx^{i+j-2}
\,dx
=
\sum_{i,j=1}^n ij \int_0^\pi x^{i+j-2}\,dx\, a_ia_j
=
\sum_{i,j=1}^n
\underbrace{ij \frac{\pi^{i+j-1}}{i+j-1}}_{\displaystyle b_{ij}}
a_ia_j
=
a^tBa
\\
\int_0^\pi u(x)'F(x) \,dx
&=
\int_0^\pi \sum_{i=1}^n ia_i x^{i-1} F(x) \,dx
=
\sum_{i=1}^n
a_i
\underbrace{\int_0^\pi 
ix^{i-1} F(x)\,dx}_{\displaystyle =b_i}
=
b^t a
\\
\int_0^\pi F(x)^2\,dx &=: c
\end{align*}
ausgewertet werden können.
Gesucht ist jetzt also ein Koeffizientenvektor $a$, der den Ausdruck
\index{Koeffizientenvektor}%
\[
Q(a) = a^t B a + b^t a + c
\]
minimiert.
Der Summand $c$ kann natürlich weggelassen werden.

Die Randbedingungen müssen natürlich auch erfüllt sein, auch
diese können wir durch die Polynomkoeffizienten ausdrücken:
\begin{align*}
u_0=u(0)   &= a_0
\\
u_1=u(\pi) &= a_0 + a_1\pi + a_2\pi^2+\dots a_n\pi^n.
\end{align*}
Dies lässt sich auch in Matrixform mit der Matrix
\[
A=\begin{pmatrix}
1&0&0&\dots&0\\
1&\pi&\pi^2&\dots&\pi^n
\end{pmatrix}
\]
als
\[
Aa = \begin{pmatrix}u_0\\u_1\end{pmatrix}
\]
schreiben.

Wir machen das Problem noch etwas konkreter und verlangen $f(x)=1$
und $u_0=u_1=0$.
Dann ist $F(x)=x$ eine mögliche Stammfunktion und damit lässt sich
der Vektor $b$ berechnen als
\[
b_i
=
\int_0^\pi i F(x) x^{i-1}\,dx
=
\int_0^\pi i x^i\,dx
=
\frac{i}{i+1}
\pi^{i+1}.
\]
Die Matrix $B$ ist
\begin{align*}
B&=\begin{pmatrix}
\pi   & \pi^2            & \pi^3            \\
\pi^2 & \frac{4}{3}\pi^3 & \frac{3}{2}\pi^4 \\
\pi^3 & \frac{3}{2}\pi^4 & \frac{9}{5}\pi^5
\end{pmatrix}.
\end{align*}
Aus den Randbedingung folgt $a_0=0$ und
\[
a_1\pi+a_2\pi^2=0
\qquad\Rightarrow\qquad
a_1 = \pi a_2.
\]
Die gesuchte Lösung muss sich also den Ausdruck
\[
a_2
\begin{pmatrix}
1\\\pi
\end{pmatrix}^t
\pi^3
\begin{pmatrix}
\frac43   &\frac32\pi\\
\frac32\pi&\frac95\pi^2
\end{pmatrix}
\begin{pmatrix}
1\\\pi
\end{pmatrix}
a_2
-
\begin{pmatrix}
\frac23\pi^3\\\frac34\pi^4
\end{pmatrix}
\begin{pmatrix}
1\\\pi
\end{pmatrix}
a_2
=
\biggl(
\frac43\pi^3+3\pi^5 +\frac95\pi^7
\biggr)
a_2^2
-\biggl(\frac23\pi^3+\frac34\pi^5\biggr)
a^2.
\]
Gesucht wird also nur das Minimum eines quadratischen Ausdrucks 
in $a_2$, und dieses wird bei $a_2=\frac12$ angenommen.
Tatsächlich kann man nachprüfen, dass
\[
u(x)
=
\frac12\biggl(x-\frac{\pi}2\biggr)^2 -\frac{\pi^2}8
=
\frac12x^2 -\frac12 x\pi +\frac{\pi^2}8 - \frac{\pi^2}8
=
\frac12x\biggl(x-\frac{\pi}2\biggr)
\]
eine Lösung des Problems ist.

\subsubsection{Allgemeine Formulierung}
Etwas allgemeiner kann man das Problem wie folgt formulieren.
Gesucht sei die Lösung der Differentialgleichung $u''(x)=f(x)$ auf
dem Intervall $[a,b]$ mit Randbedingungen $u(a)=u_0$ und $u(b)=u_1$.
Das zugehörige Minimalprinzip verlangt, dass
\[
I(u) = \int_a^b (u'(x) - F(x))^2\,dx,
\qquad \text{mit}\qquad F(x) = \int_a^x f(\xi)\,d\xi
\]
minimiert wird mit der Nebenbedingungen $u(a)=u_0$ und $u(b)=u_1$.
Gesucht ist die Lösung $u(x)$ als Linearkombination einer linear
unabhängigen Menge von Funktionen
$\varphi_0(x),\varphi_1(x),\dots,\varphi_n(x)$.
Zu bestimmen sind die Koeffizienten $a_k\in\mathbb R$ der Linearkombination
\[
u(x) = a_0\varphi_0(x) + a_1\varphi_1(x) + \dots + a_n\varphi_n(x)
=
\sum_{k=0}^n a_k\varphi_k(x).
\]
Das Minimalprinzip kann durch die Koeffizienten $a_k$ als
\begin{align*}
\int_a^b (u'(x)-F(x))^2 \,dx
&=
\int_a^b 
\sum_{i,j=0}^n
a_ia_j
\varphi'_i(x)\varphi'_j(x) 
-2F(x)
\sum_{i=0}^n \varphi'_i(x)
+
F(x)^2
\,dx
\\
&=
\sum_{i,j=0}^n a_ia_j
\underbrace{\int_a^b \varphi'_i(x)\varphi'_j(x)\,dx}_{\displaystyle=b_{ij}}
+
\sum_{i=0}^n a_k \underbrace{\int_a^b \varphi_i'(x)F(x)\,dx }_{\displaystyle=c_i}
+
\int_a^b F(x)^2\,dx
\end{align*}
ausgedrückt werden.
Der letzte Summand hat keinen Einfluss auf das Minimum und kann daher
weggelassen werden.
Die Randbedingungen können ebenfalls vektoriell geschrieben werden:
\[
\left.
\begin{aligned}
u_0&=u(a) = \sum_{i=0}^n a_i\varphi_i(a) \\
u_1&=u(b) = \sum_{i=0}^n a_i\varphi_i(b)
\end{aligned}
\quad
\right\}
\qquad\Rightarrow\qquad
\underbrace{\begin{pmatrix}
\varphi_0(a)& \varphi_1(a) & \dots & \varphi_n(a) \\
\varphi_0(b)& \varphi_1(b) & \dots & \varphi_n(b) \\
\end{pmatrix}}_{\displaystyle =A}
\begin{pmatrix}a_0\\\vdots\\a_n\end{pmatrix}
=
\begin{pmatrix}
u_0\\u_1
\end{pmatrix} = b
\]
Das Problem ist damit auf das quadratische Minimalproblem
\index{quadratisches Minimalproblem}%
\[
\begin{aligned}
&\text{minimiere}&
Q(a) &= a^tBa + c^ta 
\\
&\text{Nebenbedingung:}&
Aa&=b
\end{aligned}
\]
mit der Matrix
\[
B=(b_{ij})
\quad\text{mit}\quad
b_{ij}
= 
\int_a^b \varphi'_i(x)\varphi'_j(x)\,dx
\]
und dem Vektor
\[
c=(c_i)
\quad
\text{mit}\quad
c_i
=
\int_a^b \varphi_i'(x) F(x)\,dx
\]
zurückgeführt.

\subsubsection{Höhere Dimensionen}
Das Beispielproblem der vorangegangenen Abschnitte war eindimensional,
was erlaubt hat, bekannte Formeln wie die partielle Integration aus der
Analysis zu verwenden.
Ziel dieses Abschnitts ist, ein paar Eigenheiten der Verallgemeinerung
auf ein mehrdimensionales Problem zu diskutieren.

Sei $\Omega\subset\mathbb R^2$ ein zweidimensionales Gebiet mit glattem
Rand $\partial \Omega$,
zum Beispiel ein Rechteck wie in Abbildung~\ref{buch:pde:pfadintegral}.
%Es ist klar, dass dies auch auf höhere Dimensionen verallgemeinert werden.
Ausserdem ist $f\colon\Omega\to\mathbb R$ und
$g\colon\partial\Omega\to\mathbb R$ geben.
Das Differentialgleichungsproblem sucht
eine Funktion $u\colon\overline{\Omega}\to\mathbb R$ mit
\begin{equation}
\begin{aligned}
\Delta u&=f &&\text{in $\Omega$}
\\
u&=g&&\text{auf $\partial\Omega$}.
\end{aligned}
\label{buch:pde:eqn:dgl2d}
\end{equation}
Die Rechnung zum eindimensionalen Problem suggeriert, dass 
das äquivalente Minimalproblem 
\begin{equation}
I(u)
=
\int_{\Omega} (\nabla u(x,y) - F(x,y))^2\,dx\,dy
\label{buch:pde:eqn:minimal2d}
\end{equation}
ist, wobei $F(x,y)$ eine beliebige vektorwertige Funktion mit
$\nabla\cdot F(x,y) = f(x,y)$ ist.

In den meisten Fällen ist es einfach, eine solche Funktion zu
finden.
Besonders einfach ist es, wenn die Funktion $f\colon\Omega\to\mathbb R$
stetig zu einer Funktion $\bar{f}\colon\mathbb R^2\to R$ ausgedehnt
werden kann.
Dies ist nicht immer möglich, wie die Funktion $\varphi$ des Polarwinkels
des Punktes $(x,y)$ auf dem Gebiet von
Abbildung~\ref{buch:pde:figure:ringgebiet} zeigt.
\index{Polarwinkel}%
\begin{figure}
\centering
\includegraphics{chapters/70-pde/images/ringgebiet.pdf}
\caption{Die auf dem Gebiet $\Omega$ definiert Funktion, $\varphi$,
die einem Punkt den Polarwinkel $\varphi$ in Polarkoordinaten zuordnet,
kann nicht stetig auf ganz $\mathbb R^2$ ausgedehnt werden.
\index{Polarkoordinaten}%
Der Grund dafür sind die unterschiedlichen Grenzwert im Punkt $P$.
\label{buch:pde:figure:ringgebiet}}
\end{figure}
Gäbe es eine stetige Funktion $\bar{\varphi}$, die $\varphi$ erweitert, dann
müssten die Grenzwerte von $\bar{\varphi}$ im Punkt $P$ von ``oben'' und
von ``unten'' übereinstimmen.
Der Grenzwert im Punkt $P$ hängt aber von der Richtung ab,
es ist
\begin{align*}
\lim_{y\to 0+}\varphi(1,y) &= \lim_{y\to 0+}\arctan\frac{y}{1} = 0
\\
\lim_{y\to 0-}\varphi(1,y) &= 2\pi + \lim_{y\to 0-}\arctan\frac{-y}{1} = 2\pi.
\end{align*}
Die Grenzwerte sind also verschieden.

Nehmen wir jetzt also an, dass es eine stetige Funktion
$\bar{f}\colon\mathbb R^2\to\mathbb R$ gibt mit $\bar{f}(x,y)=f(x,y)$
für $(x,y)\in\Omega$.
Wir betrachten die Vektorfunktion $F(x,y)$ mit Komponenten
\begin{equation}
\left.
\begin{aligned}
F_x(x,y) &= \int_0^x \bar{f}(\xi, y)\,d\xi \\
F_y(x,y) &= 0
\end{aligned}
\quad
\right\}
\qquad\Rightarrow\qquad
F(x,y)
=
\begin{pmatrix}F_x(x,y)\\F_y(x,y)\end{pmatrix}
=
\begin{pmatrix}F_x(x,y)\\0\end{pmatrix}.
\label{buch:pde:eqn:divFf}
\end{equation}
Die Divergenz von $F(x,y)$ ist
\begin{align*}
\nabla\cdot F(x,y)
=
\frac{\partial F_x}{\partial x}(x,y)
+
\frac{\partial F_y}{\partial y}(x,y)
=
\frac{\partial}{\partial x}
\int_0^x \bar{f}(\xi, y)\,d\xi
=
f(x,y).
\end{align*}
Die Funktion $F(x,y)$ definiert in \eqref{buch:pde:eqn:divFf}
ist also eine Funktion der gesuchten Art.
Die Erweiterbarkeit der Funktion $f(x,y)$ auf $\bar{f}(x,y)$ 
stellt sicher, dass das Integral und damit $F(x,y)$ eine stetige Funktion
von $x$ und $y$ wird, so dass das Minimalproblem wohldefiniert ist.

Für das vorgeschlagene Minimalprinzip können wir jetzt wieder die
Richtungsableitung für eine Änderung von $u$ um eine Funktion $h$
mit verschwindenden Randwerten berechnen.
\index{Richtungsableitung}%
Wie früher finden wir
\begin{align*}
\frac{dI(u+\varepsilon h)}{d\varepsilon}\bigg|_{\varepsilon=0}
&=
\int_\Omega 2(\nabla u(x,y) -F(x,y))\cdot \nabla h(x,y)\,dx\,dy
\\
&=
\int_{\partial\Omega} (\nabla u(x,y) - F(x,y))h(x,y)\cdot d\vec{n}
-
\int_{\Omega} (\Delta u(x,y) -\nabla F(x,y)) h(x,y) \,dx\,dy
\intertext{Darin verschwindet der erste Term, da $h(x,y)=0$ ist auf dem Rand.
Es bleibt}
&=
-
\int_{\Omega} (\Delta u(x,y) - f(x,y)) h(x,y) \,dx\,dy.
\end{align*}
Da dieser Ausdruck für jede Funktion $h$ mit $h(x,y)=0$ mit
$(x,y)\in\partial\Omega$
verschwinden muss, folgt wieder, dass
\[
\Delta u = f
\]
in $\Omega$ gelten muss.
Das Minimalprinzip~\eqref{buch:pde:eqn:minimal2d} gehört also tatsächlich
zur Differentialgleichung~\eqref{buch:pde:eqn:dgl2d}.

%Für die Ausdehnbarkeit der Funktion $f$ vom Gebiet $\Omega$ mit glattem Rand
%auf $\mathbb R^2$ genügt es, dass sich $f$ stetig auf den Rand
%$\partial \Omega$ fortsetzen.
%Damit werden Fälle wie das Beispiel in
%Abbildung~\ref{buch:pde:figure:ringgebiet}
%bereits ausgeschlossen.




%
% problemstellung.tex -- Beispiel-File für die Beschreibung des Problems
%
% (c) 2020 Prof Dr Andreas Müller, Hochschule Rapperswil
%

\documentclass{scrartcl}

\usepackage[utf8]{inputenc}
\usepackage[T1]{fontenc}
\usepackage{lmodern}
\usepackage[ngerman]{babel}
\usepackage{amsmath}
\usepackage{physics}
\usepackage{mathrsfs}
\usepackage{graphicx}

\begin{document}

\section{Folgerungen
\label{laplace:section:folgerungen}}
Damit die Approximation von $\tilde{f}(t)$ möglichst genau $f(t)$ repräsentiert, müssen die passenden Werte für $\lambda, \sigma $ und $\nu $ gefunden werden. 
Dies geschah in diesem Falle rein iterativ durch probieren.

Insbesondere wurden folgende Funktionen ausgewertet $ F_{1}(s)=\frac{1}{1-s} $ und $F_{2}(s) = \frac{2}{s^{3}}$. 
Die Werte welche für die Auswertung verwendet wurden sind in der untenstehenden Tabelle abgebildet.

\begin{center}
\begin{tabular}[c]{c|c|c}
& $F_{1}(s)=\frac{1}{1-s}$ & $F_{2}(s) = \frac{2}{s^{3}}$ \\
\hline
Parameter & $\lambda=1.288$ & $\lambda=0.101$ \\
& $\sigma=1.000001$ & $\sigma=0.965$ \\
& $\nu=0.81$ & $\nu=0.098953$ \\
\end{tabular}
\end{center}
Die untenstehenden Abbildungen zeigen die Verläufe der absoluten Fehler.

\begin{figure}[h]
\begin{center}
\includegraphics[width=8cm]{"Error_1divide_sminus1"}
\caption{Fehler von $F_{1}(s)$}
\end{center}
\end{figure}

\begin{figure}[h]
\begin{center}
\includegraphics[width=8cm]{"Error_1divide_sminus1_bis_tgleich5"}
\caption{Fehler von $F_{1}(s)$}
\end{center}
\end{figure}

\begin{figure}[h]
\begin{center}
\includegraphics[width=8cm]{"Error_2divide_s_pow3"}
\caption{Fehler von $F_{2}(s)$}
\end{center}
\end{figure}

\begin{figure}[h]
\begin{center}
\includegraphics[width=8cm]{"Error_2divide_s_pow3_bis_tgleich5"}
\caption{Fehler von $F_{2}(s)$}
\end{center}
\end{figure}


Es ist deutlich ersichtlich, dass der Fehler nur für ein gewisses Zeitintervall akzeptabel ist. Im Bereich wo der Fehler sich in einem tolerierbaren Bereich befindet, wurden die Parameter $\lambda, \sigma $ und $\nu $ für ein bestimmtes $t_{0}$ ermittelt. 
Diese Parameter sind für andere Zeitpunkt $t_{x}$ nicht mehr sinnvoll. 
Für den Zeitpunkt $t_{0}=1$ wurden die Fehler in der Grössenordnung $10^{-8}$ respektive $10^{-6}$ erreicht. 
Dies geschah mittels Erhöhung der Iterationszahl N.

\begin{center}
\begin{tabular}[c]{c|c|c}
& $F_{1}(s)$ & $F_{2}(s)$ \\
\hline
Parameter & $\lambda=1.288$ & $\lambda=0.101$ \\
 & $\sigma=1.000001$ & $\sigma=0.965$ \\
 & $\nu=0.81$ & $\nu=0.098953$ \\
\hline
$t_{0}$ & $1$ & $1$ \\
\hline
Fehler & $1.7076~*~10^{-08}$ & $1.8447~*~10^{-06}$ \\
\end{tabular}
\end{center}

\end{document}



\printbibliography[heading=subbibliography]
\end{refsection}
\end{document}
