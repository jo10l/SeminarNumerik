  
%
% rationalezahlen.tex -- 
%
% (c) 2020 Benjamin Bouhars-Keller
%
\section{Rationale Zahlen
\label{kettenbruch:section:Zahlen}}
\rhead{Rationale Zahlen}
\subsection{Definition}
Eine rationale Zahl ist eine Zahl, die man als Bruch $\frac{z}{n}$ mit ganzen
Zahlen, $z$ und $n \ne 0$ schreiben kann. $z$ heisst der Zähler,
$n$ der Nenner. Dieser Bruch stellt jene Zahl dar, die mit $n$
multipliziert $z$ ergibt, also $\frac{z}{n} \cdot n = z$.
Die Menge aller rationalen Zahlen wird mit $\mathbb{Q}$ bezeichnet.
Ein endlicher Kettenbruch ist ein Bruch der Form
\begin{equation}
a_0 + \cfrac{b_1}{a_1\cfrac{b_2}{\cfrac{a_2}{\cdots}\frac{b_n-1}{a_n-1 + \frac{b_n}{a_n}}}}
\end{equation}
in welchem $a_0, a_1,\dots,a_n$ und $b_1,b_2,\dots,b_n$ reelle Zahlen
darstellen, die mit Ausnahme möglicherweise von $a_0$ alle positiv sind.
Die Kettenbruchentwicklung von $x \in \mathbb{R}$ bricht genau dann
nach endlich vielen Schritten ab, wenn $x$ rational ist. So bilden
Rationale Zahlen endliche Kettenbrüche.

\subsection{Euklidischer Algorithmus}
Die Umwandlung einer rationalen Zahl in einen Kettenbruch erfolgt
mit Hilfe des euklidischen Algorithmus.
Als Beispiel rechnen wir für $\frac{17}{10} = [1;1,2,3]$.

\begin{equation}
\frac{17}{10}
=
1 + \frac{7}{10}
=
1 + \cfrac{1}{\frac{10}{7}}
=
1 + \cfrac{1}{1+\frac{3}{7}}
=
1 + \cfrac{1}{1+\cfrac{1}{1+\cfrac{1}{\frac{7}{3}}}}
=
1 + \cfrac{1}{1+\cfrac{1}{2+\frac{1}{3}}}
\end{equation}
\begin{align*}
17 &= 1\cdot + 7 \\
10 &= 1\cdot 7 + 3 \\
7 &= 2\cdot 3 + 1 \\
3 &= 3\cdot 1 + 0
\end{align*}
Wie man es im Beispiel sieht, ist die intuitive Berechnung der
Näherrungsbrüche eines Kettenbruchs, indem man ihn von unten her
auflöst, sehr umständlich. Durch eine rekursive Bildungsgesetz für
Zähler und Nenner kann man die Berechnung erheblich vereinfachen.
Ausserdem kann man mit Hilfe dieser Rekursionsformel die Grenzwerte
unenlicher Kettenbrüche untersuchen.
Betrachten wir die Zahl $\frac{3141}{2718}$
\begin{align*}
3141 &= 1\cdot 2718 + 423 \\
2718 &= 6\cdot 423 + 180 \\
423 &= 2\cdot 180 + 63 \\
180 &= 2\cdot 63 + 54 \\
63 &= 1\cdot 54 + 9 \\
54 &= 6\cdot 9 + 0
\end{align*}
Also
\begin{equation}
\frac{3141}{2718} = [1,6,2,2,1,9]
\end{equation}
Jede rationale Zahl $x$ lässt sich auf eindeutige Weise in einen
endlichen Kettenbruch entwicklen, dessen letzer Teilnenner grösser
oder gleich 2 ist.