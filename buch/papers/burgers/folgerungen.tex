% !TEX root = ../../buch.tex
% problemstellung.tex -- Beispiel-File für die Beschreibung des Problems
%
% (c) 2020 Prof Dr Andreas Müller, Hochschule Rapperswil
%
\section{Resultate
\label{burgers:section:results}}
\rhead{Resultate}

	Folgend werden die Berechnungen f\"ur eine Normalverteilte Startbedingung mit der expliziten und impliziten Methode gezeigt.
	Um den Vergleich zwischen den verschiedenen Methoden besser aufzuzeigen, werden die Resultate in 2-D aufgezeigt.
	F\"ur jeden Zeitschritt wird einen neuen Subplot gezeigt.
	
	\subsection{Explizit}

    \begin{figure}
	\centering
	\includegraphics[width=1\textwidth]{papers/burgers/BurgersEquation/lin_paper.pdf}
	\caption{L\"osung Linear}
	\label{burgers:fig:lin}
	\end{figure}
	
	\subsection{Implizit}

    \begin{figure}
	\centering
	\includegraphics[width=1\textwidth]{papers/burgers/BurgersEquation/imp_paper.pdf}
	\caption{L\"osung Implizit}
	\label{burgers:fig:imp}
	\end{figure}
