%
% problemstellung.tex -- Beispiel-File für die Beschreibung des Problems
%
% (c) 2020 Prof Dr Andreas Müller, Hochschule Rapperswil
%
\section{Problemstellung
\label{perturbation2:section:problemstellung}}
\rhead{Idee}

Das Eigenwertproblem sucht die Vektoren $v_i \in \mathbb {R}^{n} $, die angewendet mit einer Matrix $\bm H$ nicht die Richtung ändern und sich dabei nur mit $\lambda_i \in \mathbb R$ skalieren:
\begin{equation}
    \bm H \bm v_i = \lambda_i \bm v_i
\end{equation}

Manche Applikation benötigen nur Eigenwerte einer Matrix $\bm H(\varepsilon)$, die nur wenig von der Matrix $\bm H$ mit bekannten Eigenwerten und Eivenvektoren abweicht.
Dies lässt sich ausdrücken mit der summe
\begin{gather*}
    \bm H = \bm H_0 + \varepsilon \bm H_1 \\
    \varepsilon \in \mathbb{R^+} \ll 1 \\
    \lambda^{(0)}, \bm v^{(0)} \quad \text{bekannt},
\end{gather*}
wobei $\varepsilon \ll 1 $ ausdrücken soll, dass der zweite Term der Summe viel kleiner ist als $\bm H_0$

Die Erigenwertperturbation erlaubt es, die Eigenwerte $\lambda_i(\varepsilon)$ und Eigenvektoren  $v_i(\varepsilon)$ von $\bm H$ zu approximieren.
Das Verfahren hat jedoch die limitierung, dass die Eigenvektoren von $H_0$ zueinander othogonal sein, was für alle symetrischen $\bm H$ zutrifft.

\section{Anwendungen}

Die Eigenwertperturbation kann überall angewendet werden wo ein grobes modell mit kleinen Einflüssen erweitert werden kann.

Das wohl grösste Anwendungsgebiet ist das lösen der der Schrödingergleichung in der Quantenmechanik.
Aus dieser Anwendung wurde die Eigenwertperturbationstheorie auch entwickelt. %TODO check this
%TODO elaborate
                
Einfluss von Planeten and die Umlaufbahn anderer Planeten

Berechnung einer Trajektorie unter berücksichtigung der Luftfeuchtigkeit %TODO ref paper in matsem

\section{Idee}


$\bm H(\varepsilon)$ kann als Taylorreihe geschrieben werden.

\begin{equation*}
    \bm H(\varepsilon) = \bm H_0 + \varepsilon \bm H_1 \GR{ + \varepsilon^2 \bm H_2  + \varepsilon^3 \bm H_3 + \dots}
\end{equation*}

\begin{align*}
    \bm v_i(\varepsilon) = \bm v_{0i} + \varepsilon \bm v_{1i} \GR{ + \varepsilon^2 \bm v_{2i}  + \varepsilon^3 \bm v_{3i} + \dots} \\
    \lambda_i(\varepsilon) = \lambda_{0i} + \varepsilon \lambda_{1i} \GR{ + \varepsilon^2 \lambda_{0i}  + \varepsilon^3 \lambda_{3i} + \dots}
\end{align*}

Als approximation genügen wir uns mit einer Approximation erster ordnung. 



Gesucht ist nun eine lösung für das Gleichungssystem
\begin{align}
    \bm H(\varepsilon) \bm v_i(\varepsilon) &= \lambda_i(\varepsilon) \bm v_i(\varepsilon) \\
    (\bm H_0 + \varepsilon \bm H_1)
    (\bm v_{0i} + \varepsilon \bm v_{1i})
    &=
    (\lambda_{0i} + \varepsilon \lambda_{1i})
    (\bm v_{0i} + \varepsilon \bm v_{1i}) \\
    & \phantom{2} \vdots \nonumber\\
    \bm v_{0j}^T \bm H_0 \bm v_{0i}
    &=
    \delta_{ij} \lambda_{1i} + 
    ( \lambda_{0i} - \lambda_{0j} )
    \bm v_{0j}^T  \bm v_{0i} .
\end{align}

\begin{equation*}
    \delta_{ij} = \bm v_{0j}^T \bm H_0 \bm v_{0i}
    = \begin{cases}
        0 \quad (i \neq j),\\
        1 \quad (i = j)
        \end{cases}
\end{equation*}




\begin{alignat*}{3}
    i = j \quad & \rightarrow  \quad && \lambda_{1i}&& = \bm v_{0i}^T \bm H_1 \bm v_{0i} \\
    i \neq j \quad & \rightarrow \quad && \bm v_{0j}^T \bm v_{1i}&& = \frac{\bm v_{0j}^T \bm H_1 \bm v_{0i}}{\lambda_{0i} - \lambda_{0j}}
\end{alignat*}

Berechnung der Eigenwerte

\begin{align*}
    \lambda_i(\varepsilon)
    &=
    \lambda_{0i} + \varepsilon \lambda_{1i} \\
    &=
    \lambda_{0i} + \varepsilon \bm v_{0i}^T \bm H_1 \bm v_{0i}
\end{align*}

Berechnung der Eigenvektoren

\begin{align*}
    \bm v_i(\varepsilon)
    &=
    \bm v_{0i} + \varepsilon \bm v_{1i} \\
    &=
    \bm v_{0i} + \varepsilon \sum_{j} ( \bm v_{0j}^T \bm v_{1i}) \, \bm v_{0j} \\
    &=
    \bm v_{0i} + \varepsilon ( \bm v_{0i}^T \bm v_{1i}) \bm v_{0i} + \varepsilon \sum_{j \neq i} (\bm v_{0j}^T v_{1i}) \, \bm v_{0j}
\end{align*}

\begin{align*}
    \bm v_i(\varepsilon)
    &=
    \bm v_{0i} + \varepsilon ( \bm v_{0i}^T \bm v_{1i}) \bm v_{0i} + \varepsilon \sum_{j \neq i} (\bm v_{0j}^T \bm v_{1i}) \, \bm v_{0j} \\
    &=
    \bm v_{0i} ( 1 + (\bm v_{0i}^T \bm v_{1i}) ) + \varepsilon \sum_{j \neq i}
    \frac{\bm v_{0j}^T \bm H_1 \bm v_{0i}}{\lambda_{0i} - \lambda_{0j}}
    \, \bm v_{0j} \\
    &=
    \bm v_{0i} ( 1 + \mathrm{Im}(\varepsilon \gamma) ) + \varepsilon \sum_{j \neq i}
    \frac{\bm v_{0j}^T \bm H_1 \bm v_{0i}}{\lambda_{0i} - \lambda_{0j}}
    \, \bm v_{0j}
    \quad
    \QED
\end{align*}


Zusammenfassend, 

\begin{align*}
    \lambda_i(\varepsilon)
    &=
    \lambda_{0i} + \varepsilon \bm v_{0i}^T \bm H_1 \bm v_{0i}\\
    \bm v_i(\varepsilon)
    &=
    \lambda_{0i} + \varepsilon \bm v_{0i}^T \bm H_1 \bm v_{0i}
    \bm v_{0i} ( 1 + \mathrm{Im}(\varepsilon \gamma) ) + \varepsilon \sum_{j \neq i}
    \frac{\bm v_{0j}^T \bm H_1 \bm v_{0i}}{\lambda_{0i} - \lambda_{0j}}
    \, \bm v_{0j}
\end{align*}



\section{Entartung}

Was, wenn zwei oder mehr Eigenwerte gleich sind?

Die Eigenvektoren sind nicht mehr eindeutig.

Sie werden von numerischen Programmen einfach gewählt:


Entartete Eigenwerte können Division durch Null auslösen

\begin{equation*} %TODO maybe ref to equation
    \bm v_i(\varepsilon)
    =
    \bm v_{0i} ( 1+ \mathrm{Im}(\varepsilon \gamma)) + \varepsilon \sum_{j \neq i}
    \frac{\GN{\bm v_{0j}^T \bm H_1 \bm v_{0i}}}{\RD{\lambda_{0i} - \lambda_{0j}}}
    \, \bm v_{0j}
\end{equation*}

Kein Problem, wenn Zähler auch Null ist %TODO not true since 0/0.. show with previous equation

Falls entartet, wähle $\bm v_{0i}$, so dass
\begin{equation*}
    \GN{\bm v_{0j}^T \bm H_1 \bm v_{0i}} = 0 \quad \forall \quad i,j \quad entartet
\end{equation*}


Bilde $\bm H_1$ auf basis von entarteten, zufällig gewählten Eigenvektoren ab
\begin{equation*}
    \bm H^\prime = \bm v_{0i}^T \bm H_1 \bm v_{0i} \quad \forall \quad i \quad entartet
\end{equation*}
Löse kleines Eigenwertproblem (alle $\lambda_i$ sind noch bekannt)
\begin{equation*}
    \bm H^\prime \bm v_{i}^\prime = \lambda_{i} \bm v_i^\prime \quad \forall \quad i \quad entartet
\end{equation*}
Transformiere gefundene Eigenvektoren zurück und verwende diese als die neuen, korrekten Eigenvektoren
\begin{equation*}
    \bm v_{0i} \gets \bm v_{0i} \bm v_{i}^\prime  \quad \forall \quad i \quad entartet
\end{equation*}

\begin{align*}
    \lambda_i(\varepsilon)
    & \gets
    \lambda_{0i} + \varepsilon \bm v_{0i}^T \bm H_1 \bm v_{0i}\\
    \bm H^\prime & \gets \bm v_{0i}^T \bm H_1 \bm v_{0i} \quad \forall \quad i \quad entartet \\
    \bm v^\prime & \gets \mathrm{Eig} \Big( \bm H^\prime \Big) \\
    \bm v_{0i} & \gets \bm v_{0i} \bm v^\prime  \quad \forall \quad i \quad entartet \\
    \bm v_i(\varepsilon)
    & \gets
    \lambda_{0i} + \varepsilon \bm v_{0i}^T \bm H_1 \bm v_{0i}
        \bm v_{0i} ( 1 + \mathrm{Im}(\varepsilon \gamma) ) + \varepsilon \sum_{j \neq i, \,nicht\,entartet}
        \frac{\bm v_{0j}^T \bm H_1 \bm v_{0i}}{\lambda_{0i} - \lambda_{0j}}
        \, \bm v_{0j}
\end{align*}