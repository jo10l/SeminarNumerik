%
% problemstellung.tex -- Beispiel-File für die Beschreibung des Problems
%
% (c) 2020 Prof Dr Andreas Müller, Hochschule Rapperswil
%

\section{Problemstellung
\label{pade:section:Problemstellung}}
Eine Padé-Approximation ist ein Bruch welcher aus zwei Polynomen, welche aus den Koeffizienten der Taylorreihe einer Funktion gewonnen werden, gebildet wird. 
Das Ziel dieses Kapitels ist es dem Leser den Nutzen der Padé-Approximation näher zu bringen und zu zeigen wie man aus einer analytischen Funktion eine solche Approximation bilden kann.
Des weiteren wird auf mehrere praktische Beispiele eingegangen und mögliche Fehlerquellen aufgezeigt. 



\subsection{Potenzreihen
\label{pade:subsection:Potenzreihen}}
\rhead{Potenzreihen}
Die Koeffizienten einer Potenzreihe können verwendet werden um die Koeffizienten einer Padé-Approximation zu gewinnen. 
In diesem Kapitel soll kurz aufgezeigt werden wie man die Potenzreihe einer Funktion erhalten kann. 

In der Analysis kann eine Funktion mit einer Taylorreihe um eine Stelle $x_{0}$ durch eine Potenzreihe dargestellt werden. 
Diese Potenzreihen werden um einen vorgegebene Stelle $x_{0}$ als eine unendliche Summe 
\begin{equation}
f(x)=\sum_{n=0}^{\infty} a_{n} (x-x_{0})^{n} 
\label{pade:expofunk}
\end{equation}
gebildet.
Für viele Funktionen sind die dazugehörigen Potenzreihen schon bekannt. 
Funktionen welche durch eine Potenzreihe dargestellt werden können, werden auch analytische Funktionen genannt.
Es gibt verschiedene Methoden um eine Potenzreihe einer Funktion zu erhalten. 
Bei analytischen Funktionen sind die Potenzreihen schon bekannt und können etwa durch Differentialgleichungen oder eine andere Reihenentwicklungsmethoden hergeleitet werden.
Wie die Herleitung mit einer Potenzreihe mit einer Differentialgleichung funktioniert wird in dem Abschnitt \ref{pade:section:Bsp_Potenzreihen} gezeigt.

\subsubsection{Beispiel Potenzreihen
\label{pade:section:Bsp_Potenzreihen}}
In diesem Beispiel möchten wir die Potenzreihe der Exponentialfunktion erhalten welche an der Stelle $x_0 = 0$ entwickelt wird. 
Wir wissen welche Eigenschaften die Exponentialfunktion hat und nutzen genau diese um Gleichungen aufzustellen. 
Gesucht ist eine Potenzreihe welche das Verhalten 
\begin{equation*}
	f^{\prime}(x) = f(x) , \text{ für alle } x \in \mathbb{R} 
\end{equation*}
aufweist.

Die Form der Potenzreihe ist gegeben durch die schon gezeigte Summe \ref{pade:expofunk}.
Weiter wissen wir das die Exponentialfunktion 
\begin{equation*}
f(0) = 1
\end{equation*}
erfüllen muss.
Aus dem können wir schliessen, dass
\begin{equation*}
f(x)=\sum_{n=0}^{\infty} a_{n} \cdot x^{n}
\qquad\Rightarrow\qquad
\sum_{n=0}^{\infty} a_{n} \cdot 0^{n} 
=
a_{0} \cdot 0^{0} + a_{1} \cdot 0^{1} + a_{2} \cdot 0^{2} \dots = 1
\qquad\Rightarrow\qquad
a_{0} = 1
\end{equation*}
und somit ist der erste Koeffizient $a_0$ schon gefunden.
Anschliessend wird das Polynom abgeleitet

\begin{equation*}
f^{\prime}(x)
=
\sum_{n=0}^{\infty}(n+1) \cdot a_{n+1} \cdot x^{n}
\end{equation*}
um die weiteren Koeffizienten $a_n$ zu erhalten.
Da immer noch die Anforderung $f(x) = f^{\prime}(x)$ gilt, wissen wir das die Koeffizienten vor $x^n$ 
\begin{equation*}
(n+1) \cdot a_{n+1} 
= 
a_{n} , \text{ für alle } n \in \mathbb{R}
\end{equation*}
sein müssen. 
Mit dem schon bekannten $a_0 = 1$ folgt rekursiv
\begin{equation*}
a_{n+1} 
= 
\frac{1}{(n+1)!}
\qquad\Rightarrow\qquad
a_{n} 
= 
\frac{1}{n!}
\end{equation*}
und damit sind die Koeffizienten des Polynoms der Exponentialfunktion schon ermittelt.
Ausgeschrieben 


\begin{equation}
e^{x}
=
\sum_{n=0}^{\infty} \frac{x^{n}}{n !}
=
\frac{x^{0}}{0 !}+\frac{x^{1}}{1 !}+\frac{x^{2}}{2 !}+\frac{x^{3}}{3 !}+\cdots 
\label{pade:potenzexp}
\end{equation}
Aus der bekannten Potenzreihe der Exponentialfunktion \ref{pade:potenzexp} können nun auch andere Potenzreihen gewonnen werden.
Für die Potenzreihen gilt alle $x \in \mathbb{R}$ d.h., der Konvergenzradius der Potenzreihe ist unendlich. 
Aus der Exponentialfunktion kann sogleich die Potenzreihe für den Sinus und Kosinus ermittelt werden.
Um dies zu erreichen wird zuerst die Reihe Komplex erweitert
\begin{equation*}
e^{x}
=
\sum_{n=0}^{\infty} \frac{x^{n}}{n !}
\qquad\Rightarrow\qquad
e^{ix}
=
\sum_{n=0}^{\infty} \frac{ix^{n}}{n !}.
\end{equation*}
Die komplexe Reihe bringen wir dann in eine sehr bekannte Form durch den Gebrauch von Euler 
\begin{align*}
e^{ix}
&=
\sum_{n=0}^{\infty} \frac{(ix)^{n}}{n !}
=
\frac{(ix)^{0}}{0 !}+\frac{(ix)^{1}}{1 !}+\frac{(ix)^{2}}{2 !}+\frac{(ix)^{3}}{3 !}+\cdots
\\
&=
\left(1-\frac{x^{2}}{2 !}+\frac{x^{4}}{4 !}-\ldots\right)+\mathrm{i} \cdot\left(x-\frac{x^{3}}{3 !}+\frac{x^{5}}{5 !}-\ldots\right)
\\
&=
\cos(x)+i\cdot \sin(x).
\end{align*}
Man sieht gleich das die komplexen Zahlen bei den geraden Exponenten verschwinden und bei denn ungeraden Exponenten noch bestehen bleiben. 
Sortiert man nun das Polynom nach komplex und reellen werten erkennt man die $cis$ Form und kann diese beide Teile als Sinus und Kosinus aufschreiben.
Wobei die Sinusfunktion 
\begin{equation*}
\sin (x)=\sum_{n=0}^{\infty}(-1)^{n} \frac{x^{2 n+1}}{(2 n+1) !}=\frac{x}{1 !}-\frac{x^{3}}{3 !}+\frac{x^{5}}{5 !} \mp \cdots
\end{equation*}
und die Kosinusfunktion 
\begin{equation*}
\cos (x)=\sum_{n=0}^{\infty}(-1)^{n} \frac{x^{2 n}}{(2 n) !}=\frac{x^{0}}{0 !}-\frac{x^{2}}{2 !}+\frac{x^{4}}{4 !} \mp \cdots
\end{equation*}
beide als einzelne Summen ausgeschrieben werden können.



\subsection{Padé-Approximation erstellen
	\label{pade:subsection:Pade_erstellen}}

Die Padé-Approximation 
\begin{equation}
[L/M] = 
=
\frac{a_0 + a_1 x + \dots + a_L x^L}{den}
\end{equation}






