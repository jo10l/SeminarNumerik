%
% problemstellung.tex -- Beispiel-File für die Beschreibung des Problems
%
% (c) 2020 Prof Dr Andreas Müller, Hochschule Rapperswil
%
\section{Problemstellung
\label{fem:section:problemstellung}}
\rhead{Problemstellung}

\subsection{Was sind Ansatzfunktionen?}
Um eine gesuchte Funktion als Lösung für die Partielle Differentialgleichung (DGL) Problem in der Ebene zu finden, wird diese mit einfacheren Funktionen approximiert. Diese einfacheren Approximations- Funktionen werden in diesem Kapitel als Ansatzfunktion $u(x)$ bezeichnet. 

Als einfachere Funktionen bieten sich Polynome mit niedrigem Grad an. Diese sind einfach integrier- und differenzierbar. Der Polynomgrad 1 ist einfach anzuwenden bzw. zu berechnen. Polynomgrade 3 und 4 beispielsweis bieten dafür eine bessere Approximation an. Wie im Kapitel 7 ist auch hier das Ziel Gleichungen mit wenigen Koeffizienten zu erstellen. Ist es nun möglich ebenfalls in der Ebene Ansatzfunktionen zu finden mit wenigen Koeffizienten, obwohl über ein Gebiet integriet wird? Ja. Allerdings gibt es besondere Herausforderungen zu beachten, die in den nächsten Zeilen genauer beschrieben werden.
%Die Ansatzfunktion ist frei wählbar und gibt auch den Freiheitsgrad vor. Je höher die Ordnung der Ansatzfunktion desto besser soll die Approximation werden. So die Hoffnung. Allerdings hat dies auch Konsequenzen, die im Verlauf dieses Kapitels beschrieben werden. 


\subsection{Herausforderung FEM in der Ebene}
Wie auch in der Dimension 1 wird das gesamte Gebiet $\Omega$ bzw. Ebene in Teilgebiete unterteilt. Auf diese Teilgebiete werden dann die einfachen Ansatzfunktionen angewendet und dann aufsummiert nach \eqref{fem:equationSummGebiete}.
\begin{equation}
\int_{\Omega} (\nabla u)^2 \, dx \, dy = \sum \limits_{i=1}^n \int_{\Omega_i} (\nabla u)^2 \, dx \, dy 
\label{fem:equationSummGebiete}
\end{equation}
Eine Approximation eines gebietes kann Beispielsweise mit Dreiecken, Rechtecken oder Parallelogrammen vorgenommen werden. Aus dieser Approximation resultieren zwei wesentliche Herausforderungen die da wären:
\begin{itemize}
	\item 1. stetig differenzierbar in den Stützstellen (Steigung) wie z.B: im Dreieck in den Ecken.
	\item 2. stetig differenzierbar an den Übergängen entlang den Rändern eines Elements zum anliegenden Rand eines benachbarten Elements. (Krümmung)
\end{itemize}
\begin{figure}[h!]
	\centering
	\includegraphics[scale=0.8]{papers/fem/Images/Rand.jpeg}
	\caption{differenzierbar entlang eines Randes}
	\label{fig:Randbedingung}
\end{figure}
Aus der Abbildung \ref{fig:Randbedingung} lässt sich erkennen, dass die Ansatzfunktion für jedes Element (hier Dreieck) unterschiedlich ist. Auch müssen die Ansatzfunktion flexibel auf die grösse des Flächenelements sein.
Ein weiterer Unteschiedlich zeigt sich im Vergleich der Differential Gleichung mit 1 Dimension und 2 Dimensionen. Die DGL wird folgender massen geändert 
\begin{equation}
	u'' = \lambda u \rightarrow \Delta u = \lambda u 
	\label{fem:DGL2D}
\end{equation} 
während der Laplace Operator die 2. Ableitungen nach den beiden Variablen darstellt.

\begin{equation}
	\Delta = \frac{\partial ^2}{\partial x^2} + \frac{\partial ^2}{\partial y^2}
\end{equation} 
Daraus lässt sich erkennen, dass sich die Vorgehensweise angepasst werden muss, da zweifache Differenzierbarkeit gefordert ist. %Zudem muss die Ansatzfunktion die Approximation genügen genau beschreiben. 
%Poisson-Gleichung:

%\begin{equation}
%\frac{\partial^2 u(x,y)}{\partial x^2} \frac{\partial^2 u(x,y)}{\partial y^2} = - u(x,y)  \in %\Omega
%\label{fem:equation5}
%\end{equation}

%Randbedingungen:
%\begin{equation}
%u = 0, (x,y)\in \Omega
%\label{fem:rand1}
%\end{equation}

%\begin{equation}
%u = U (x,y)\in \Omega
%\label{fem:rand2}
%\end{equation}

%\begin{equation}
%\frac{\partial u}{n} = 0, (x,y)\in \Omega
%\label{fem:rand3}
%\end{equation}
 
%$\Rightarrow$ 4 Parameter $\Rightarrow$ Polynom 3. Grades\\


%\begin{equation}
%\iint_{\!\!\!\!\!\!\!\Omega} \limits (u_2^2 + u_y^2)) \,dx dy
%\label{fem:equation1}
%\end{equation}

%\begin{equation}
%\iint_{\!\!\!\!\!\!\!\Omega} \limits u^2  \,dx dy
%\label{fem:equation2}
%\end{equation}

%\begin{equation}
%\iint_{\!\!\!\!\!\!\!\Omega} \limits u  \,dx dy
%\label{fem:equation3}
%\end{equation}
Aus den Integralen der Teilgebiete muss dann die Lösung gefunden werden um die Koeffizienten der folgenden Gleichung zu finden.

\begin{equation}
\int g(x) \space dx = f_i \int h_0 \space dx+ f_{i+1}\int h_1 \space dx + s_i\int h_0^1 dx + s_{i+1}\int h_1^1 dx
\label{fem:equation5}
\end{equation}
Da es verschiedene Dreieck- Arten gibt sowie auch verschiedene Parallelogramme, wird in xXx auch eine Lösung aufgezeigt wie mit Hilfe einer Transformation das Dreieck Flächenelements in ein weniger aufwändigere berechenbare Flächenelement überführt werden kann.
Nochmals kurz zusammengefasst was bis hier hin aufgezeigt wurde.
\begin{itemize}
	\item Ebene wird in Teilgebiete unterteilt
	\item Teilgebiete können Dreiecke, Rechtecke oder Parallelogramme sein
	\item Ansatzfunktion soll ein Polynom nierigen gerades sein
	\item auf jedes Teilgebiet wird eine Ansatzfunktion angewendet
\end{itemize} 
Was bis jetzt noch nicht klar ist, wie die Ansafunktion sich zusammenstellt. Dies wird under anderem im folgenden Kapitel beschrieben.

%\subsection{De finibus bonorum et malorum
%\label{fem:subsection:finibus}}

%\begin{equation}
%\int_a^b x^2\, dx
%=
%\left[ \frac13 x^3 \right]_a^b
%=
%\frac{b^3-a^3}3.
%\label{fem:equation1}
%\end{equation}



