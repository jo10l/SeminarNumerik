%
% problemstellung.tex -- Beispiel-File für die Beschreibung des Problems
%
% (c) 2020 Prof Dr Andreas Müller, Hochschule Rapperswil
%
\section{Folgerungen
\label{fem:section:folgerungen}}
\rhead{Folgerungen}
In diesem Kapitel wurde die Bildung von Ansatzfunktionen in der Ebene eingeführt sowie die Eigenschaften einiger Typen aufgezeigt. Es wurde dabei gezeigt, dass:
\begin{itemize}
	\item dass die Wahl der Ansatzfunktionen bestimmte Voraustzungen erfüllen müssen
	\item es unterschiedliche Arten von Ansatzfunktionen gibt
	\item was der Sinn dieser Ansatzfunktionen ist
\end{itemize}
Auch wurde gezeigt wie die Berechnungen durch die Transformation in ein Einheitsform vereinfacht wird und wie diese z.B. am Dreieck durchgeführt wird.


\subsection{De finibus bonorum et malorum
\label{fem:subsection:malorum}}
At vero eos et accusamus et iusto odio dignissimos ducimus qui
blanditiis praesentium voluptatum deleniti atque corrupti quos
dolores et quas molestias excepturi sint occaecati cupiditate non
provident, similique sunt in culpa qui officia deserunt mollitia
animi, id est laborum et dolorum fuga. Et harum quidem rerum facilis
est et expedita distinctio. Nam libero tempore, cum soluta nobis
est eligendi optio cumque nihil impedit quo minus id quod maxime
placeat facere possimus, omnis voluptas assumenda est, omnis dolor
repellendus. Temporibus autem quibusdam et aut officiis debitis aut
rerum necessitatibus saepe eveniet ut et voluptates repudiandae
sint et molestiae non recusandae. Itaque earum rerum hic tenetur a
sapiente delectus, ut aut reiciendis voluptatibus maiores alias
consequatur aut perferendis doloribus asperiores repellat.


