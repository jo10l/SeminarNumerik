%
% loesung.tex -- Beispiel-File für die Beschreibung der Loesung
%
% (c) 2020 Prof Dr Andreas Müller, Hochschule Rapperswil
%
\section{Lösung
\label{fem:section:loesung}}
\rhead{Lösung}

\subsection{Minimalproblem bilden}
Zuerst muss die DGL in ein äqualentes Minimalprobelm übersetzt werden in der Form

Im Gegensatz zu den finiten Elementen im 1 dimensionalen Raum muss bei der Bildung der abgeschwächten Form bzw. in der Übersetzung in das Minimalproblem der Fakt des höheren Dimension mitberücksichtigt werden. Daher wird anstatt der Form für den 1 dimensionalen Fall

\begin{equation}
			\int_0^1 u \textcolor{red}{'} (x)^2 + \lambda u(x)^2 dx
			\label{fem:Minimal1D}
\end{equation}

die mehrdimensionale Form mit dem Laplace Operator verwendet

\begin{equation}
			\int_{\Omega} (\textcolor{red}{\nabla} u)^2 + \lambda u^2 dx
			\label{fem:Minimal2D}
\end{equation}

Wie zu erkennen ist, wird die Ableitung im 1 Dimensionalen durch einen Laplace Operator ersetzt in der Anwendung in der Ebene.

\subsection{Vorbereitung}

Damit die Integration über das gewählte Flächenelement Dreieck oder Parallelogramm einfacher fällt, wird als Vorbereitung im entsprechenden Fall das gewählte Flächenelement in ein Einheitsdreieck oder in ein Eineheitsparallelogramm transformiert. 

Das Allgemeine Dreieck mit den Eckpunkten $P_1(x_1, y_1$),$ P_2(x_2, y_2)$ und $P3(x_3,y_3)$ kann mit Hilfe der linearen Transformation:

\begin{equation}
			x = x_1 + (x_2 - x_1)\xi + (x_3 - X_1)\eta
			y = y_1 + (y_2 - y_1)\xi + (y_3 - y_1)\eta
			\label{fem:linTransformation}
\end{equation}

auf das gleichschenklig rechtwinklige Einheitsdreieck mit Kathetenlänge 1 überfürt werden.

\begin{equation}
			J = \left[ \begin{array}{rr}
\frac{\partial x}{\partial \xi} & \frac{\partial y}{\partial \xi}  \\
\frac{\partial x}{\partial \eta} & \frac{\partial y}{\partial \eta}  \\
\end{array}\right] 
			\label{fem:JocobiDeterminante}
\end{equation}

\begin{equation}
			dy dy = J d\xi d\eta
			\label{fem:newTransformation}
\end{equation}

\begin{equation}
			\xi_x = \frac{y_3 - y_1}{J}, 
			\label{fem:newTransformation}
\end{equation}

\subsection{Ansatzfunktionen wählen}

Als Ansatzfunktionen stehen diverse Standard- Funktionen in der Ebene zur Verfügung. Sie reichen von einfacheren linearen Termen bis hinzu quadratischen. Es kann in der ebene sowohl mit Dreicecken als auch mit Rechtecken gearbeitet werden. Die Wahl der Ansatzfunktionen bzw. der Flächen- Elementen sollte so erfolgen, dass diese die zu approximierende Fläche möglichst gut nachbildet.

In einem ersten Schritt wird die lineare Ansatzfunktionen mit Einheitsdreecken angewendet.

Das Einheitsdreieck hat 

%\begin{equation}
%A =  \sum_{k=0} \iint_{\!\!\!\!\!\!\!\Omega} \epsilon \cdot \nabla N_k(x,y) \cdot N_l(x,y) %dS = \sum \iint_{\!\!\!\!\!\!\!\Omega} h(x) \cdot N_l(x,y) dS, l = 1,2,...N
%\label{fem:equation3}
%\end{equation}

%Zu lösendes GL- System
%Au(x,y) = b
%\label{fem:GL}
\%end{equation}

%wobei A eine Matrix, u der Vektor der Unbekannten und b der Ansatzfunktionenvektor ist.

\subsubsection{linearer Ansatzfunktion mit Dreiecken
\label{fem:subsection:bonorum}}

Die lineare Ansatzfunktion wir gemäss folgender Formel angewendet bzw verwendet.

\begin{equation}
u(x,y) = c_1 + c_2x + c_3y
\label{fem:equationSchwarzLinear}
\end{equation}

Zu dieser Ansatzfunktion wird dann noch die Formfunktion eines Einheitsdreieckes verwendet.


\begin{equation}
\int_a^b \int_c^{d-(d-c)(x-a)/(b-a)} f(x,y) dx dy
\label{fem:equation3}
\end{equation}




\begin{equation}
c_i= \int_{\triangle i} (\nabla u)^2 +\lambda u^2 dx
\label{fem:equation3}
\end{equation}

oder\\
mit einem Allgemeinen Dreieck:

\begin{equation}
N(x,y) = \int_a^b\int_c^{d-(d-c)(x-a)/(b-a)} u(x,y) dxdy
\label{fem:Dreieck_alg}
\end{equation}

Gemäss Abschnitt 2.2.2 Buch von Schwarz sind die Werte entlang einer Seite eines Dreiecks- Elements gleich  der angrenzenden Linie eines anderen Dreiecks, wenn die Eckpunktwerte gleich sind. Die Frage Warum soll dass so sein? Weil Funktion $h(x)$ bei jedem Element gleich ist sowie es sich bei $h(x)$ hierbei um eine Lineare Funktion handelt.

\subsection{Quadratischer Ansatz
\label{fem:subsection:bonorum}}
 Abschnitt 
