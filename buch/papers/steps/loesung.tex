%
% loesung.tex -- Beispiel-File für die Beschreibung der Loesung
%
% (c) 2020 Prof Dr Andreas Müller, Hochschule Rapperswil
%
\section{Lösung
\label{steps:section:loesung}}
\rhead{Lösung}
Um die bestmögliche Schrittweite zu bestimmen wird meist eine Fehlerschätzung verwendet.
Diese erhält man beispielsweise indem mann vom aktuellen Punkt aus einen Probeschritt macht.
Dabei reicht auch beim Runge-Kutta Algorithmus ein einfacher Eulerschritt aus.

\subsection{Simple Schrittweitensteuerung mit konstannter Testschrittweite
\label{steps:subsection:simplestep}}

\subsection{Simple Schrittweitensteuerung mit dynamischer Testschrittweite
\label{steps:subsection:dynstep}}



