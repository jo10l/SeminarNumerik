%
% main.tex -- Paper zum Thema <cg>
%
% (c) 2020 Hochschule Rapperswil
%
\chapter{Die Methode der konjugierten Gradienten\label{chapter:cg}}
\lhead{Konjugierte Gradienten}
\begin{refsection}
\chapterauthor{Raphael Unterer}
\index{Unterer, Raphael}%
\index{konjugierte Gradienten}%

{\parindent0pt
Die} Methode der konjugierten Gradienten (kurz CG) bezeichnet einen schnellen Algorithmus zum Lösen grosser linearer Gleichungssysteme.
Grosse lineare Gleichungssysteme zu lösen ist ein klassisches numerisches Problem.
Diese treten in diversen Anwendungen auf, unter anderem beim Lösen von partiellen Differenzialgleichungen.

Im Kapitel \ref{chapter:linsys} wurden bereits einige numerische Algorithmen vorgestellt um lineare Gleichungssysteme approximativ zu lösen.
Die Methode der konjugierten Gradienten löst im Gegensatz dazu ein lineares Gleichungssystem nicht approximativ sondern exakt.
Dazu benötigt CG genau $N$ Schritte, wobei $N$ die Anzahl Gleichungen und Unbekannten bezeichnet.

Im Rahmen dieses Kapitels wird dieser CG-Algorithmus hergeleitet und einige Untersuchungen werden vorgenommen.

\section{Voraussetzungen\label{cg:section:voraussetzungen}}
\rhead{Voraussetzungen}

Der CG-Algorithmus versucht ein (grosses) lineares Gleichungssystem der Form
\begin{equation}
Ax = b \quad x, b \in \mathbb{R}^N
\end{equation}
zu lösen.
Wir definieren folgende Voraussetzungen für die $N\times N$ Matrix $A$:
\begin{itemize}
	\item $A$ ist symmetrisch
	\item $A$ ist positiv definit, d.h. $x^T A x > 0$
\end{itemize}
Diese Voraussetzungen sind die Bedingung, damit das CG-Verfahren erfolgreich verläuft.
Für eine schnelle Konvergenz sind ausserdem gut konditionierte Matrizen von Vorteil, mit wenigen Einträgen abseits der Diagonalen.
%
% einleitung.tex -- Beispiel-File für die Einleitung
%
% (c) 2020 Prof Dr Andreas Müller, Hochschule Rapperswil
%
\section{Gradient Descent\label{cg:section:steepest_descent}}
\rhead{Gradient Descent}

Um den CG-Algorithmus zu verstehen, ist es hilfreich zuerst die Gradient Descent Methode zu analysieren.
Gradient Descent ist eine bekannte Methode um iterativ ein Minimierungsproblem zu lösen.
Dabei wird immer eine gewisse Schrittweite weit entlang des Gradienten des Minimierungsproblems abgestiegen.
Eine Darstellung von einem 2-dimensionalen Minimierungsproblem mit Gradient Descent findet sich in Abbildung \ref{cg:abb:steepest_descent}.

\begin{figure}	
	\centering
	\includegraphics{papers/cg/images/descent-1}
	\caption{Gradient Descent für ellipsenförmige Niveaulinien (schlechte Konditionierung) in 2D. 
		Abbildung aus dem Seminar Buch von 2014 \cite{cg:book:hpc}.}
	\label{cg:abb:steepest_descent}
\end{figure}

\subsection{Minimierungsproblem \label{cg:subsection:Minimierungsproblem}}

Eine Lösung für $x$ kann durch Minimieren von
\begin{equation}
\Phi(x) = \frac{1}{2} x^T A x - x^T b
\end{equation}
gefunden werden.
Der folgende Beweis zeigt, wieso dies ein sinnvoller Ansatz ist.

\begin{proof}[Beweis]
	Wir definieren eine zweite Variable $z = x + \lambda y$, was uns erlaubt die folgende Differenz auszurechnen
	\begin{align}
	\Phi(z) - \Phi(x) 
	&= 
	\frac{1}{2} \left(x + \lambda y\right) ^T A \left(x + \lambda y\right)  - \left(x + \lambda y\right) ^T b
	- 
	\frac{1}{2} x^T A x + x^T b 
	\\
	&= 
	\frac{1}{2} \left(x^T A x + x^T A \lambda y + \lambda y^T A x + \lambda y^T A \lambda y\right) 
	-
	x^T b - \lambda y^T b
	- 
	\frac{1}{2} x^T A x + x^T b .	
	\end{align}
	Da $A$ symmetrisch ist, können die Terme $x^T A \lambda y$ und $\lambda y^T A x$ zusammengefasst werden (analog zur binomischen Formel)
	\begin{align}
	\Phi(z) - \Phi(x) 
	&= 
	\frac{1}{2}\cancel{ x^T A x} + x^T A \lambda y + \frac{1}{2} \lambda y^T A \lambda y
	-
	\bcancel{x^T b} - \lambda y^T b
	- 
	\frac{1}{2}\cancel{ x^T A x} + \bcancel{x^T b} \\
	&=
	\lambda x^T A y	+ \frac{1}{2} {\lambda}^2 y^T A y - \lambda y^T b \\
	&=
	\frac{{\lambda}^2}{2} y^T A y + \lambda y^T \left(Ax -b \right) .
	\end{align}
	Nun können wir den Beweis führen, indem wir $\Phi(z) \ge \Phi(x)$ setzen (da $\Phi(x)$ ja minimiert wird)
	\begin{align}
	\Phi(z) &\ge \Phi(x) 
	\\
	\Phi(z) &= \frac{{\lambda}^2}{2} y^T A y + \lambda y^T \left(Ax -b \right) + \Phi(x) 
	\\
	0 &\le \frac{{\lambda}^2}{2} y^T A y + \lambda y^T \left(Ax - b \right) \quad \forall \quad y \in \mathbb{R}^N  .
	\end{align}
	Der erste Term ist dabei quadratisch in $\lambda$, $A$ ist positiv definit und somit ist immer $\frac{{\lambda}^2}{2} y^T A y \ge 0$.
	Beim zweiten Term ist diese Bedingung nur erfüllt für alle $y$, wenn $Ax - b = 0$.
	Damit ist bewiesen, dass eine Lösung für die Gleichung $Ax = b$ durch Minimierung von $\Phi(x)$ gefunden wird.
\end{proof}

\subsection{Berechnung der Schrittweite \label{cg:subsection:schrittweite}}
Falls eine Suchrichtung gegeben ist, kann die optimale Schrittweite bestimmt werden um $\Phi(x)$ minimal werden zu lassen.
Gegeben: 
\begin{itemize}
	\item Aktueller Index $k$
	\item Suchrichtung $d_k$
	\item Startpunkt $x_k$
\end{itemize}
Wir suchen nun die optimale Schrittweite $\alpha$, um möglichst nahe an die Lösung zu kommen in der gegebenen Suchrichtung.
Dazu stellen wir wieder ein Minimierungsproblem auf
\begin{equation}
	\Phi(x_k + \alpha d_k) 
	= 
	\frac{1}{2} x_k^T A x_k + \alpha x_k^T A d_k + \frac{1}{2} {\alpha}^2 d_k^T A d_k
	-
	x_k^T b - \alpha d_k^T b .
\end{equation}
In $\alpha$ haben wir hier eine quadratische Gleichung, welche einer nach oben geöffneten Parabel entspricht (da $d_k^T A d_k \ge 0$).
Somit ist es möglich ein klares Minimum zu finden, indem wir die Gleichung nach $\alpha$ ableiten und null setzen:
\begin{equation}
	\frac{\partial \Phi(x_k + \alpha d_k) }{\partial \alpha}
	= 
	x_k^T A d_k + \alpha  d_k^T A d_k - d_k^T b
	=
	0 .
\end{equation}
Dies ergibt für $\alpha$ 
\begin{equation}
	\alpha
	= 
	\frac{d_k^T b - x_k^T A d_k}{d_k^T A d_k}
	=
	\frac{d_k^T \left(b - A x_k\right)}{d_k^T A d_k}.
\end{equation}
Wenn wir nun den Fehler der momentanen Approximation als Residuum $r_k = b - A x_k$ bezeichnen erhalten wir
\begin{equation}
	\alpha
	= 
	\frac{\langle d_k , r_k \rangle}{\langle d_k , d_k \rangle_A},
\end{equation}
wobei $\langle d_k , d_k \rangle_A = d_k^T A d_k$ das verallgemeinerte Skalarprodukt zu $A$ darstellt.
Somit haben wir nun die optimale Schrittlänge gefunden.

Daraus lässt sich nun der nächste Punkt $x_{k+1}$ berechnen als
\begin{equation}
	x_{k+1} = x_k + \frac{\langle d_k , r_k \rangle}{\langle d_k , d_k \rangle_A} d_k.
\end{equation}

\subsection{Berechnung des Gradienten}
Die neue Suchrichtung $d_{k+1}$ entspricht dem negativen Gradienten von $\Phi(x_{k+1})$.
Dieser Gradient lässt sich berechnen als
\begin{equation}
	d_{k+1} = - \nabla \Phi(x_{k+1}) = - \nabla \frac{1}{2} x_{k+1}^T A x_{k+1} - x_{k+1}^T b = -(Ax_{k+1} - b) = b - Ax_{k+1}.
\end{equation}
Somit ist die neue Abstiegsrichtung identisch zum neuen Residuum $r_{k+1}$.

\subsection{Probleme beim Gradient Descent}
Mit den hier hergeleiteten Resultaten kann eine approximative Lösung gefunden werden.
Die Konvergenzgeschwindigkeit ist allerdings stark abhängig von der Form von $A$.
Das Beispiel in Abbildung \ref{cg:abb:steepest_descent} zeigt (in 2D), wie Gradient Descent bei ovalen Niveaulinien zu oszillieren beginnt.
Die Konvergenz ist also in diesem Fall sehr langsam, die exakte Lösung wird nie erreicht.
Falls die Niveaulinien allerdings eher rund sind, erhöht sich die Konvergenzgeschwindigkeit.

Diese Probleme werden mit der Erweiterung zum CG-Algorithmus behoben.

\section{Herleitung des Algorithmus}
\label{cg:section:herleitung}
\rhead{Herleitung des Algorithmus}

\subsection{Intuition}
Die Idee des CG-Algorithmus ist, die Schwächen von Gradient Descent auszubessern.
Dies geschieht, indem pro Dimension des Minimierungsproblems nur ein Schritt benötigt wird.
Intuitiv kann man sich dies als Abstieg entlang der Koordinatenachsen vorstellen, wie in Abbildung \ref{cg:abb:koordabstieg} gezeigt.
Ein solches Verfahren findet immer in $N$ Schritten die exakte Lösung.

\begin{figure}	
	\centering
	\includegraphics{papers/cg/images/descent-2}
	\caption{2D Abstieg entlang der Koordinatenachsen, 2 Schritte genügen um die exakte Lösung zu finden. 
		Abbildung aus dem Seminar Buch von 2014 \cite{cg:book:hpc}.}
	\label{cg:abb:koordabstieg}
\end{figure}

Auf den ersten Blick ist dies sehr vielversprechend.
Allerdings ist die Konvergenz dieses Verfahrens unter Umständen sehr schlecht, da die Koordinatenachsen in hochdimensionalen Problemen weit weg vom Gradienten sind.
Dabei haben viele Dimensionen keinen richtigen Einfluss und ein Abstieg in deren Richtung verbessert die Approximation nur minimal.
In Abbildung \ref{cg:abb:koordabstieg2} sieht man dieses Verhalten noch besser an einem 3D Beispiel.
Es wäre also wünschenswert einen Algorithmus zu finden, welcher:
\begin{itemize}
	\item in $N$ Schritten die exakte Lösung findet
	\item trotzdem schnell konvergiert (eine gute Approximation findet sich bereits nach weniger Schritten)
\end{itemize}
Dies wird mit dem CG-Algorithmus erreicht.

\begin{figure}	
	\centering
	\includegraphics[width=0.8\hsize]{papers/cg/images/descent3d.jpg}
	\caption{3D Abstieg entlang der Koordinatenachsen, 3 Schritte genügen um die exakte Lösung zu finden. 
		Abbildung aus dem Seminar Buch von 2014 \cite{cg:book:hpc}.}
	\label{cg:abb:koordabstieg2}
\end{figure}



\subsection{Optimale Suchrichtung \label{cg:subsection:suchrichtung}}

Erreicht werden kann dies, indem jeweils orthogonalisierte Richtungen verwendet werden.
Kombiniert mit der optimalen Abstiegslänge aus \ref{cg:subsection:schrittweite} führt dies dazu, dass genau $N$ Schritte zur exakten Lösung führen.
Wie beim Abstieg entlang der Koordinatenachsen werden also nur neue Richtungen $d_{k+1}$ verwendet, welche orthogonal (in $A$) auf allen bisherigen Richtungen stehen
\begin{equation} 
	d_{k+1}  \perp_A  d_k  \perp_A  d_{k-1}  \dots \perp_A  d_0.
\end{equation}
Jetzt stellt sich also noch die Frage, wie diese orthogonalen Richtungen zu wählen sind um eine schnelle Konvergenz zu erreichen.
Es bietet sich wiederum der negative Gradient $r_{k+1} = b - Ax_{k+1}$ (Residuum) an, welcher danach mithilfe eines Orthogonalisierungsverfahrens orthogonalisiert werden kann.
Der negative Gradient entspricht per Definition der lokal optimalen Richtung um schnell abzusteigen und ist somit die beste Wahl.

Wir verwenden das Gram-Schmidt-Orthogonalisierungsverfahren um den Gradienten auf den bisherigen Richtungen zu orthogonalisieren.
Dabei wollen wir orthogonal in Bezug auf $A$ sein, weshalb das verallgemeinerte Skalarprodukt $\langle \dots \rangle_A$ verwendet wird 
\begin{equation}\label{cg:eq:gram}
	d_{k+1} 
	= 
	r_{k+1} - \sum_{i=0}^{k} \frac{\langle d_i , r_{k+1} \rangle_A}{\langle d_i , d_i \rangle_A} d_i.
\end{equation}
Diese vielen Orthogonalisierungen sind rechenintensiv und würden die Performance des CG-Algorithmus beeinträchtigen.
Im nächsten Abschnitt wird gezeigt, dass es genügt das Orthogonalisierungsverfahren auf der letzten Abstiegsrichtung durchzuführen.
Die neue Richtung ist dann automatisch auch auf allen vorherigen Richtungen orthogonal.
Dies führt zur folgenden, vereinfachten Berechnung
\begin{equation} 
	d_{k+1}
	= 
	r_{k+1} - \frac{\langle d_k , r_{k+1} \rangle_A}{\langle d_k , d_k \rangle_A} d_k.
\end{equation}
Abbildung \ref{cg:abb:cg1} zeigt wie in einem zweidimensionalen Problem in 2 Schritten die exakte Lösung gefunden wird und schon nach Schritt 1 eine möglichst gute Approximation erreicht wird.
Die orthogonalisierungen sind besser im dreidimensionalen Beispiel von Abbildung \ref{cg:abb:cg2} zu sehen.
Orthogonale Vektoren bezüglich $A$ werden auch als konjugiert bezeichnet, weshalb der Name konjugierte Gradienten Sinn macht.
\begin{figure}	
	\centering
	\includegraphics{papers/cg/images/descent-3}
	\caption{2D Abstieg mit dem CG-Algorithmus, 2 Schritte genügen um die exakte Lösung zu finden. 
		Abbildung aus dem Seminar Buch von 2014 \cite{cg:book:hpc}.}
	\label{cg:abb:cg1}
\end{figure}

\begin{figure}	
	\centering
	\includegraphics[width=0.8\hsize]{papers/cg/images/cg3d-large.jpg}
	\caption{3D Abstieg mit dem CG-Algorithmus, 3 Schritte genügen um die exakte Lösung zu finden. 
		Abbildung aus dem Seminar Buch von 2014 \cite{cg:book:hpc}.}
	\label{cg:abb:cg2}
\end{figure}

\subsection{Wieso genügt Orthogonalisierung auf der letzten Richtung?}
Für die Effizienz des Algorithmus ist es entscheidend, dass eine einzige Orthogonalisierung genügt.
Wieso dies funktioniert, soll hier in abgekürzter Form bewiesen werden.

\begin{proof}[Beweis]
Für den Beweis genügt es den Fall $b=0$ zu betrachten, da $b$ nur einen Offset bei der Residuumsberechnung darstellt.
Damit eine $A$-Orthogonalisierung des Residuums auf $d_k$ genügt, muss das Residuum bereits $A$-orthogonal auf den vorherigen $\langle d_1, \dots ,d_{k-1} \rangle$ Richtungen stehen
\begin{equation} \label{cg:eq:ortho1}
	r_{k+1} \perp_A \langle d_1, \dots ,d_{k-1} \rangle.
\end{equation} 
Aus dem Aufbau des Algorithmus lässt sich die Vermutung ableiten, dass die Residuen und die Richtungen den selben Raum aufspannen.
Wir nehmen also an, dass
\begin{equation}\label{cg:eq:ortho2}
\langle r_1, \dots ,r_{k-1} \rangle 
= 
\langle d_1, \dots ,d_{k-1} \rangle.
\end{equation}
Die Korrektheit dieser Annahme wird nachträglich bewiesen. 
Mit dieser Eigenschaft lässt sich die Gleichung \eqref{cg:eq:ortho1} umstellen zu
\begin{align}\label{cg:eq:ortho3}
	r_{k+1} 	&\perp_A \langle r_1, \dots ,r_{k-1} \rangle \nonumber \\
	0 			&= \langle r_{k+1}, r_i \rangle_A \quad \forall i < k-1 \nonumber\\
				&= \langle r_{k+1}, Ar_i \rangle \quad \forall i < k-1 
\end{align} 
Wir nehmen ausserdem an, dass sich die Residuen als Linearkombinationen von $r_1$ und $A$ ausdrücken lassen
\begin{equation}
	\langle r_1, \dots ,r_k \rangle = \langle r_1, Ar_1 \dots ,A^{k-1}r_1 \rangle.
\end{equation}
Dann können wir die Menge der Vektoren $Ar_i$ aus \eqref{cg:eq:ortho3} schreiben als
\begin{align}\label{cg:eq:ortho4}
	A \langle r_1, r_2, \dots , r_{k-1} \rangle &= A \langle r_1, Ar_1 \dots ,A^{k-2}r_1 \rangle \nonumber\\
												&= \langle Ar_1, A^2r_1 \dots ,A^{k-1}r_1 \rangle \nonumber\\
												&\subset \langle r_1, Ar_1 \dots ,A^{k-1}r_1 \rangle = \langle r_1, r_2, \dots , r_k \rangle.
\end{align} 
Da der Algorithmus eine optimale Schrittweite verwendet, steht der neue Vektor $x_{k+1}$ $A$-orthogonal auf allen bisherigen Abstiegsrichtungen $d_1, \dots, d_k$.
Dieses Verhalten ist gut sichtbar in Abbildung \ref{cg:abb:cg1}.
Da $r_{k+1} = -Ax_k$ (weil $b=0$), führt die $A$-Orthogonalität von $x_{k+1}$ dazu, dass $r_{k+1}$ orthogonal auf $d_1, \dots, d_k$ steht
\begin{equation}\label{cg:eq:ortho5}
	\langle x_{k+1}, d_i \rangle_A = \langle Ax_{k+1}, d_i \rangle = \langle r_{k+1}, d_i \rangle = 0 \quad \forall i \le k.
\end{equation}
Durch Anwenden von Gleichung \eqref{cg:eq:ortho2} auf \eqref{cg:eq:ortho4} ist die Menge der Vektoren $Ar_i$
\begin{equation}
	\langle r_1, r_2, \dots , r_k \rangle = \langle d_1, d_2 \dots ,d_k \rangle.
\end{equation} 
Wenn wir dies mit Gleichung \eqref{cg:eq:ortho5} vergleichen, sehen wir dass dadurch die Bedingung aus Gleichung \eqref{cg:eq:ortho3} erfüllt ist.
Somit ist die Anfangsvermutung $r_{k+1} \perp_A \langle d_1, \dots ,d_{k-1} \rangle$ bewiesen.
\end{proof}

In diesem Beweis haben zwei Teile gefehlt, welche nun noch nachträglich bewiesen werden.
Das erste wäre die Annahme, dass die Residuen und die Richtungen den selben Raum aufspannen $\langle r_1, \dots ,r_{k-1} \rangle = \langle d_1, \dots ,d_{k-1} \rangle$.

\begin{proof}[Beweis]
Wir beginnen damit den Algorithmus für die Berechnungen von $d_k$ aufzuschreiben (vgl. Gleichung \eqref{cg:eq:gram})und erhalten
\begin{align}
	d_1 &= r_1 \nonumber\\
	d_2	&= r_2 - a_2 d_1\nonumber\\
	d_3	&= r_3 - a_{31} d_1 - a_{32} d_2\nonumber\\
	d_k &= r_k - \sum_{i=1}^{k-1} a_{ki} d_i.
\end{align}
Durch Umstellen nach $r_k$, sieht man dass $r_k$ aus $\langle d1, d2, \dots, d_k \rangle$ linear kombiniert werden kann.
Dasselbe macht man für $d_k$, wobei $d_i$ mit $i<k$ aufgelöst wird bis $d_1 = r_1$.
Daraus folgt dass auch $d_k$ aus $\langle r_1, r_2, \dots, r_k \rangle$ linear kombiniert werden kann.
Somit ist $\langle r_1, \dots ,r_{k-1} \rangle = \langle d_1, \dots ,d_{k-1} \rangle$ bewiesen.
\end{proof}

Das zweite wäre die Annahme, dass sich die Residuen als Linearkombinationen von $r_1$ und $A$ ausdrücken lassen $\langle r_1, \dots ,r_k \rangle = \langle r_1, Ar_1 \dots ,A^{k-1}r_1 \rangle$.

\begin{proof}[Beweis]
Wir beginnen wiederum damit, den Algorithmus aufzuschreiben (diesmal für $r_k$) und erhalten
\begin{align}
	r_1 &= -Ax_1 \\
	r_2	&= -Ax_2 = -A(x_1-r_1) = Ar_1 - r_1 \nonumber\\
	r_3 &= -Ax_3 = -A(x_2-r_2) = Ar_2 - r_2 \nonumber\\
	r_k &= Ar_{k-1} - r_{k-1} \in \langle r_1, Ar_1, \dots, A^{k-1}r_1 \rangle.
\end{align}
womit diese zweite Annahme bereits bewiesen wäre.
\end{proof}

Eine noch ausführlichere Variante dieses Beweises findet sich im Dokument \cite{cg:online:cgmueller}.

	

\section{Algorithmus\label{cg:section:algorithmus}}
\rhead{Algorithmus}

In diesem Abschnitt wird der ganze Algorithmus noch einmal formell aufgeschrieben.
Diese Formulierung des Algorithmus bildet die Grundlage für eine erfolgreiche Implementation.
Die Resultate einer solchen (einfachen) Implementation folgen im nächsten Abschnitt.

\begin{enumerate}
	\item Wähle initiales $x_1$ zufällig
	\item Berechne die erste Abstiegsrichtung als $d_1 = r_1 =  b-Ax_1$
	\item Berechne die optimale Schrittlänge  $ \alpha	= 	\displaystyle  \frac{\langle d_k , r_k \rangle}{\langle d_k , d_k \rangle_A} 
																			= \frac{d_k^T  r_k}{d_k^T A d_k }$
	\item Führe den Schritt aus $x_{k+1} = x_k + \alpha d_k$
	\item Berechne das neue Residuum $r_{k+1} = b-Ax_{k+1}$
	\item Falls $r_{k+1} = 0$: Beende den Algorithmus
	\item Berechne neue Abstiegsrichtung $d_{k+1} = d_{k+1}	= 	r_{k+1} - \displaystyle \frac{\langle d_k , r_{k+1} \rangle_A}{\langle d_k , d_k \rangle_A} d_k 
															= r_{k+1} - \displaystyle \frac{d_k^T A r_{k+1}}{d_k^T A d_k} d_k $
	\item Wiederhole ab Punkt 3.
\end{enumerate}

\section{Ergebnisse}
\label{cg:section:ergebnisse}
\rhead{Ergebnisse}

%TODO insert own results

\printbibliography[heading=subbibliography]
\end{refsection}
