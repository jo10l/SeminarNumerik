%
% einleitung.tex -- Beispiel-File für die Einleitung
%
% (c) 2020 Prof Dr Andreas Müller, Hochschule Rapperswil
%
\section{Einleitung\label{legendre:section:einleitung}}
\rhead{Einleitung}
\cmt{Einleitung schreiben}

\cmt{Legendre Polynome Definition (Integral)}

\cmt{Es gibt verschiedene Rekursions-Formeln, leider sind nicht alle numerisch stabil.}

\cmt{Wie man erkennen kann, welche davon stabil sind und warum dies so ist, wird in diesem Kapitel beschrieben.}


% Unterkapitel Beispiel
%\subsection{Titel Unterkapitel
%\label{legendre:subsection:unterkapitellabel}}

% Quelle Zitieren Beispiel
%\cite{legendre:bibtex}

% Abschnittsverweis Beispiel
%\ref{legendre:section:loesung}

% Gleichung Beispiel
%\begin{equation}
%\int_a^b x^2\, dx
%=
%\left[ \frac13 x^3 \right]_a^b
%=
%\frac{b^3-a^3}3.
%\label{legendre:equation1}
%\end{equation}

% Gleichung Referenzieren Beispiel
%\eqref{legendre:equation1}


