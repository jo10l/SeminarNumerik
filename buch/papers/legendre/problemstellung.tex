%
% problemstellung.tex -- Beispiel-File für die Beschreibung des Problems
%
% (c) 2020 Prof Dr Andreas Müller, Hochschule Rapperswil
%
\section{Problemstellung
\label{legendre:section:problemstellung}}
\rhead{Problemstellung}
Die zugeordneten Legendrepolynome sind die Lösungen der allgemeinen Legendregleichung (siehe Gleichung \eqref{legendre:legendregleichung}) \cite{legendre:assoc-legendre-poly-wolfram} \cite{legendre:assoc-legendre-diff-wolfram}.
% Legendregleichung
\begin{equation}
(1-x^2) \frac{d^2y}{dx^2}
-2x \frac{dy}{dx}
+ \left[ l(l+1)- \frac{m^2}{1-x^2} \right] y
=0
\label{legendre:legendregleichung}
\end{equation}
Dabei gilt für die beiden Parameter $l$ und $m$, dass $l>=0$ und $m=0, \ldots , l$.
Der Parameter $l$ ist für den Grad des zugeordneten Legendrepolynom verantwortlich, $m$ hingegen für die Ordnung des Polynoms.
Die zugeordneten Legendrepolynome sind zudem nur auf dem Intervall $[-1, 1]$ definiert.
Verwendung finden diese Polynome vor allem als Teil der Kugelflächenfunktionen (\textit{engl.} spherical harmonics) \cite{legendre:spherical-harmonic-wolfram}.

Wie im Abschnitt \ref{legendre:section:einleitung} bereits erwähnt, gibt es Rekursionsbeziehungen, deren Formeln nicht alle numerisch stabil sind.
Der englische Wikipedia-Artikel \cite{legendre:wikipedia} zu den zugeordneten Legendrepolynome führt eine Liste solcher Rekursionsformeln, wobei $P^{m}_{l}$ das Legendrepolynom $l$-ten Grades und $m$-ter Ordnung ist.
Bereits beim Erstellen der Graphen für die ersten beiden Rekursionsformeln des Wikipedia-Artikels ist zu sehen, dass die beiden Formeln nicht die gleichen Resultate liefern.
Vergleiche dazu die Abbildung \cmt{Plot l} mit der Abbildung \cmt{Plot m}.
Die erste Rekursionsformel \eqref{legendre:recurrence-l} verläuft in der Richtung des Parameters $l$ und produziert einen schönen Graphen, wie in Abbildung \cmt{Plot l} gut zu sehen ist.
Das Gegenteil ist bei der Verwendung der zweiten Formel \eqref{legendre:recurrence-m}, welche in der Richtung des Parameters $m$ verläuft, auszumachen.
In Abbildung \cmt{Plot m} ist anhand des gezackten Graphen gut auszumachen, dass hier numerische Instabilitäten auftreten.
% Rekursionsformel in l-Richtung
\begin{equation}
(l-m+1)P^{m}_{l+1}(x)
=(2l+1)xP^{m}_{l}(x)
-(l+m)P^{m}_{l-1}(x)
\label{legendre:recurrence-l}
\end{equation}
% Rekursionsformel in m-Richtung
\begin{equation}
2mxP^{m}_{l}(x)
=-\sqrt{1-x^2}
\left[ P^{m+1}_{l}(x) + (l+m)(l-m+1)P^{m-1}_{l}(x) \right]
\label{legendre:recurrence-m}
\end{equation}

\cmt{Plot l einfügen}
\cmt{Plot m einfügen}

Es stellt sich nun die Frage, wieso die erste Formel \eqref{legendre:recurrence-l} numerisch stabil ist und die zweite Formel \eqref{legendre:recurrence-m} es nicht ist?
Im folgenden Abschnitt \ref{legendre:section:loesung} wird auf diese Frage eingegangen und untersucht, was die Instabilität in der zweiten Formel hervorruft.