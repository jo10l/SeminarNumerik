%
% problemstellung.tex -- Beispiel-File für die Beschreibung des Problems
%
% (c) 2020 Prof Dr Andreas Müller, Hochschule Rapperswil
%
\section{Problemstellung
\label{legendre:section:problemstellung}}
\rhead{Problemstellung}
Die zugeordneten Legendrepolynome (\textit{engl.} associated Legendre polynomials \cmt{korrekt so?}) sind die Lösungen der allgemeinen Legendregleichung (siehe Gleichung \eqref{legendre:legendregleichung}) \cite{legendre:assoc-legendre-poly-wolfram} \cite{legendre:assoc-legendre-diff-wolfram}.
% Legendregleichung
\begin{equation}
(1-x^2) \frac{d^2y}{dx^2}
-2x \frac{dy}{dx}
+ \left[ l(l+1)- \frac{m^2}{1-x^2} \right] y
=0
\label{legendre:legendregleichung}
\end{equation}
Dabei gilt für die beiden Parameter $l$ und $m$, dass $l>=0$ und $m=0, \ldots , l$.
Der Parameter $l$ ist für den Grad des zugeordneten Legendrepolynom verantwortlich, $m$ hingegen für die Ordnung des Polynoms.
Die zugeordneten Legendrepolynome sind zudem nur auf dem Intervall $[-1, 1]$ definiert.
Verwendung finden diese Polynome vor allem als Teil der Kugelflächenfunktionen (\textit{engl.} spherical harmonics \cmt{korrekt so?}) \cite{legendre:spherical-harmonic-wolfram}.
Um ein zugeordnetes Legendrepolynom mit einem bestimmten $l$ und einem bestimmten $m$ zu evaluieren, wird aus Aufwandsgründen oft auf eine Rekursionsbeziehung (\textit{engl.} recurrence relations \cmt{korrekt so?}) zurückgegriffen.
Für die zugeordneten Legendrepolynome gibt es mehrere solche Rekursionsbeziehungen.
Die englische Wikipedia-Seite zu den zugeordneten Legendrepolynome führt beispielsweise eine ganze Liste mit solchen Rekursionsbeziehungen \cite{legendre:wikipedia}.
Im folgenden werden die ersten beiden Rekursionsgleichungen, die der englische Wikipedia-Artikel auflistet, auf numerische Instabilität untersucht (siehe Gleichungen \eqref{legendre:recurrence-l} und \eqref{legendre:recurrence-m}).
% Rekursionsformel in l-Richtung
\begin{equation}
(l-m+1)P^{m}_{l+1}(x)
=(2l+1)xP^{m}_{l}(x)
-(l+m)P^{m}_{l-1}(x)
\label{legendre:recurrence-l}
\end{equation}
% Rekursionsformel in m-Richtung
\begin{equation}
2mxP^{m}_{l}(x)
=-\sqrt{1-x^2}
\left[ P^{m+1}_{l}(x) + (l+m)(l-m+1)P^{m-1}_{l}(x) \right]
\label{legendre:recurrence-m}
\end{equation}