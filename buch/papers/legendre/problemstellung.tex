%
% problemstellung.tex -- Beispiel-File für die Beschreibung des Problems
%
% (c) 2020 Prof Dr Andreas Müller, Hochschule Rapperswil
%
\section{Problemstellung
\label{legendre:section:problemstellung}}
\rhead{Problemstellung}
\cmt{Problemstellung schreiben}

\cmt{(Zugeordnete) Legendre Polynome sind durch folgendes Integral definiert...}

\cmt{l ist Grad und m ist Ordnung.}

\cmt{Um diese Polynome via Computer zu evaluieren wird oft (wegen teuer) auf eine Rekursionsformel zurückgegriffen.}

\cmt{Englischer Wikipedia Artikel listet eine ganze Reihe davon.}
\cmt{Referenz Wikipedia Artikel}

\cmt{Leider sind davon nicht alle numerisch stabil.}

\cmt{Im Folgenden werden die ersten beiden Rekursionsformel der Wikipedia Seite unter die Lupe genommen und auf ihre numerische Stabilität untersucht.}

\cmt{Die beiden Rekursionsformeln einbinden.}


